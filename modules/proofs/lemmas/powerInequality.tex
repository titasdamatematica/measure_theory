\begin{proof}
    Considere, para cada $0 < \alpha < 1$, a função $f_\alpha : (0, \infty) \to \mathbb{R}$ dada por $f_\alpha(t) = t^\alpha - \alpha t$. Temos
    \begin{equation*}
        f'_\alpha(t) = \alpha t^{\alpha - 1} - \alpha = \alpha(t^{\alpha - 1} - 1).
    \end{equation*}
    Logo,
    \begin{equation*}
        f'_\alpha(t) > 0 \text{ se } 0 < t < 1
        \quad \text{e} \quad
        f'_\alpha(t) < 0 \text{ se } t > 1,
    \end{equation*}
    e $f_\alpha$ tem um máximo em $t = 1$. Portanto,
    \begin{equation*}
        f_\alpha(t) \leq f_\alpha(1)
    \end{equation*}
    para todo $t > 0$, e
    \begin{equation*}
        t^\alpha \leq \alpha t + (1 - \alpha).
    \end{equation*}

    Fazendo $t = \frac{a}{b}$ e $\alpha = \frac{1}{p}$, temos
    \begin{equation*}
        \left( \frac{a}{b} \right)^{\frac{1}{p}} \leq \frac{1}{p} \frac{a}{b} + \left(1 - \frac{1}{p}\right).
    \end{equation*}
    Multiplicando a desigualdade acima por $b$, obtemos
    \begin{equation*}
        a^{\frac{1}{p}} b^{\frac{1}{q}} \leq \frac{a}{p} + \frac{b}{q},
    \end{equation*}
    o que demonstra o lema.
\end{proof}