\begin{proof}
    Vamos denotar por $\mathcal{F}$ a família de conjuntos do enunciado. Considere $\mathcal{C}$ a família como na Definição \ref{def:product_sigma_algebra}. Assim,
    \begin{eqnarray*}
        \mathcal{F} &\coloneq& \left\{\pi_i^{-1}(E_i); E_i\in \mathcal{C}_i, i\in I\right\}\\
        \mathcal{C} &\coloneq& \left\{\pi_i^{-1}(E_i); E_i\in \Sigma_i, i\in I\right\}
    \end{eqnarray*}
    
    Pretendemos mostrar que  $\bigotimes_{i\in I} \Sigma_i=\generateSigmaAlg{\mathcal{C}}=\generateSigmaAlg{\mathcal{F}}$. Uma forma natural de chegar nesse resultado é mostrar que $\generateSigmaAlg{\mathcal{C}}\subseteq\generateSigmaAlg{\mathcal{F}}$ e $\generateSigmaAlg{\mathcal{F}}\subseteq\generateSigmaAlg{\mathcal{C}}$.
    
    Por um lado, $\mathcal{F}\subset \mathcal{C} \subset \generateSigmaAlg{\mathcal{C}}$. Usando o Lema \ref{lmm:sigma_algebra_generated_by_subset}, mostramos que $\generateSigmaAlg{\mathcal{F}}\subseteq\generateSigmaAlg{\mathcal{C}}$.
    
    Vamos mostrar o outro lado da continência. A estratégia aqui é construir uma \sigmaAlg que esteja contida em $\generateSigmaAlg{\mathcal{F}}$. Fixe um $i\in\mathcal{I}$ e considere a família de conjuntos 

    \begin{equation*}
        \mathcal{A}_i=\{E\subseteq X_i; \pi_{i}^{-1}(E)\in \generateSigmaAlg{\mathcal{F}}\}.
    \end{equation*}

    Note que $\mathcal{A}_i$ é uma \sigmaAlg. De fato, se $(A_n)_{n}^{\infty}\subseteq \mathcal{A}_i$ é uma sequência enumerável de conjuntos da família $\mathcal{A}_i$, então
    
    \begin{equation*}
        \pi_{i}^{-1}\left(\bigcup_{n=1}^{\infty} A_n\right) = \bigcup_{n=1}^{\infty} \pi_{i}^{-1}(A_n).
    \end{equation*}

    Como $\pi_{i}^{-1}(A_n)\in\generateSigmaAlg{\mathcal{F}}$ por hipótese, então a união também pertence a $\generateSigmaAlg{\mathcal{F}}$.

    Para mostrar que esta família é fechada pelo complemento, o argumento é parecido. Tome $A\in\mathcal{A}_i$, então

    \begin{equation*}
        \pi_{i}^{-1}\left(\complement(A)\right) = \complement\left(\pi_{i}^{-1}(A)\right)\in\generateSigmaAlg{\mathcal{F}}.
    \end{equation*}

    Portanto, $\mathcal{A}_i$ é uma \sigmaAlg em $X_i$ para cada $i\in\mathcal{I}$. Observe que cada $E_i\in\mathcal{C}_i$ é um conjunto tal que $\pi_i^{-1}(E_i)\in\generateSigmaAlg{\mathcal{F}}$. Portanto, cada $E_i\in\mathcal{A}_i$ e $\mathcal{C}_i\subset \mathcal{A}_i$. No entanto, pela definição de \nameref{def:generated_sigma_algebra}, temos que $\generateSigmaAlg{\mathcal{C}_i}=\Sigma_i\subseteq\mathcal{A}_i$ para todo $i\in\mathcal{I}$. Isto nos diz que todo $E\in\Sigma_i$ é tal que $\pi_{i}^{-1}(E)\in\generateSigmaAlg{\mathcal{F}}$ para todo $i\in\mathcal{I}$. Portanto, $\generateSigmaAlg{\mathcal{F}}\supseteq \generateSigmaAlg{\mathcal{C}}$.

    Em conclusão,

    \begin{equation*}
        \bigotimes_{i\in I} \Sigma_i=\generateSigmaAlg{\mathcal{C}}=\generateSigmaAlg{\mathcal{F}}=\generateSigmaAlg{\left\{\pi_i^{-1}(E_i); E_i\in \mathcal{C}_i, i\in I\right\}}.
    \end{equation*}

\end{proof}