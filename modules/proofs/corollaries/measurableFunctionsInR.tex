\begin{proof}
    Seja $f:(X,\Sigma)\rightarrow (\R,\mathcal{B})$ uma função mensurável e $\alpha$ um número real. Então $f^{-1}((\alpha, \infty))\in \Sigma$, isto é, $A_{\alpha}\in \Sigma$. Pela arbitrariedade da escolha de $\alpha$, vale a ida da proposição.

    Reciprocamente, seja $f:(X,\Sigma)\rightarrow (\R,\mathcal{B})$ uma função tal que, para todo $\alpha$ real, $A_{\alpha} \in \Sigma$. Pelo Exercício XXX, $\mathcal{B}$ pode ser gerado pelos intervalos na forma $(\alpha,\infty)$. Assim, pela Proposição \ref{prop:measurable_functions_and_sigma_algebras_generated_by_set}, só precisamos verificar os intervalos da forma $(\alpha,\infty)$, que pertencem a $\Sigma$  por hipótese. Logo, $f$ é mensurável.
\end{proof}