%%%%%%%%%%%%%%%%%%%%%%%%%%%%%%%%%
%%%%%%%%%%%%%%%%%%%%%%%%%%%%%%%%%

% Seção 2 - Modos de Convergência

%%%%%%%%%%%%%%%%%%%%%%%%%%%%%%%%%
%%%%%%%%%%%%%%%%%%%%%%%%%%%%%%%%%






\chapter{Modos de Convergência}

\section{Modos de Convergência}

\subsection{Convergência de funções}

Já conhecemos o sentido usual de convegência de funções, quando olhamos para a sequência aplicada a um ponto $x$. As principais noções de convergência estão listadas abaixo, da mais fraca para a mais forte.

\aeConvergence
\pointwiseConvergence
\uniformConvergence

\subsection{Convergência em \texorpdfstring{$L_p$}{Lp}}

\LpConvergence

Estudar a convergência de funções em $L_p$ nem sempre é intuitivo. Não podemos garantir que o limite está em $L_p$ apenas com a convergência uniforme. Pelo menos não no caso geral. Veja os exemplos abaixo.

\counterExampleUniformConvergence

Vamos estudar algumas restrições que podemos impor para garantir a convergência em $L_p$.

\uniformConvergenceLp

\dominatedConvergenceLp

\boundedConvergenceLp