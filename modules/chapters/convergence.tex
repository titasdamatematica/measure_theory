%%%%%%%%%%%%%%%%%%%%%%%%%%%%%%%%%
%%%%%%%%%%%%%%%%%%%%%%%%%%%%%%%%%

% Seção 2 - Modos de Convergência

%%%%%%%%%%%%%%%%%%%%%%%%%%%%%%%%%
%%%%%%%%%%%%%%%%%%%%%%%%%%%%%%%%%






\chapter{Modos de Convergência}

\section{Modos de Convergência}

\subsection{Convergência de funções}

Já conhecemos o sentido usual de convegência de funções, quando olhamos para a sequência aplicada a um ponto $x$. As principais noções de convergência estão listadas abaixo, da mais fraca para a mais forte.

\aeConvergence
\pointwiseConvergence
\uniformConvergence

\subsection{Convergência em \texorpdfstring{$L_p$}{Lp}}

\LpConvergence

Estudar a convergência de funções em $L_p$ nem sempre é intuitivo. Não podemos garantir que o limite está em $L_p$ apenas com a convergência uniforme. Pelo menos não no caso geral. Veja os exemplos abaixo.

\counterExampleUniformConvergence

\counterExamplePointwiseConvergence

\counterExampleAEConvergence

\counterExampleLpConvergence

Como podemos ver, não existe nenhuma relação clara entre esse dois modos de convergência. Vamos estudar algumas restrições que podemos impor para garantir a convergência em $L_p$ a partir de outras formas de convergência que já conhecemos.

\uniformConvergenceLp

\dominatedConvergenceLp

\boundedConvergenceLp

\subsection{Convergência na Medida}

Vamos estudar uma outra noção de convergência que precede a ideia da convergência em $L_p$, no sentido de que precisa de menos estrutura, mas que, como veremos a seguir, cria condições para garantir uma convergência boa em $L_p$.

\measureConvergence

Da mesma forma que existem as sequências de Cauchy, definiremos as sequências que são de Cauchy na medida.

\measureCauchyConvergence

Com estas novas definições, ganhamos relação mais próxima com a convergência em $L_p$.

\convergentInLpConvergentInMeasure

\cauchyInMeasure

\cauchyMeasureCorollary

\convergenceInMeasureLpConvergence

\section{Teoremas de Convergência (Parte 2)}

Vamos começar com mais uma noção de convergência

\almostUniformConvergence

\subsection{Teorema de Egoroff}

\almostUniformCauchyLemma

\almostUniformConvergenceProposition

\egoroff

