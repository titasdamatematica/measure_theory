% sections/introduction.tex
\chapter*{Sobre Este Material}

Estas notas foram escritas ao longo de 2024 para complementar as reuniões da Iniciação Científica. Referem-se principalmente ao livro ``The Elements of Integration and Lebesgue Measure" de Robert G. Bartle e ``Real Analysis: Modern Techniques and Their Applications'' de Geral B. Folland. Além disso, foram usadas como referências as notas de aula de Daniel Pellegrino da UFPB e Daniel Tausk da USP.

A motivação para estas notas é fornecer uma compreensão clara e concisa dos conceitos fundamentais de Medida e Integração, que são cruciais para várias áreas da matemática e suas aplicações.

\section*{Objetivos}

Os principais objetivos destas notas são:

\begin{enumerate}
    \item Complementar o material apresentado nas reuniões da Iniciação Científica.
    \item Fornecer uma base sólida para estudos mais avançados.
    \item Facilitar a compreensão dos conceitos através de teoremas e demonstrações bem detalhadas.
    \item Servir como um recurso de referência para futuros estudos e pesquisas na área.
\end{enumerate}

\section*{Estrutura das Notas}

Estas notas estão organizadas em capítulos que seguem a estrutura lógica do estudo de Medida e Integração adotada pelo Bartle. Em alguns momentos, a ordem foi substituída por fins didáticos.

Estas notas assumem o domínio de alguns assuntos como Teoria dos Conjuntos, Análise da Reta e noções de topologia.

A fim de tornar a leitura mais dinâmica, conhecimentos necessários foram colocados na Seção XXX, as demonstrações mais extensas foram movidas para a Seção XXX e soluções para vários exercícios podem ser encontradas na Seção XXX.

\section*{Versionamento}
Estas notas estão na sua versão $0.1$. Elas cobrem aproximadamente $50$\% do curso de medida.
