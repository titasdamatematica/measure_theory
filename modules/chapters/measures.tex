\chapter{Medidas}

O nosso objetivo é generalizar a idea de volumes. De forma bem superficial, existe um problema com a nossa intuição de ``mensurabilidade''. Em particular, existem conjuntos bem patológicos que não conseguimos ``medir''. Uso aspas aqui para explicitar a informalidade do termo até aqui. Para resolver o problema mencionado, podemos construir uma estrutura chamada de \texorpdfstring{$\sigma$}{sigma}-álgebra que definirá o conceito de mensurabilidade.

Com esta estrutura em maõs, vamos discutir algumas das suas propriedades e estudar a sua relação com a topologia de um espaço. Depois disso, construiremos uma função que chamaremos de medida. Esta é a função que vai associar um número a cada conjunto mensurável. Na prática, esta será a nossa generalização para volumes. Por fim, trataremos dos teoremas que nos permitem estender medidas ``ruins'' para criar medidas bem comportadas.

Dessa forma, podemos resumir os conceitos mais importantes deste capítulo na seguinte lista:

\begin{enumerate}
    \item Espaços Mensuráveis
    \item Medidas
    \item Teoremas de Extensão
    \item Construção de Medidas
\end{enumerate}

\section{Estruturas Algébricas}
Como prometido, vamos definir o que são as \textbf{\texorpdfstring{$\sigma$}{sigma}-álgebras}. Poderíamos passar o dia todo falando de estruturas algébricas que antecedem esta construção, mas, para os nossos fins, tudo o que precisamos saber é que uma \texorpdfstring{$\sigma$}{sigma}-álgebra de um conjunto é uma família de subconjuntos \textit{fechada pelas operações de complemento e união enumerável. Também pedimos que o conjunto todo pertença a esta família.} Veja a definição rigorosa abaixo.

\sigmaAlgebra

Algumas propriedades são importante, como, por exemplo, o fato de que as \texorpdfstring{$\sigma$}{sigma}-álgebras são \textit{fechadas por interseção enumerável.} Para mostrar isso, basta aplicar as \nameref{thm:de_morgan}. Desse conjunto extenso de propriedades vindas da teoria dos conjuntos, precisamos de duas, que estão enunciadas abaixo.

\deMorgan

Vamos ver vários exemplos de \texorpdfstring{$\sigma$}{sigma}-álgebras. Pule se já estiver familiarizado com o assunto. No apêndice de exercícios resolvidos é provado que cada um destes exemplos é, de fato, uma \texorpdfstring{$\sigma$}{sigma}-álgebra.

\trivialSigmaAlgebra
\powerSetSigmaAlgebra
\countableSetsSigmaAlgebra

Para partir para problemas mais interessantes precisamos de uma forma de criar \texorpdfstring{$\sigma$}{sigma}-álgebras. Fazemos isso com o conceito de \textbf{\texorpdfstring{$\sigma$}{sigma}-algebra gerada por conjuntos.} Intuitivamente, esta será a menor $\sigma$-álgebra que contém os elementos. Formalizamos esta ideia com a definição abaixo.

\generatedSigmaAlgebra

A definição acima pode ser bem abstrata e difícil de trabalhar. Assim, introduziremos uma caracterização mais tangível. A fim de chegar nesta caracterização, provemos dois lemas.

\sigmaAlgebraIntersection
\generatedSigmaAlgebraIsUnique

Com esta ferramenta em mãos, vamos introduzir a \nameref{lmm:generated_sigma_algebra_characterization}.

\generatedSigmaAlgebraCharacterization

O resultado abaixo é bem direto, mas útil.

\sigmaAlgebraGeneratedBySubset