\chapter{Os Espaços de Lebesgue}



%%%%%%%%%%%%%%%%%%%%%%%%%%%%%%%%%
%%%%%%%%%%%%%%%%%%%%%%%%%%%%%%%%%

%  Seção 1 - Espaços Lp

%%%%%%%%%%%%%%%%%%%%%%%%%%%%%%%%%
%%%%%%%%%%%%%%%%%%%%%%%%%%%%%%%%%







\section{Espaços \texorpdfstring{$L^p$}{Lp}}

O nosso objetivo neste capítulo é construir um espaço de Banach com as funções. Para tanto, vamos mostrar que o espaço das funções é um espaço vetorial, introduzir uma norma e, por fim, mostrar que ele é um espaço completo.

Vamos começar introduzindo uma \nameref{def:semi_norm}.

\functionSeminorm

Definimos uma função integrável como aquela que tem norma finita. Isto dá origem aos nossos espaços $\mathcal{L}_p$.

\spaceLpSeminorm

Não é difícil verificar que este é, de fato, um espaço vetorial equipa com uma semi-norma. Um grande incoveniente dos espaços $\mathcal{L}_p$ é que ele não diferencia funções, isto é, $\norm{\cdot}$ não é uma norma. Vamos contornar este problema com uma construção inteligente.

\equivalenceRelationLp

Vamos definir o conjunto de todas as funções mensuráveis que são equivalentes a $f$, ou seja, funções que diferem de $f$ apenas em um conjunto de medida nula. Com este truque, colocamos todas as funções iguais quase sempre em um mesmo bolo, eliminando o nosso problemas inicial.

\equivalenceClassOfFunction

A partir desta relação, identificamos a norma de uma classe de equivalência com a norma de um representante daquela classe.

\LpSpace

Tratamos o caso em que $p=+\infty$ de forma análoga. Definimos a norma como a norma do supremo.

\functionInftySeminorm


%%%%%%%%%%%%%%%%%%%%%%%%%%%%%%%%%
%%%%%%%%%%%%%%%%%%%%%%%%%%%%%%%%%

%      Seção 2 - Integração

%%%%%%%%%%%%%%%%%%%%%%%%%%%%%%%%%
%%%%%%%%%%%%%%%%%%%%%%%%%%%%%%%%%






\section{Modos de Convergência}


%%%%%%%%%%%%%%%%%%%%%%%%%%%%%%%%%
%%%%%%%%%%%%%%%%%%%%%%%%%%%%%%%%%

%  Seção 3 - Teoremas de Convergência

%%%%%%%%%%%%%%%%%%%%%%%%%%%%%%%%%
%%%%%%%%%%%%%%%%%%%%%%%%%%%%%%%%%






\section{Teoremas de Decomposição}