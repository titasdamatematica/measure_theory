\chapter{Os Espaços de Lebesgue}



%%%%%%%%%%%%%%%%%%%%%%%%%%%%%%%%%
%%%%%%%%%%%%%%%%%%%%%%%%%%%%%%%%%

%  Seção 1 - Espaços Lp

%%%%%%%%%%%%%%%%%%%%%%%%%%%%%%%%%
%%%%%%%%%%%%%%%%%%%%%%%%%%%%%%%%%







\section{Espaços \texorpdfstring{$L^p$}{Lp}}

\subsection{Construção de um espaço vetorial normado}

O nosso objetivo neste capítulo é construir um espaço de Banach com as funções. Para tanto, vamos mostrar que o espaço das funções é um espaço vetorial, introduzir uma norma e, por fim, mostrar que ele é um espaço completo.

Vamos começar introduzindo uma semi-norma.

\functionSeminorm

Definimos uma função integrável como aquela que tem norma finita. Isto dá origem aos nossos espaços $\mathcal{L}_p$.

\spaceLpSeminorm

Não é difícil verificar que este é, de fato, um espaço vetorial equipa com uma semi-norma. Um grande incoveniente dos espaços $\mathcal{L}_p$ é que ele não diferencia funções, isto é, $\norm{\cdot}$ não é uma norma. Vamos contornar este problema com uma construção inteligente.

\equivalenceRelationLp

Vamos definir o conjunto de todas as funções mensuráveis que são equivalentes a $f$, ou seja, funções que diferem de $f$ apenas em um conjunto de medida nula. Com este truque, colocamos todas as funções iguais quase sempre em um mesmo bolo, eliminando o nosso problemas inicial.

\equivalenceClassOfFunction

A partir desta relação, identificamos a norma de uma classe de equivalência com a norma de um representante daquela classe.

\LpSpace

Tratamos o caso em que $p=+\infty$ de forma análoga. Definimos a norma como a norma do supremo.

\functionInftySeminorm

Para que esta seminorma vire uma norma, fazemos a mesma construção via classes de equivalência.

\LinftySpace

\subsection{Desigualdades de Hölder e Minkowski}

As desigualdades de Hölder e Minkowski são fundamentais para o estudo dos espaços $L_p$ e suas propriedades. Em particular, a desigualdade de Minkowski é a prova que $\norm{\cdot}_p$ satisfaz a deigualdade triangular.

Para provar a desigualdade de Hölder, vamos provar o lema abaixo.

\powerInequality

\holder

\minkowski

\subsection{Teorema de Riesz-Fischer}

Vamos relembrar o nosso objetivo. O que foi dito no começo do capítulo é que queremos mostrar que os espaços $L_p$ são espaços de Banach. Assim, vamos definir exatamente o que queremos e provar o Teorema de Riesz-Fischer, essencial para provar este fato.

\banachSpace

No nosso context, temos um espaço vetorial normado, $L_p$. Vamos definir nossas sequências de Cauchy.

\cauchySequenceLp

De fato, o conceito de sequência de Cauchy é mais fraco que o de sequência convergente. Um resultado geral sobre espaços métricos mostra que toda sequência convergente é de Cauchy. No entanto, a recíproca nem sempre é verdadeira. Os espaços completos são aqueles onde as duas definições são equivalentes: uma sequência é convergente se, e somente se, ela for de Cauchy.

O teorema de Riesz-Fischer garante que o espaço $L^p$ é completo, ou seja, que toda sequência de Cauchy em $L^p$ converge para um elemento em $L^p$.

\rieszFischer



%%%%%%%%%%%%%%%%%%%%%%%%%%%%%%%%%
%%%%%%%%%%%%%%%%%%%%%%%%%%%%%%%%%

%  Seção 3 - Teoremas de Convergência

%%%%%%%%%%%%%%%%%%%%%%%%%%%%%%%%%
%%%%%%%%%%%%%%%%%%%%%%%%%%%%%%%%%






\section{Teoremas de Decomposição}