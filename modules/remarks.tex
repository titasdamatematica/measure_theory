\newcommand{\subsetsMayNotBeMeasurable}{
    \begin{remark}{Subconjuntos podem não ser mensuráveis}{subsets_may_not_be_measurable}
        Seja $N\in \Sigma$ um conjunto de medida nula. Então, todo $A\subset N$, pela monotonicidade da medida, também tem medida nula. No entanto, $A$ não precisa pertencer a $\Sigma$.
    \end{remark}
}

\newcommand{\measurableFunctionsNotation}{
    \begin{remark}{Notação para funções mensuráveis	}{measurable_functions_notation}
        Denotaremos por $M(X,\Sigma)$ o conjunto das funções reais $\Sigma$-mensuráveis. O conjunto $M^{+}(X,\Sigma)$ é o conjunto das funções mensuráveis não negativas.
    \end{remark}
}