\newcommand{\trivialSigmaAlgebra}{
    \begin{example}{\texorpdfstring{$\sigma$}{sigma}-álgebra Trivial}{trivial_sigma_algebra}
        Seja $X$ um conjunto. Então $\Sigma=\{\varnothing, X\}$ é uma \nameref{def:sigma_algebra} conhecida como a \textbf{\texorpdfstring{$\sigma$}{sigma}-álgebra trivial de $X$}.
    \end{example}
}

\newcommand{\powerSetSigmaAlgebra}{
    \begin{example}{\texorpdfstring{$\sigma$}{sigma}-álgebra das Partes}{power_set_sigma_algebra}
        Seja $X$ um conjunto. Então $\Sigma=\powerset{X}$ é uma \nameref{def:sigma_algebra} conhecida como a \textbf{\texorpdfstring{$\sigma$}{sigma}-álgebra das partes de $X$}.
    \end{example}
}

\newcommand{\countableSetsSigmaAlgebra}{
    \begin{example}{\texorpdfstring{$\sigma$}{sigma}-álgebra dos Conjuntos Enumeráveis}{countable_sets_sigma_algebra}
        Seja $X$ um conjunto não enumerável. Então
        \begin{equation*}
            \Sigma=\{E\in X; \ E\text{ é enumerável ou } \complement(E) \text{ é enumerável}\}
        \end{equation*}
        é uma \nameref{def:sigma_algebra} conhecida como a \textbf{\texorpdfstring{$\sigma$}{sigma}-álgebra dos conjuntos enumeráveis e co-enumeráveis de $X$}.
    \end{example}
}