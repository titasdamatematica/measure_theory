\newcommand{\trivialSigmaAlgebra}{
    \begin{example}{\texorpdfstring{$\sigma$}{sigma}-álgebra Trivial}{trivial_sigma_algebra}
        Seja $X$ um conjunto. Então $\Sigma=\{\varnothing, X\}$ é uma \nameref{def:sigma_algebra} conhecida como a \textbf{\texorpdfstring{$\sigma$}{sigma}-álgebra trivial de $X$}.
    \end{example}
}

\newcommand{\powerSetSigmaAlgebra}{
    \begin{example}{\texorpdfstring{$\sigma$}{sigma}-álgebra das Partes}{power_set_sigma_algebra}
        Seja $X$ um conjunto. Então $\Sigma=\powerset{X}$ é uma \nameref{def:sigma_algebra} conhecida como a \textbf{\texorpdfstring{$\sigma$}{sigma}-álgebra das partes de $X$}.
    \end{example}
}

\newcommand{\countableSetsSigmaAlgebra}{
    \begin{example}{\texorpdfstring{$\sigma$}{sigma}-álgebra dos Conjuntos Enumeráveis}{countable_sets_sigma_algebra}
        Seja $X$ um conjunto não enumerável. Então
        \begin{equation*}
            \Sigma=\{E\in X; \ E\text{ é enumerável ou } \complement(E) \text{ é enumerável}\}
        \end{equation*}
        é uma \nameref{def:sigma_algebra} conhecida como a \textbf{\texorpdfstring{$\sigma$}{sigma}-álgebra dos conjuntos enumeráveis e co-enumeráveis de $X$}.
    \end{example}
}

\newcommand{\countingMeasure}{
    \begin{example}{Medida de Contagem}{counting_measure}
        Seja $(X,\Sigma)$ um \nameref{def:measure_space} onde $X$ é um conjunto não enumerável e $\Sigma$ é a \nameref{exm:countable_sets_sigma_algebra}. A função
    \begin{equation*}
        \mu (E)=
        \begin{cases}
            \#E, \text{ se } E \text{ é finito}, \\
            +\infty, \text{ se } E \text{ é infinito},
        \end{cases}
    \end{equation*}
    é uma medida em $X$.
    \end{example}
}

\newcommand{\diracMeasure}{
    \begin{example}{Medida de Dirac}{dirac_measure}
        Sejam $(X,\powerset{X})$ um \nameref{def:measure_space} e $x_0\in X$ um ponto fixado. A função
    \begin{equation*}
        \mu (E)=
        \begin{cases}
            1, \text{ se } x_0\in E, \\
            0, \text{ se } x_0\not\in E,
        \end{cases}
    \end{equation*}
    é uma medida em $X$.
    \end{example}
}

\newcommand{\constantFunction}{
    \begin{example}{Função Constante}{constant_function}
        Seja $(X,\Sigma)$ um espaço mensurável e $f:(X,\Sigma)\rightarrow (\R,\mathcal{B})$ uma função constante. Então $f$ é uma função mensurável.
    \end{example}
}

\newcommand{\characteristicFunction}{
    \begin{example}{Função Característica}{characteristic_function}
        Seja $(X,\Sigma)$ um espaço mensurável, $E\in \Sigma$ um conjunto mensurável e $\chi_{E}:(X,\Sigma)\rightarrow (\R,\mathcal{B})$ uma função definida por:
    \begin{equation*}
        \chi_{E}(x)=
        \begin{cases}
            1, x\in E \\
            0, x\in \complement(E)
        \end{cases}.
    \end{equation*}
    \end{example}
}

\newcommand{\counterExampleUniformConvergence}{
    \begin{example}{Convergência Uniforme \texorpdfstring{$\not\Rightarrow$}{->} Convergência em $L_p$}{counter_example_uniform_convergence}
        Seja $f_n:x\in X\mapsto n^{-1}\chi_{(0,n)}$ para todo $n\in\N$. Esta sequência converge uniformemente para $0$ pois, para todo $\varepsilon>0$ existe, pelo princípio de arquimedes, $N(\varepsilon) \in \N$ tal que $0<N^{-1}<\varepsilon$. Se $n\geq N$, então $n^{-1}<\varepsilon$. Por fim, note que , para todo $x\in X$, $f_n(x)\leq n^{-1}<\varepsilon$, o que mostra a converência uniforme. No entanto, $\norm{f_n}_p=1$, logo $\lim \norm{f_n-0}_p \neq 0$ e $f_n$ não pode convergir para $0$ em $L_p$.
    \end{example}
}

