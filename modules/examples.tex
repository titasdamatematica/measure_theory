\newcommand{\trivialSigmaAlgebra}{
    \begin{example}{\texorpdfstring{$\sigma$}{sigma}-álgebra Trivial}{trivial_sigma_algebra}
        Seja $X$ um conjunto. Então $\Sigma=\{\varnothing, X\}$ é uma \nameref{def:sigma_algebra} conhecida como a \textbf{\texorpdfstring{$\sigma$}{sigma}-álgebra trivial de $X$}.
    \end{example}
}

\newcommand{\powerSetSigmaAlgebra}{
    \begin{example}{\texorpdfstring{$\sigma$}{sigma}-álgebra das Partes}{power_set_sigma_algebra}
        Seja $X$ um conjunto. Então $\Sigma=\powerset{X}$ é uma \nameref{def:sigma_algebra} conhecida como a \textbf{\texorpdfstring{$\sigma$}{sigma}-álgebra das partes de $X$}.
    \end{example}
}

\newcommand{\countableSetsSigmaAlgebra}{
    \begin{example}{\texorpdfstring{$\sigma$}{sigma}-álgebra dos Conjuntos Enumeráveis}{countable_sets_sigma_algebra}
        Seja $X$ um conjunto não enumerável. Então
        \begin{equation*}
            \Sigma=\{E\in X; \ E\text{ é enumerável ou } \complement(E) \text{ é enumerável}\}
        \end{equation*}
        é uma \nameref{def:sigma_algebra} conhecida como a \textbf{\texorpdfstring{$\sigma$}{sigma}-álgebra dos conjuntos enumeráveis e co-enumeráveis de $X$}.
    \end{example}
}

\newcommand{\countingMeasure}{
    \begin{example}{Medida de Contagem}{counting_measure}
        Seja $(X,\Sigma)$ um \nameref{def:measure_space} onde $X$ é um conjunto não enumerável e $\Sigma$ é a \nameref{exm:countable_sets_sigma_algebra}. A função
    \begin{equation*}
        \mu (E)=
        \begin{cases}
            \#E, \text{ se } E \text{ é finito}, \\
            +\infty, \text{ se } E \text{ é infinito},
        \end{cases}
    \end{equation*}
    é uma medida em $X$.
    \end{example}
}

\newcommand{\diracMeasure}{
    \begin{example}{Medida de Dirac}{dirac_measure}
        Sejam $(X,\powerset{X})$ um \nameref{def:measure_space} e $x_0\in X$ um ponto fixado. A função
    \begin{equation*}
        \mu (E)=
        \begin{cases}
            1, \text{ se } x_0\in E, \\
            0, \text{ se } x_0\not\in E,
        \end{cases}
    \end{equation*}
    é uma medida em $X$.
    \end{example}
}
