% modules\theorems.tex

%%%%%%%%%%%%%%%%%%%%%%%%%%%%%%%%%%%%%%%%%%%%%%%%%%%%
%FIRST CHAPTER
%%%%%%%%%%%%%%%%%%%%%%%%%%%%%%%%%%%%%%%%%%%%%%%%%%%%

% first chapter lemmas
\newcommand{\sigmaAlgebraIntersection}{
    \begin{lemma}{Interseção de \texorpdfstring{$\sigma$}{sigma}-álgebras é \texorpdfstring{$\sigma$}{sigma}-álgebra}{sigma_algebra_intersection}
        Seja $X$ um conjunto e $\{\Sigma_i\}_{i\in\mathcal{I}}$ uma família de \nameref{def:sigma_algebra} de $X$. Então, $\cap_{i\in\mathcal{I}} \Sigma_i $ é uma \texorpdfstring{$\sigma$}{sigma}-álgebra de $X$.
    \end{lemma}
}

\newcommand{\generatedSigmaAlgebraIsUnique}{
    \begin{lemma}{Unicidade da \texorpdfstring{$\sigma$}{sigma}-álgebra gerada}{generated_sigma_algebra_is_unique}
        Seja $X$ um conjunto e $\mathcal{C}\subset \powerset{X}$. Então, a \nameref{def:generated_sigma_algebra} por $\mathcal{C}$ é única.
    \end{lemma}
}

\newcommand{\sigmaAlgebraGeneratedBySubset}{
    \begin{lemma}{\texorpdfstring{$\sigma$}{sigma}-álgebra Gerada Por Subconjunto}{sigma_algebra_generated_by_subset}
        Seja $\mathcal{C}\in \powerset{X}$. Se $E\in \sigma [\mathcal{C}]$, então $\sigma [E] \subset \sigma [\mathcal{C}]$.
    \end{lemma}
}

% first chapter propositions
\newcommand{\generatedSigmaAlgebraCharacterization}{
    \begin{proposition}{Caracterização da \texorpdfstring{$\sigma$}{sigma}-álgebra gerada}{generated_sigma_algebra_characterization}
        Sejam $X$ um conjunto, $\mathcal{C}\subset \powerset{X}$ e $\mathcal{J}=\{\Sigma_j\}$ a família de todas as $\sigma$-álgebras de $X$ que contém $\mathcal{C}$. Então,
        \begin{equation*}
            \sigma [\mathcal{C}] = \bigcap_{\Sigma_j\in\mathcal{J}} \Sigma_{j}.
        \end{equation*}
    \end{proposition}
}

% first chapter theorems
\newcommand{\deMorgan}{
    \begin{theorem}{Leis de De Morgan}{de_morgan}
        Seja $X$ um conjunto e $\{O_i\}_{i\in \mathcal{I}}\subset \powerset{X}$. Então,
        \begin{enumerate}
            \item \begin{equation*}
                \complement \left(\bigcup_{i\in\mathcal{I}}O_i\right) = \bigcap_{i\in\mathcal{I}} \complement \left(O_i \right),
            \end{equation*}
            \item \begin{equation*}
                \complement \left(\bigcap_{i\in\mathcal{I}}O_i\right) = \bigcup_{i\in\mathcal{I}} \complement \left(O_i \right).
            \end{equation*}
        \end{enumerate}
    \end{theorem}
}