% modules\theorems.tex

%%%%%%%%%%%%%%%%%%%%%%%%%%%%%%%%%%%%%%%%%%%%%%%%%%%%
%%%%%%%%%%%%%%%%%%%%%%%%%%%%%%%%%%%%%%%%%%%%%%%%%%%%
%
%            FIRST CHAPTER STATEMENTS
%
%%%%%%%%%%%%%%%%%%%%%%%%%%%%%%%%%%%%%%%%%%%%%%%%%%%%
%%%%%%%%%%%%%%%%%%%%%%%%%%%%%%%%%%%%%%%%%%%%%%%%%%%%



%~~~~~~~~~~~~~~~~~~~~~~~~~~~~~~~~~~~~~~~~~~~~~~~~~~~~
% first chapter lemmas
%~~~~~~~~~~~~~~~~~~~~~~~~~~~~~~~~~~~~~~~~~~~~~~~~~~~~

\newcommand{\sigmaAlgebraIntersection}{
    \begin{lemma}{Interseção de \texorpdfstring{$\sigma$}{sigma}-álgebras é \texorpdfstring{$\sigma$}{sigma}-álgebra}{sigma_algebra_intersection}
        Seja $X$ um conjunto e $\{\Sigma_i\}_{i\in\mathcal{I}}$ uma família de \nameref{def:sigma_algebra} de $X$. Então, $\cap_{i\in\mathcal{I}} \Sigma_i $ é uma \texorpdfstring{$\sigma$}{sigma}-álgebra de $X$.
    \end{lemma}
    \begin{proof}
    Denotemos $\Sigma = \cap_{i\in\mathcal{I}}\Sigma_i$. Como $A_n\in \Sigma$ para todo $n\in \N$, então $A_n^{c}\in \Sigma$ para todo $n$ e $\left(A_n^{c}\right)_{n=1}^\infty \subset \Sigma$. Assim $\bigcup_{n=1}^\infty \left( A_n^{c} \right)\in \Sigma$. Note bem, pelo \nameref{thm:de_morgan}, 
    \begin{equation*}
        \bigcup_{n=1}^\infty \left( A_n^{c} \right) = \left( \bigcap_{n=1}^{\infty} A_n \right)^{c} \in \Sigma.
    \end{equation*}
    Portanto $\left(\bigcup_{n=1}^\infty \left( A_n^{c} \right)\right)^{c} = \bigcap_{n=1}^{\infty} A_n \in \Sigma$.
\end{proof}
}

\newcommand{\generatedSigmaAlgebraIsUnique}{
    \begin{lemma}{Unicidade da \texorpdfstring{$\sigma$}{sigma}-álgebra gerada}{generated_sigma_algebra_is_unique}
        Seja $X$ um conjunto e $\mathcal{C}\subset \powerset{X}$. Então, a \nameref{def:generated_sigma_algebra} por $\mathcal{C}$ é única.
    \end{lemma}
    \begin{proof}
    Sejam $\mathcal{A}$ e $\mathcal{B}$ duas $\sigma$-álgebras geradas por $\mathcal{C}$. Então, pelo primeiro item da definição de $\mathcal{A}$, temos que $\mathcal{C}\subset \mathcal{A}$. Por outro lado, usando o segundo item da definição de $\mathcal{B}$, temos que $\mathcal{B}\subset \mathcal{A}$. Analogamente, mostramos que $\mathcal{A}\subset \mathcal{B}$ e $\mathcal{A}=\mathcal{B}$.
\end{proof}
}

\newcommand{\sigmaAlgebraGeneratedBySubset}{
    \begin{lemma}{\texorpdfstring{$\sigma$}{sigma}-álgebra Gerada Por Subconjunto}{sigma_algebra_generated_by_subset}
        Seja $\mathcal{C}\in \powerset{X}$. Se $E\in \sigma [\mathcal{C}]$, então $\sigma [E] \subset \sigma [\mathcal{C}]$.
    \end{lemma}
    \begin{proof}
    Temos, por hipótese que $\sigma [\mathcal{C}]$ é uma $\sigma$-álgebra de $X$ que contém $E$. Logo, pelo segundo item da Definição \ref{def:generated_sigma_algebra}, $\sigma [E] \subset \sigma [\mathcal{C}]$.
\end{proof}
}



%~~~~~~~~~~~~~~~~~~~~~~~~~~~~~~~~~~~~~~~~~~~~~~~~~~~~
% first chapter propositions
%~~~~~~~~~~~~~~~~~~~~~~~~~~~~~~~~~~~~~~~~~~~~~~~~~~~~
\newcommand{\generatedSigmaAlgebraCharacterization}{
    \begin{proposition}{Caracterização da \texorpdfstring{$\sigma$}{sigma}-álgebra gerada}{generated_sigma_algebra_characterization}
        Sejam $X$ um conjunto, $\mathcal{C}\subset \powerset{X}$ e $\mathcal{J}=\{\Sigma_j\}$ a família de todas as $\sigma$-álgebras de $X$ que contém $\mathcal{C}$. Então,
        \begin{equation*}
            \sigma [\mathcal{C}] = \bigcap_{\Sigma_j\in\mathcal{J}} \Sigma_{j}.
        \end{equation*}
    \end{proposition}
    \begin{proof}
    Primeiramente, note que $\mathcal{J}$ não é uma família vazia, pois $\powerset{X}$ é uma $\sigma$-álgebra (Exemplo \ref{exm:power_set_sigma_algebra}) que contém $\mathcal{C}$ trivialmente, logo $\mathcal{C}$ está na interseção. Em segundo lugar, observe que, pelo Lema \ref{lmm:sigma_algebra_intersection}, a interseção dos elementos de $\mathcal{J}$ é uma $\sigma$-álgebra. Por último, verifique que $\bigcap_{\Sigma_j\in\mathcal{J}} \Sigma_{j} \subseteq \Sigma_j$ para todo $\Sigma_j \in \mathcal{J}$. Portanto, $\bigcap_{\Sigma_j\in\mathcal{J}} \Sigma_{j}$ é uma $\sigma$-álgebra gerada por $\mathcal{C}$ e, pelo Lema \ref{lmm:generated_sigma_algebra_is_unique}, concluímos que $\sigma [\mathcal{C}] = \bigcap_{\Sigma_j\in\mathcal{J}} \Sigma_{j}$.
\end{proof}
}

\newcommand{\productSigmaAlgebraOfCountableFamily}{
    \begin{proposition}{\sigmaAlg produto de família enumerável}{product_sigma_algebra_of_countable_family}
        Seja $I$ um conjunto enumerável e $\{X_i\}_{i\in I}$ uma família de espaços mensuráveis. Então
    \begin{equation*}
        \bigotimes_{i\in I} \Sigma_i=\generateSigmaAlg{\left\{\prod_{i\in I} E_i; E_i\in \Sigma_i\right\}}.
    \end{equation*}
    \end{proposition}
    \begin{proof}
    Vamos denotar por $\mathcal{F}$ a família de conjuntos do enunciado. Considere $\mathcal{C}$ a família como na Definição \ref{def:product_sigma_algebra}. Pretendemos mostrar que  $\generateSigmaAlg{\mathcal{C}}=\generateSigmaAlg{\mathcal{F}}$. Uma forma natural de chegar nesse resultado é mostrar que $\generateSigmaAlg{\mathcal{C}}\subseteq\generateSigmaAlg{\mathcal{F}}$ e $\generateSigmaAlg{\mathcal{F}}\subseteq\generateSigmaAlg{\mathcal{C}}$.
    
    Por um lado, podemos mostrar (ver Exercício X) que
    
    \begin{equation*}
        \prod_{i\in I} E_i = \bigcap_{i\in I} \pi_{i}^{-1} E_i.
    \end{equation*}
    
    Usando o Lema \ref{lmm:sigma_algebra_intersection}, mostramos que $\bigcap_{i\in I} \pi_{i}^{-1} (E_i) \in \generateSigmaAlg{\mathcal{C}}$. Com isso, concluímos que $\mathcal{F} \subseteq \generateSigmaAlg{\mathcal{C}}$ e, pelo Lema \ref{lmm:sigma_algebra_generated_by_subset}, $\generateSigmaAlg{\mathcal{F}}=\generateSigmaAlg{\mathcal{C}}$.
    
    Resta mostrar a segunda inclusão. Usaremos a mesma ideia. Para tanto, precisamos mostrar (ver Exercício Y) que, para todo $E_i\in\Sigma_i$ onde $i\in I$,
    
    \begin{eqnarray*}
        \pi_{i}^{-1} (E_i) 
        &=& X_1 \times \dots \times E_i \times\dots \\
        &=& \prod_{j\in I} E_j, \quad j\neq i \Rightarrow E_j=X_j.
    \end{eqnarray*}
    
    Sabemos que cada $X_j\in\Sigma_j$ pela definição de $\sigma$-álgebra e $E_i\in\Sigma_i$ por hipótese. Logo, $\pi_{i}^{-1} (E_i) \in \mathcal{F} \subset \generateSigmaAlg{\mathcal{F}}$. Usamos o Lema \ref{lmm:sigma_algebra_generated_by_subset} mais uma vez para concluir que $\generateSigmaAlg{\mathcal{C}}\subseteq\generateSigmaAlg{\mathcal{F}}$.
\end{proof}
}

\newcommand{\productSigmaAlgebraOfFamiliesGeneratedBySet}{
    \begin{proposition}{\sigmaAlg produto de famílias geradas por conjuntos}{product_sigma_algebra_of_families_generated_by_sets}
        Seja $\{X_i\}_{i\in I}$ uma família de espaços mensuráveis tal que cada $\Sigma_i=\sigma [\mathcal{C}_i]$. Então
    \begin{equation*}
        \bigotimes_{i\in I} \Sigma_i=\generateSigmaAlg{\left\{\pi_i^{-1}(E_i); E_i\in \mathcal{C}_i, i\in I\right\}}.
    \end{equation*}
    \end{proposition}
    \begin{proof}
    Vamos denotar por $\mathcal{F}$ a família de conjuntos do enunciado. Considere $\mathcal{C}$ a família como na Definição \ref{def:product_sigma_algebra}. Pretendemos mostrar que  $\generateSigmaAlg{\mathcal{C}}=\generateSigmaAlg{\mathcal{F}}$. Uma forma natural de chegar nesse resultado é mostrar que $\generateSigmaAlg{\mathcal{C}}\subseteq\generateSigmaAlg{\mathcal{F}}$ e $\generateSigmaAlg{\mathcal{F}}\subseteq\generateSigmaAlg{\mathcal{C}}$.
    
    Por um lado, podemos mostrar (ver Exercício X) que
    
    \begin{equation*}
        \prod_{i\in I} E_i = \bigcap_{i\in I} \pi_{i}^{-1} E_i.
    \end{equation*}
    
    Usando o Lema \ref{lmm:sigma_algebra_intersection}, mostramos que $\bigcap_{i\in I} \pi_{i}^{-1} (E_i) \in \generateSigmaAlg{\mathcal{C}}$. Com isso, concluímos que $\mathcal{F} \subseteq \generateSigmaAlg{\mathcal{C}}$ e, pelo Lema \ref{lmm:sigma_algebra_generated_by_subset}, $\generateSigmaAlg{\mathcal{F}}=\generateSigmaAlg{\mathcal{C}}$.
    
    Resta mostrar a segunda inclusão. Usaremos a mesma ideia. Para tanto, precisamos mostrar (ver Exercício Y) que, para todo $E_i\in\Sigma_i$ onde $i\in I$,
    
    \begin{eqnarray*}
        \pi_{i}^{-1} (E_i) 
        &=& X_1 \times \dots \times E_i \times\dots \\
        &=& \prod_{j\in I} E_j, \quad j\neq i \Rightarrow E_j=X_j.
    \end{eqnarray*}
    
    Sabemos que cada $X_j\in\Sigma_j$ pela definição de $\sigma$-álgebra e $E_i\in\Sigma_i$ por hipótese. Logo, $\pi_{i}^{-1} (E_i) \in \mathcal{F} \subset \generateSigmaAlg{\mathcal{F}}$. Usamos o Lema \ref{lmm:sigma_algebra_generated_by_subset} mais uma vez para concluir que $\generateSigmaAlg{\mathcal{C}}\subseteq\generateSigmaAlg{\mathcal{F}}$.
\end{proof}
}

\newcommand{\productSigmaAlgebraOfMetricSpaces}{
    \begin{proposition}{\sigmaAlg produto de família de espaços métricos}{product_sigma_algebra_of_metric_spaces}
        Seja $\{X_i\}_{i=1}^{n}$ uma família finita de espaços métricos com a topologia, $\tau_i$, induzida pela distância. Então
    \begin{equation*}
        \bigotimes_{i=1}^{n} \borel{X_i}\subseteq \borel{\prod_{i=1}^{n}X_i}.
    \end{equation*}
    \end{proposition}
    \begin{proof}
    Note bem, estamos trabalhando com um número finito de espaços métricos, cuja $\sigma$-álgebra é gerada pelos abertos de cada conjunto. Assim, valem, as hipóteses do Corolário \ref{cor:product_sigma_algebra_of_countable_generated_families}. Com isso, ganhamos que a $\sigma$-álgebra produto de $X$ é gerada pelo produto de todos os abertos. 

    \begin{equation*}
        \mathcal{C}\coloneq \left\{\prod_{i=1}^{n} E_i; E_i \in \tau_i \right\} \Longrightarrow \generateSigmaAlg{\mathcal{C}} = \bigotimes_{i=1}^{n} \mathcal{B}(X_i)
    \end{equation*}
    
    Por outro lado, o produto cartesiano de finitos abertos na \nameref{def:product_topology} (ver Proposição \ref{prop:finite_product_of_open_sets_is_open}) e, portanto, pertence a $\borel{X}$. Portanto, $\mathcal{C}\in\borel{X}$ e, pelo Lema \ref{lmm:sigma_algebra_generated_by_subset}, temos que
    \begin{equation*}
        \generateSigmaAlg{\mathcal{C}} = \bigotimes_{i=1}^{n} \mathcal{B}(X_i) \subseteq \borel{X}.
    \end{equation*}
\end{proof}
}

\newcommand{\measureIsMonotonic}{
    \begin{proposition}{Monotonicidade da Medida}{measure_is_monotonic}
        Seja $(X,\Sigma,\mu)$ um espaço de medida. Então, $\mu$ é uma \nameref{def:monotonic_function}.
    \end{proposition}
    \begin{proof}
    Sejam $E,F\in\Sigma$ tais que $E\subset F$. Podemos escrever $F$ como a união disjunta entre $E$ e $F\setminus E$. Como a medida $\mu$ é aditiva contável, então $\mu(F)=\mu(E)+\mu(F\setminus E)$. Note que $F\setminus E=F\cap E^{c}$, logo $F\setminus E\in \Sigma$. Como, $\mu(A)\geq 0$ para todo $A\in \Sigma$, temos que $\mu(E)\leq \mu(E)+\mu(F\setminus E)=\mu(F)$.
\end{proof}s
}

\newcommand{\measureIsSubadditive}{
    \begin{proposition}{Subaditividade da medida}{measure_is_subadditive}
        Seja $(X,\Sigma,\mu)$ um espaço de medida. Então $\mu$ é uma \nameref{def:countably_subadditive_function}.
    \end{proposition}
    \begin{proof}
    Vamos construir uma sequência disjunta $(F_n)$ a partir da sequência $(E_n)$ dada. Seja $F_1=E_1$ e defina
    \begin{equation*}
        F_k = E_k \setminus \left(\bigcup_{n=1}^{k-1} E_n\right).
    \end{equation*}
    Por construção, $F_i \cap F_j =\varnothing$ para quaisquer $i,j\in \N$ tais que $i\neq j$. Também temos que $F_n \subset E_n$ para todo $n\in \N$. Por último, notamos que
    \begin{equation}\label{eq/proof/measure_subadditivity}
        \bigcup_{n=1}^{\infty} E_n = \bigcup_{n=1}^{\infty} F_n \Longrightarrow \mu \left(\bigcup_{n=1}^{\infty} E_n\right) = \mu \left(\bigcup_{n=1}^{\infty} F_n\right).
    \end{equation}
    Como $(F_n)$ é disjunta e $\mu$ é \nameref{def:countably_additive_function}, concluímos, a partir da Equação \eqref{eq/proof/measure_subadditivity} que
    \begin{equation}\label{eq/proof/measure_monotonicity}
        \mu \left(\bigcup_{n=1}^{\infty} E_n\right) = \mu \left(\bigcup_{n=1}^{\infty} F_n\right) = \sum_{n=1}^{\infty} \mu (F_n).
    \end{equation}
    Agora, aplicando o fato de que $F_n\subset E_n$ \hyperref[prop:measure_is_monotonic]{medida é uma função monotônica}, temos que $\mu(F_n) \leq \mu(E_n)$ para todo $n\in N$. Assim, as séries também seguem esta desigualdade. Portanto, pela Equação \eqref{eq/proof/measure_monotonicity},
    \begin{equation*}
        \mu \left(\bigcup_{n=1}^{\infty} E_n\right) = \sum_{n=1}^{\infty} \mu (F_n)\leq \sum_{n=1}^{\infty} \mu (E_n).
    \end{equation*}
\end{proof}
}

\newcommand{\measureIsContinuousFromBelow}{
    \begin{proposition}{Continuidade por baixo}{measure_is_continuous_from_below}
        Seja $(X,\Sigma,\mu)$ um espaço de medida. Se $(E_n)_{n=1}^{\infty}\subset \Sigma$ é uma sequência crescente, então
        \begin{equation*}
            \mu\left(\bigcup_{n=1}^{\infty} E_n\right) = \lim \mu (E_n).
        \end{equation*}
    \end{proposition}
    \begin{proof}
    Vamos começar mostrando o primeiro item. Seja $(E_n)_{n=1}^{\infty}\subset \Sigma$ uma sequência crescente, isto é, $E_n\subset E_{n+1}$ para todo $n$ natural. Então, como na demonstração da Proposição \ref{prop:measure_is_monotonic}, $E_{n}=E_{n-1}\cup (E_{n}\setminus E_{n-1})$ para todo $n$ natural.
    
    Se definirmos $A_1=E_1$ e $A_n=E_{n}\setminus E_{n-1}$ para todo $n\geq 2$, então a sequência $(A_j)$ é disjunta. Para provar esta afirmação basta tomar $A_i$ e $A_j$ com $i\neq j$. Suponha, sem perda de generalidade, que $i < j$. Neste caso, $A_i=E_{i}\setminus E_{i-1}$ e $A_j=E_{j}\setminus E_{j-1}$ mas, por hipótese, $E_i\subseteq E_{j-1}$. Portanto, nenhum elemento de $A_i$ pode pertencer a $A_j$.

    Mostrado isso, podemos reescrever $E_n$ como a união de todos os $A_j$ até $n$.
    \begin{equation*}
        E_n=\bigcup_{j=1}^{n}A_j.
    \end{equation*}
    De fato, se $x\in E_n$ então $x\in E_{n-1} \cup A_n$. Por indução, $x\in\cup_{j=1}^{n}A_j$. Por outro lado, se $x\in \cup_{j=1}^{n}A_j$, então existe $j$ entre $1$ e $n$ tal que $x\in A_j=E_j\setminus E_{j-1}=E_j\cap E_{j-1}^{c}$. Logo, $x\in E_j\subseteq E_n$.

    Por fim, podemos concluir que
    \begin{equation*}
        \mu\left(\bigcup_{n=1}^{\infty} E_n\right)=\mu\left(\bigcup_{j=1}^{\infty} A_j\right)=\sum_{j=1}^{\infty} \mu(A_j)=\lim_{k\rightarrow \infty} \sum_{j=1}^{k} \mu(A_j).
    \end{equation*}
    Pela Proposição \ref{prop:measure_is_monotonic}, temos que
    \begin{equation*}
        \sum_{j=1}^{k} \mu(A_j) = \sum_{j=1}^{k} \left(\mu(E_j)-\mu(E_{j-1})\right) = \mu(E_k)\Longrightarrow \lim_{k\rightarrow \infty} \sum_{j=1}^{k} \mu(A_j)=\lim_{n\rightarrow \infty}\mu(E_n).
    \end{equation*}
    Note que, como a sequência $(E_n)$ é estritamente crescente, podemos nos certificar que cada $A_j$ é finito. Assim vale a \nameref{prop:measure_is_continuous_from_below}.
    
\end{proof}
}

\newcommand{\measureIsContinuousFromAbove}{
    \begin{proposition}{Continuidade por cima}{measure_is_continuous_from_above}
        Seja $(X,\Sigma,\mu)$ um espaço de medida. Se $(F_n)_{n=1}^{\infty}\subset \Sigma$ é uma sequência decrescente e a medida de $\mu (F_1)$ é finita, então
        \begin{equation*}
            \mu\left(\bigcap_{n=1}^{\infty} F_n\right) = \lim \mu (F_n).
        \end{equation*}
    \end{proposition}
    \begin{proof}
    Para provar, vamos usar o resultado da Proposição \ref{prop:measure_is_continuous_from_below}. A ideia é criar uma sequência crescente. Note bem, se a sequência é decrescente e o primeiro conjunto é finito, então todos os outros também são. Definimos, assim, os conjuntos $E_n=F_1\setminus F_n$ para formar a sequência crescente $(E_n)$.

    Por um lado, por \nameref{prop:measure_is_continuous_from_below} temos que
    \begin{equation*}
        \mu\left(\bigcup_{n=1}^{\infty} E_n\right) = \lim \mu (E_n) = \mu(F_1) - \lim \mu(F_n).
    \end{equation*}

    Por outro lado,
    \begin{equation*}
        \mu\left(\bigcup_{n=1}^{\infty} E_n\right) = \mu\left(F_1\setminus \bigcap_{n=1}^{\infty} F_n\right) = \mu(F_1) - \mu\left( \bigcap_{n=1}^{\infty} F_n\right).
    \end{equation*}
    Igualando as duas equações e cancelando o termo $\mu(F_1)$ temos que $\mu(\cap F_n)=\lim \mu(F_n)$.
    
\end{proof}
}



%~~~~~~~~~~~~~~~~~~~~~~~~~~~~~~~~~~~~~~~~~~~~~~~~~~~~
% first chapter corollary
%~~~~~~~~~~~~~~~~~~~~~~~~~~~~~~~~~~~~~~~~~~~~~~~~~~~~
\newcommand{\productSigmaAlgebraOfCountableGeneratedFamilies}{
    \begin{corollary}{\sigmaAlg produto de família enumerável geradas}{product_sigma_algebra_of_countable_generated_families}
        Seja $I$ um conjunto enumerável e $\{X_i\}_{i\in I}$ uma família de espaços mensuráveis tal que cada $\Sigma_i=\sigma [\mathcal{C}_i]$. Então $\bigotimes_{i\in I} \Sigma_i$ é a $\sigma$-álgebra gerada por 
    \begin{equation*}
        \bigotimes_{i\in I} \Sigma_i=\generateSigmaAlg{\left\{\prod_{i\in I} E_i; E_i\in \mathcal{C}_i\right\}}.
    \end{equation*}
    \end{corollary}
    \begin{proof}
    Este corolário é uma consequência direta das aplicações das Proposições \ref{prop:product_sigma_algebra_of_countable_family} e \ref{prop:product_sigma_algebra_of_families_generated_by_sets}.
\end{proof}
}

\newcommand{\productSigmaAlgebraOfSeparableMetricSpaces}{
    \begin{corollary}{\sigmaAlg produto de espaços métricos separáveis}{product_sigma_algebra_of_separable_metric_spaces}
        Sejam $\{X_i\}_{i=1}^{n}$ e $X$ como na Proposição \ref{prop:product_sigma_algebra_of_metric_spaces}. Se cada $X_i$ for separável, então $\bigotimes_{i=1}^{n} \mathcal{B}(X_i)= \mathcal{B}(X)$.
        \begin{equation*}
            \bigotimes_{i=1}^{n} \borel{X_i} = \borel{\prod_{i=1}^{n}X_i}.
        \end{equation*}
    \end{corollary}
    \begin{proof}
    Pela Proposição \ref{prop:product_sigma_algebra_of_metric_spaces}, temos a primeira inclusão. Queremos mostrar que $\borel{X}\subseteq \bigotimes_{i=1}^{n} \mathcal{B}(X_i)$.  
    
    Como cada $X_i$ é separável por hipótese, então existe $E_i\subset X_i$ denso e enumerável. É possível mostrar que existe uma base enumerável $B_i$ para cada $X_i$ (ver Proposição \ref{prop:metrizable_separable_space_is_second_countable}). Por outro lado, $\borel{X}$ é gerado por $\{\prod_{i=1}^{n} A_i; A_i\in B_i \} \subseteq \mathcal{C}$ (definido como na Proposição \ref{prop:product_sigma_algebra_of_metric_spaces}). Logo, pelo Lema \ref{lmm:sigma_algebra_generated_by_subset}, temos que $\borel{X}\subseteq \bigotimes_{i=1}^{n} \mathcal{B}(X_i)$.
\end{proof}
}

\newcommand{\productSigmaAlgebraOfRn}{
    \begin{corollary}{\sigmaAlg produto de $\R^n$}{product_sigma_algebra_of_Rn}
        \begin{equation*}
            \mathcal{B}(\R^n)=\bigotimes_{i=1}^{n} \mathcal{B}(\R).
        \end{equation*}
    \end{corollary}
    \begin{proof}
    Usando o Fato \ref{prop:rn_is_separable} de que $\R$ é separável, podemos aplicar diretamente o Corolário \ref{cor:product_sigma_algebra_of_countable_generated_families} para provar o resultado.
\end{proof}
}

\newcommand{\measureOfDiffenrece}{
    \begin{corollary}{Medida da Diferença}{measure_of_diffenrece}
        Nas condições da Proposição \ref{prop:measure_is_monotonic}, se $E,F\in \Sigma$ e $\mu(E)<\infty$, então
        \begin{equation*}
            \mu (F \setminus E) = \mu (F) - \mu (E).
        \end{equation*}
    \end{corollary}
    \begin{proof}
    Se $\mu(E)<\infty$, então podemos subtrair este valor dos dois lados sem correr o risco de chegar em uma diferença de infinitos. Portanto, $\mu (F \setminus E) = \mu (F) - \mu (E)$.
\end{proof}
}



%~~~~~~~~~~~~~~~~~~~~~~~~~~~~~~~~~~~~~~~~~~~~~~~~~~~~
% first chapter theorems
%~~~~~~~~~~~~~~~~~~~~~~~~~~~~~~~~~~~~~~~~~~~~~~~~~~~~
\newcommand{\deMorgan}{
    \begin{theorem}{Teorema de De Morgan}{de_morgan}
        Seja $X$ um conjunto e $\{O_i\}_{i\in \mathcal{I}}\subset \powerset{X}$. Então,
        \begin{enumerate}
            \item \begin{equation*}
                \complement \left(\bigcup_{i\in\mathcal{I}}O_i\right) = \bigcap_{i\in\mathcal{I}} \complement \left(O_i \right),
            \end{equation*}
            \item \begin{equation*}
                \complement \left(\bigcap_{i\in\mathcal{I}}O_i\right) = \bigcup_{i\in\mathcal{I}} \complement \left(O_i \right).
            \end{equation*}
        \end{enumerate}
    \end{theorem}
}

\newcommand{\measureCompletion}{
    \begin{theorem}{Completamento de Medida}{measure_completion}
        Seja $(X,\Sigma,\mu)$ um espaço de medida. Existe uma única $\overline{\mu}$, extensão de $\mu$ a uma $\sigma$-álgebra $\overline{\Sigma}$, tal que $(X,\overline{\Sigma},\overline{\mu})$ é um \nameref{def:complete_measure} .
    \end{theorem}
    \begin{proof}
    Denotemos por $\mathcal{N}$ a família de todos os conjuntos de medida nula.
    \begin{equation*}
        \mathcal{N}\coloneqq \{N\in\Sigma; \ \mu(N)=0\}.
    \end{equation*}
    
    Agora vamos expandir a nossa \texorpdfstring{$\sigma$}{sigma}-álgebra original para incluir todos os subconjuntos de conjuntos de medida nula.
    
    \begin{equation*}
        \overline{\Sigma}\coloneqq \left\{E\cup F; \ E\in \Sigma \text{ e } F\subset N\in\mathcal{N}\right\}.
    \end{equation*} 
    A ideia aqui é criar uma segunda função $\overline{\mu}:\overline{\Sigma}\rightarrow \left[0,+\infty\right]$ tal que $\overline{\mu}$ restrita a $\Sigma$ coincida com $\mu$, uma extensão da medida original. Como queremos que $\overline{\mu}$ seja uma medida, é preciso mostrar que $\overline{\Sigma}$ é uma \texorpdfstring{$\sigma$}{sigma}-álgebra (Passo XXX).

    Como argumentamos anteriormente, para $\overline{\mu}$ ser extensão $\mu$, ela deve coincidir com o valor assumido por $\mu$ para conjuntos em $\Sigma$. Definiremos, então
    \begin{equation*}
    \overline{\mu}(E\cup F) = \mu(E) \text{ para todo } E\cup F \in \overline{\Sigma}.
    \end{equation*}

    Para que isso funcione, é preciso mostrar que a função está bem definida (Passo YYY). Também precisamos mostrar que $\overline{\mu}$ satisfaz a definição de medida (Passo ZZZ), além de mostrar que ela é a única extensão existente (Passo AAA).

    Assim, mostramos que sempre podemos considerar um espaço de medida maior do que o original onde a nova medida será completa.
\end{proof}
}

\newcommand{\caratheodory}{
    \begin{theorem}{Teorema de Extensão de Carathéodory}{caratheodory}
        Sejam $\mathcal{H}$ um \texorpdfstring{$\sigma$}{sigma}-anel hereditário e $\mu^{*}$ uma medida exterior em $\mathcal{H}$. Então:
    \begin{enumerate}
        \item A coleção $\mathfrak{M}\subset \mathcal{H}$ dos conjuntos $\mu^{*}$-mensuráveis é um \texorpdfstring{$\sigma$}{sigma}-anel.\label{teo:caratheodory/item/sigma_anel}
        \item Dados $A\in\mathcal{H}$ e uma sequência $(E_n)_{n=1}^{\infty}\subset \mathfrak{M}$ de elementos dois a dois disjuntos, então
        \begin{equation*}\label{teo:caratheodory/item/aditiva_contavel}
            \mu^{*}\left(A\cap \bigcup_{n=1}^{\infty} E_n \right) = \sum_{n=1}^{\infty} \mu^{*}\left(A\cap E_n\right).
        \end{equation*}
        \item A restrição de $\mu^{*}$ a $\mathfrak{M}$ é \nameref{def:countably_additive_function}, uma medida.\label{teo:caratheodory/item/sigma_aditiva}
        \item Se $E\in\mathcal{H}$ é tal que $\mu^{*}(E)=0$ então $E\in\mathfrak{M}$.\label{teo:caratheodory/item/medida_mula}
    \end{enumerate}
    \end{theorem}
    \begin{proof}
    Para demonstrar o teorema, precisamos mostrar vários resultados. Aqui serão apresentadas as ideias gerais. Você poderá seguir até o Apêndice para conferir a demonstração passo a passo.
    
    Para mostrar o item \ref{teo:caratheodory/item/sigma_anel}, precisamos provar que $\mathfrak{M}$ é fechado para a união enumerável e para a diferença de conjuntos (Exercício \ref{exe:m_is_sigma_ring}).
    
    A fim de verificar o item \ref{teo:caratheodory/item/aditiva_contavel}, podemos usar indução para mostrar que, para algum $t\in\N$,
    
    \begin{equation*}
        \mu^{*}\left(A\cap \bigcup_{n=1}^{t} E_n \right) = \sum_{n=1}^{t} \mu^{*}\left(A\cap E_n\right).
    \end{equation*}
    Depois, podemos usar a monotonicidade em 
    \begin{equation*}
        A\cap\left(\bigcup_{n=1}^{\infty} E_n\right)^{\complement}\subset A\cap\left(\bigcup_{n=1}^{t} E_n\right)^{\complement}
    \end{equation*}
    para mostrar que 
    \begin{equation*}
        \mu^{*}(A)\geq\left(\sum_{n=1}^{t}\mu^{*}(A\cap E_n)\right) + \mu^{*}\left(A\cap\left(\bigcup_{n=1}^{\infty} E_n\right)^{\complement}\right).
    \end{equation*}
    Por fim, usamos a subaditividade e tomamos o limite $t\rightarrow \infty$ para concluir que
    \begin{equation*}
        \mu^{*}(A) =\left(\sum_{n=1}^{\infty}\mu^{*}(A\cap E_n)\right) + \mu^{*}\left(A\cap\left(\bigcup_{n=1}^{\infty} E_n\right)^{\complement}\right).
    \end{equation*}
    Chegamos no resultado desejado substituindo $A$ por $A\cap\left(\bigcup_{n=1}^{\infty} E_n\right)$.
    
    Os últimos dois itens são mais simples.

    Para mostrar o Item \ref{teo:caratheodory/item/sigma_aditiva}, tome $A=\bigcup_{k=1}^{\infty} E_k$ e aplique no Item \ref{teo:caratheodory/item/aditiva_contavel}. Temos, então

    \begin{equation*}
        \mu^{*}\left(A\cap \bigcup_{n=1}^{\infty} E_n \right) = \mu^{*}\left(\bigcup_{n=1}^{\infty} E_n \right) = \sum_{n=1}^{\infty} \mu^{*}\left(E_n\right).
    \end{equation*}

    Para mostrar o Item \ref{teo:caratheodory/item/medida_mula}, tome um conjunto $E\in\mathcal{H}$ de medida nula e um outro $A\in\mathcal{H}$ qualquer. Pela \hyperref[prop:measure_is_monotonic]{monotonicidade} da medida,

    \begin{equation*}
        \mu^*(A\cap E) + \mu^*(A\cap \complement E) \leq \mu^*(E) + \mu^*(A) = \mu^*(A).
    \end{equation*}

    Logo, $E$ é $\mu^*$-mensurável.
    \end{proof}
}





%%%%%%%%%%%%%%%%%%%%%%%%%%%%%%%%%%%%%%%%%%%%%%%%%%%%
%%%%%%%%%%%%%%%%%%%%%%%%%%%%%%%%%%%%%%%%%%%%%%%%%%%%
%
%            SECOND CHAPTER STATEMENTS
%
%%%%%%%%%%%%%%%%%%%%%%%%%%%%%%%%%%%%%%%%%%%%%%%%%%%%
%%%%%%%%%%%%%%%%%%%%%%%%%%%%%%%%%%%%%%%%%%%%%%%%%%%%

%~~~~~~~~~~~~~~~~~~~~~~~~~~~~~~~~~~~~~~~~~~~~~~~~~~~~
% second chapter lemmas
%~~~~~~~~~~~~~~~~~~~~~~~~~~~~~~~~~~~~~~~~~~~~~~~~~~~~

\newcommand{\integralOfSimpleFunctionSemiStandard}{
    \begin{lemma}{Integral de uma função simples na representação SP}{integral_of_simple_function_semi_standard}
        Seja $\varphi\in M^{+}(X,\Sigma)$ uma função simples. Se
        \begin{equation*}
            \varphi=\sum_{i=1}^{n} b_i\chi E_i
        \end{equation*}
        é uma representação semi-padrão de $\varphi$, então
        \begin{equation*}
            \int_X \varphi \ d\mu = \sum_{i=1}^{n} b_i \mu(E_i).
        \end{equation*}
    \end{lemma}
    \begin{proof}
    Suponha que $\varphi$ seja uma função simples com a representação 
    \[
    \varphi = \sum_{j=1}^{m} \alpha_j 1_{A_j}.
    \]
    A representação semi-padrão de $\varphi$ é
    \[
    \varphi = \sum_{k=1}^{n} \beta_k 1_{B_k}.
    \]
    Para cada $j = 1, \dots, m$, defina
    \[
    K_j = \{ k : 1 \leq k \leq n \ \text{e} \ \beta_k = \alpha_j \}.
    \]
    Daí,
    \[
    A_j = \bigcup_{k \in K_j} B_k.
    \]
    Assim, temos:
    \begin{align*}
        \int_X \varphi d\mu &= \sum_{j=1}^{m} \alpha_j \mu(A_j) \\
        &= \sum_{j=1}^{m} \alpha_j \sum_{k \in K_j} \mu(B_k) \\
        &= \sum_{j=1}^{m} \sum_{k \in K_j} \alpha_j \mu(B_k) \\
        &= \sum_{j=1}^{m} \sum_{k \in K_j} \beta_k \mu(B_k) \\
        &= \sum_{k=1}^{n} \beta_k \mu(B_k),
    \end{align*}
    pois
    \[
    \bigcup_{j=1}^{m} K_j = \{ 1, \dots, n \}.
    \]
\end{proof}

}

\newcommand{\integralInequalities}{
    \begin{lemma}{Desigualdade de integrais}{integral_inequalities}
        Sejam $f,g\in M^{+}(X,\Sigma)$ e $E,F\subset \Sigma$.
    \begin{enumerate}
        \item Se $f\leq g$, então $\int_{X} f \ d\mu \leq \int_{X} g \ d\mu$.
        \item Se $E\subseteq F$, então $\int_{E} f \ d\mu \leq \int_{F} f \ d\mu$.
    \end{enumerate}
    \end{lemma}
    \begin{proof}
    Para mostrar o primeiro item, perceba que toda função $\varphi \in M^{+}(X,\Sigma)$ simples tal que $\varphi \leq f$ também é, por hipótese, tal que $\varphi \leq g$. Ou seja,
    \begin{equation*}
        \left\{ \int_X \varphi \, d\mu; \ \varphi \in  M^{+}(X,\Sigma) \text{ e } 0 \leq \varphi \leq f \right\} \subseteq \left\{ \int_X \varphi \, d\mu; \ \varphi \in  M^{+}(X,\Sigma) \text{ e } 0 \leq \varphi \leq g \right\}.
    \end{equation*}
    Tomando o supremo dos dois lados, obtemos, pela Definição \ref{def:lebesgue_integral_non_negative_function},
    \begin{equation*}
        \int_{X} f \ d\mu \leq \int_{X} g \ d\mu.
    \end{equation*}
    Já para o segundo item basta notar que $f\chi_E\leq f\chi_F$ por hipótese. A demonstração segue da aplicação do primeiro item e da Definição \ref{def:lebesgue_integral_restricted}.
\end{proof}
}

\newcommand{\fatou}{
    \begin{lemma}{Lema de Fatou}{fatou}
        Seja $(f_n)\subset M^{+}(X,\Sigma)$. Então,
    \begin{equation*}
        \int_{X} \left(\liminf_{n\rightarrow \infty} f_n\right) \ d\mu \leq \liminf_{n\rightarrow \infty} \left( 
 \int_{X} f_n \ d\mu\right).
    \end{equation*}
    \end{lemma}
    \begin{proof}
    Defina $g_n=\inf_{k\geq n} f_k$. Pela definição de $g_n$, para todo $k\geq n$, $g_n\leq f_k$. Assim, pelo Lema \ref{lmm:integral_inequalities},
    \begin{equation*}
        \int_{X} g_n \ d\mu \leq \int_{X} f_k \ d\mu \quad \forall \ k\geq n.
    \end{equation*}
    Portanto, tomando o ínfimo dos dois lados,
    \begin{equation*}
        \int_{X} g_n \ d\mu \leq \inf_{k\geq n}\int_{X} f_k \ d\mu.
    \end{equation*}
    Tomemos o limite dos dois lados.
    \begin{equation}\label{eq/passo:fatou}
        \lim_{n\rightarrow \infty}\left(\int_{X} g_n \ d\mu\right) \leq \lim_{n\rightarrow \infty}\left(\inf_{k\geq n}\int_{X} f_k \ d\mu\right)=\liminf_{n\rightarrow\infty}\left(\int_{X} f_k \ d\mu\right) .
    \end{equation}
    Estamos bem próximo dos resultado final, basta observar que $(g_n)\subset M^{+}(X,\Sigma)$, pelo Lema \ref{prop:measurable_functions_sequences}, e que esta é uma sequência monótona crescente que converge para $\liminf f_n$. Portanto, podemos aplicar o Teorema da Convergência Monótona (\ref{thm:mct}) e concluir que
    \begin{equation*}
        \lim_{n\rightarrow\infty}\left(\int_{X} g_n \ d\mu\right) = \int_{X} \liminf_{n\rightarrow\infty} f_n \ d\mu.
    \end{equation*}
    Assim, juntando isso com \eqref{eq/passo:fatou}, temos que 
    \begin{equation*}
        \int_{X} \left(\liminf_{n\rightarrow \infty} f_n\right) \ d\mu \leq \liminf_{n\rightarrow \infty} \left( 
 \int_{X} f_n \ d\mu\right).
    \end{equation*}
\end{proof}
}

\newcommand{\integralIsZeroIffFunctionIsZeroAlmostEverywhere}{
    \begin{lemma}{Integral de funções nulas quase sempre}{integral_is_zero_iff_function_is_zero_almost_everywhere}
        Seja $(X,\Sigma,\mu)$ um \nameref{def:measure_space} e $f\in L$. Então,
        \begin{equation*}
            \int_{X} \abs{f} \ d\mu = 0 \Longleftrightarrow f = 0 (\mu\text{-qtp}).
        \end{equation*}
    \end{lemma}
    \begin{proof}
    Suponha que $\int_X \abs{f} \, d\mu = 0$. Defina $E_n = \{x \in X : f(x) > \frac{1}{n} \}$. Como $f \geq \frac{1}{n} \chi_{E_n}$, temos
    \[
    0 = \int_X \abs{f} \, d\mu \geq \frac{1}{n} \mu(E_n) \geq 0.
    \]
    Logo, $\mu(E_n) = 0$ para todo $n \in \mathbb{N}$ e, portanto, 
    \[
    \mu(\{ x \in X : f(x) > 0 \}) = \mu\left( \bigcup_{n=1}^{\infty} E_n \right) = 0.
    \]
    Portanto, $f = 0$ $\mu$-qtp.

    Reciprocamente, se $f = 0$ ($\mu$-qtp), então $|f| = 0$ ($\mu$-qtp) e, por consequência, 
    \[
    \int_X f \, d\mu = \int_X \abs{f} \, d\mu = 0.
    \]
\end{proof}

}


%~~~~~~~~~~~~~~~~~~~~~~~~~~~~~~~~~~~~~~~~~~~~~~~~~~~~
% second chapter propositions
%~~~~~~~~~~~~~~~~~~~~~~~~~~~~~~~~~~~~~~~~~~~~~~~~~~~~
\newcommand{\measurableFunctionsAndSigmaAlgebrasGeneratedBySet}{
    \begin{proposition}{Funções mensuráveis e \sigmaAlgs geradas por \texorpdfstring{$\mathcal{C}$}{C}}{measurable_functions_and_sigma_algebras_generated_by_set}
        Sejam $(X,\Sigma)$, $(Y,\generateSigmaAlg{\mathcal{C}}) $ \nameref{def:measurable_space} para algum $\mathcal{C}\subset \mathcal{P}(Y)$ e $f:(X,\Sigma)\rightarrow (Y,\generateSigmaAlg{\mathcal{C}})$. Então $f$ é mensurável se, e somente se, $f^{-1}(E)\in\Sigma$ para todo $E\in \mathcal{C}$.
    \end{proposition}
    \begin{proof}
    A ida é imediata, segue da Definição \ref{def:measurable_functions}. Para mostrar a volta, precisamos provar que $f^{-1}(E)\in\Sigma$ para cada $E\in \sigma(\mathcal{C})$. Definamos o conjunto abaixo.
    \begin{equation*}
        \mathcal{S}=\{E\in \sigma(\mathcal{C}); \ f^{-1}(E)\in \Sigma\}.
    \end{equation*}

    Note bem, temos, por hipótese, que $\mathcal{C}\subset \mathcal{S}$ por hipótese. Se este conjunto for uma $\sigma$-álgebra, então $\sigma(\mathcal{C}) \subset \mathcal{S}$, Logo, $\mathcal{S}=\mathcal{C}$ e $f$ é mensurável. Vamos provar, portanto, que $\mathcal{S}$ é uma $\sigma$-álgebra.

    Primeiramente, note que $\varnothing$ e $X$ pertencem a $\mathcal{S}$ pois $f^{-1}(\varnothing)=\varnothing$ e $f^{-1}(Y)=X$. Em segundo lugar, perceba que se $A\in \mathcal{S}$ então $f^{-1}(A)\in \Sigma$. Como $\Sigma$ é $\sigma$-álgebra, então $(f^{-1}(A))^{c}=f^{-1}(A^{c})\in \Sigma$. Assim, $A^{c}\in \mathcal{S}$. Por último, verifique que:
    \begin{equation*}
        \left(A_n\right)_{n=1}^\infty \subset \mathcal{S} \Rightarrow \left(f^{-1}(A_n)\right)_{n=1}^\infty \in \Sigma\Rightarrow  \bigcup_{n=1}^{\infty} f^{-1}(A_n)= f^{-1}\left(\bigcup_{n=1}^{\infty} A_n\right) \in \Sigma \Rightarrow \bigcup_{n=1}^{\infty} A_n\in S.
    \end{equation*}
    Mostrado que $S$ é uma $\sigma$-álgebra, concluímos o argumento de que $f$ é mensurável.
\end{proof}
}

\newcommand{\measurableFunctionsOperations}{
    \begin{proposition}{Operações com funções reais mensuráveis}{measurable_functions_operations}
        Sejam $f,g:X\rightarrow \R$ funções mensuráveis. Então,
    \begin{itemize}
        \item $f+g$,
        \item $\lambda f$ para todo $\lambda\in \R$,
        \item $f^2$,
        \item $|f|$,
        \item $\max\{f,g\}$,
        \item $\min\{f,g\}$,
        \item $fg$
    \end{itemize}
    são mensuráveis.
    \end{proposition}
    \begin{proof}
    Seja $\alpha\in\R$. Como $f$ e $g$ são funções mensuráveis, então os conjuntos
    \begin{eqnarray*}
        \{x\in X; \ f(x)> r\}\in \Sigma\\
        \{x\in X; \ g(x)> \alpha-r\}\in \Sigma
    \end{eqnarray*}
    para todo $r$ racional. Chamemos
    \begin{equation*}
        S_r=\{x\in X; \ f(x)> r\}\cap \{x\in X; \ g(x)> \alpha-r\}\in\Sigma.
    \end{equation*}
    Como, cada $S_r$ é mensurável, a sua união também é pois os racionais são enumeráveis. Assim, basta mostrar que
    \begin{equation*}
        \{x\in X; \ f(x)+g(x)> \alpha\}= \bigcup_{r\in \mathbb{Q}} S_r\in\Sigma.
    \end{equation*}
    Assim, pela propriedade, $f+g$ é mensurável.

    Para mostrar que $\lambda f$ é mensurável, é razoável desconsiderar o caso em que $\lambda =0$ pois a função identicamente nula é mensurável. Quando este não for o caso, basta mostrar que

    \begin{equation*}
        \{x\in X; \ \lambda f(x)> \alpha\}= \{x\in X; \ f(x)> \frac{\alpha}{\lambda}\}\in \Sigma
    \end{equation*}
        pois $f$ é mensurável por hipótese.

        Para mostrar que $f^2$ é mensurável, usamos uma estratégia similar. De fato, se $\alpha$ for negativo então a pré-imagem será $X\in\Sigma$. Caso contrário, então 
        \begin{equation*}
        \{x\in X; \ \lambda f^2(x)> \alpha\}= \{x\in X; \ f(x)> \sqrt{\alpha}\}\cup\{x\in X; \ f(x)< -\sqrt{\alpha}\} \in \Sigma
    \end{equation*}
    pois $f$ é mensurável. O caso é análogo para $|f|$.

    Para mostrar que $\max\{f,g\}$ e $\min\{f,g\}$ são mensuráveis, mostre que
    \begin{eqnarray*}
        \{x\in X; \ \max\{f,g\}> \alpha\}= \{x\in X; \ f(x)< \alpha\}\cap\{x\in X; \ g(x)<\alpha\} \in \Sigma\\
        \{x\in X; \ \min\{f,g\}> \alpha\}= \{x\in X; \ f(x)> \alpha\}\cap\{x\in X; \ g(x)>\alpha\} \in \Sigma\\
    \end{eqnarray*}

    Para mostrar que $fg$ é mensurável, escreva $fg=\frac{1}{4}((f+g)^2-(f-g)^2)$ e aplique as propriedades anteriores.
\end{proof}
}

\newcommand{\measurableFunctionsOperationsRExtend}{
    \begin{proposition}{Operações com funções \texorpdfstring{$\Rextend$}{R}-mensuráveis}{measurable_functions_operations_r_extend}
        Sejam $f,g:X\rightarrow \Rextend$ funções mensuráveis. Então,
    \begin{itemize}
        \item $\lambda f$ para todo $\lambda\in \R$,
        \item $f^2$,
        \item $|f|$,
        \item $\max\{f,g\}$,
        \item $\min\{f,g\}$,
        \item $fg$
    \end{itemize}
    são mensuráveis.
    \end{proposition}
    \begin{proof}
    Aqui os argumentos são os mesmos, mas a soma pode falhar quando as duas funções assumem valores infinitos, caindo no caso patológico $\infty - \infty$.
\end{proof}
}

\newcommand{\measurableFunctionsSequences}{
    \begin{proposition}{Sequências de funções mensuráveis}{measurable_functions_sequences}
        Seja $(f_n)\subset M(X,\Sigma)_{n=1}^{\infty}$ uma sequência de funções mensuráveis. Então,
    \begin{enumerate}
        \item $f(x)=\inf_{n\in\N} f_n(x)$,
        \item $F(x)=\sup_{n\in\N} f_n(x)$,
        \item $f^{*}(x)=\lim\inf_{n\in\N} f_n(x)$,
        \item $F^{*}(x)=\lim\sup_{n\in\N} f_n(x)$,
        \item $g(x)=\lim_{n\in\N} f_n(x)$ (dado que a sequência é pontualmente convergente),
    \end{enumerate}
    são mensuráveis
    \end{proposition}
    \begin{proof}
    \textbf{(1)} Para provar que $ f(x) = \inf_{n \in \mathbb{N}} f_n(x) $ é mensurável, observe que, para qualquer número real $\alpha$,
    \begin{equation*}
    \{ x \in X : f(x) \geq \alpha \} = \bigcap_{n=1}^{\infty} \{ x \in X : f_n(x) \geq \alpha \}.
    \end{equation*}
    Como cada $ f_n $ é mensurável, o conjunto $ \{ x \in X : f_n(x) \geq \alpha \} \in \Sigma $. Logo, $ f(x) $ é mensurável, já que a interseção enumerável de conjuntos mensuráveis pertence a $\Sigma$.

    \textbf{(2)} De forma análoga, para mostrar que $ F(x) = \sup_{n \in \mathbb{N}} f_n(x) $ é mensurável, observe que, para qualquer número real $\alpha$,
    \begin{equation*}
    \{ x \in X : F(x) \leq \alpha \} = \bigcap_{n=1}^{\infty} \{ x \in X : f_n(x) \leq \alpha \}.
    \end{equation*}
    Como cada $ f_n $ é mensurável, $ \{ x \in X : f_n(x) \leq \alpha \} \in \Sigma $, o que implica que $ F(x) $ é mensurável.

    \textbf{(3)} Para $ f^{*}(x) = \liminf_{n \to \infty} f_n(x) $, recorde que
    \begin{equation*}
    f^{*}(x) = \sup_{k \geq 1} \inf_{n \geq k} f_n(x).
    \end{equation*}
    Dado que $ f(x) = \inf_{n \in \mathbb{N}} f_n(x) $ e que o $ \sup $ de funções mensuráveis é mensurável, segue que $ f^{*}(x) $ é mensurável.

    \textbf{(4)} Para $ F^{*}(x) = \limsup_{n \to \infty} f_n(x) $, temos que
    \begin{equation*}
    F^{*}(x) = \inf_{k \geq 1} \sup_{n \geq k} f_n(x).
    \end{equation*}
    Usando o mesmo raciocínio do $ f^{*}(x) $, as operações de $ \sup $ e $ \inf $ pontuais preservam a mensurabilidade, então $ F^{*}(x) $ é mensurável.

    \textbf{(5)} Finalmente, se $ g(x) = \lim_{n \to \infty} f_n(x) $, dado que a sequência $ (f_n(x)) $ converge para todo $x \in X$, então $ g(x)=f^{*}(x)=F^{*}(x) $. Portanto, $ g(x) $ é mensurável.
\end{proof}

}

\newcommand{\integralOfSimpleFunctionIsLinear}{
    \begin{proposition}{Linearidade da integral de funções simples}{integral_of_simple_function_is_linear}
        Sejam $\varphi,\psi \in M^{+}(X,\Sigma)$ funções simples e $c\geq 0$. Então
    \begin{enumerate}
        \item $\int_{X} \ c\varphi \ d\mu = c\int_{X} \ \varphi \ d\mu$.
        \item $\int_{X} \ \varphi + \psi \ d\mu = \int_{X} \ \varphi \ d\mu + \int_{X} \ \psi \ d\mu$
    \end{enumerate}
    \end{proposition}
    \begin{proof}
    Para o primeiro item precisamos considerar dois casos. No primeiro caso, $c=0$ e a função simples que devemos integrar é a função simples constante. Isto valida a propriedade trivialmente pois teríamos que multiplicar a medida do conjunto inteiro por $0$. Mesmo se a medida for infinita, assumiremos ao longo destas notas que $0\cdot \infty=0$. Se $c > 0$ e $\varphi$ assume os valores $\{c_j\}_{j=1}^{n}$, então
    \begin{equation*}
        \int_{X}\ c\varphi \ d\mu = \sum_{j=i}^{n} c \, c_j \mu(f^{-1}(c_j)) = c\sum_{j=i}^{n}c_j \mu(f^{-1}(c_j))=c \int_{X}\varphi \ d\mu.
    \end{equation*}

    Vamos demonstrar o segundo item. Comece escrevendo $\varphi$ e $\psi$ na forma canônica.

    \begin{equation*}
        \varphi = \sum_{i=1}^{n} a_i \chi_{A_i} \quad \quad \text{e} \quad \quad 
        \psi = \sum_{j=1}^{m} b_j \chi_{B_j}.
    \end{equation*}

    Pelo Lema (Colocar no Apêndice), podemos $\chi_{A_i}=\sum_{j=1}^{m}\chi_{A_i\cap B_j}$. Portanto,

    \begin{equation*}
        \varphi = \sum_{i=1}^{n} a_i \chi_{A_i} = \sum_{i=1}^{n} a_i \sum_{j=1}^{m}\chi_{A_i\cap B_j}
    \end{equation*}

    Como $a_i$ é constante (com relação ao somatório de dentro), podemos passar para dentro. O processo é análogo para $\psi$, de onde vem que

    \begin{equation*}
        \varphi = \sum_{i=1}^{n}\sum_{j=1}^{m} a_i \chi_{A_i\cap B_j} \quad \quad \text{e} \quad \quad 
        \psi = \sum_{j=1}^{m}\sum_{i=1}^{n} b_j \chi_{A_i\cap B_j}.
    \end{equation*}

    Podemos trocar a ordem dos somatórios de $\psi$ uma vez que é uma soma finita. Com isso, temos

    \begin{equation*}
        \varphi + \psi = \sum_{i=1}^{n}\sum_{j=1}^{m} (a_i + b_j) \chi_{A_i\cap B_j}.
    \end{equation*}

    Note que esta é uma representação semi-padrão da função simples $\varphi+\psi$. Logo, pelo Lema \ref{lmm:integral_of_simple_function_semi_standard},
    \begin{eqnarray*}
        \int_X \varphi + \psi \ d\mu &=& \sum_{i=1}^{n}\sum_{j=1}^{m} (a_i + b_j) \mu(A_i\cap B_j)\\
        &=&\sum_{i=1}^{n}\sum_{j=1}^{m} a_i \mu(A_i\cap B_j) + \sum_{j=1}^{m}\sum_{i=1}^{n} b_j \mu(A_i\cap B_j)\\
        &=& \int_X \varphi d\mu + \int_X \psi d\mu
    \end{eqnarray*}
    
\end{proof}
}

\newcommand{\integralOfNonNegativeFunctionIsLinear}{
    \begin{proposition}{Linearidade da integral de funções não negativas}{integral_of_non_negative_function_is_linear}
        Sejam $f,g \in M^{+}(X,\Sigma)$ e $\lambda \geq 0$. Então
    \begin{enumerate}
        \item $\int_{X} \ \lambda f \ d\mu = \lambda \int_{X} \ f \ d\mu$.
        \item $\int_{X} \ f + g \ d\mu = \int_{X} \ f \ d\mu + \int_{X} \ g \ d\mu$
    \end{enumerate}
    \end{proposition}
    \begin{proof}
    \textbf{(1)} Se $\lambda = 0$, a igualdade é clara. Para $\lambda > 0$, seja $(\varphi_n)_{n=1}^{\infty}$ uma sequência crescente de funções simples em $M^+(X, \Sigma)$ que converge para $f$. Como $\lambda \varphi_n$ é uma sequência crescente de funções simples que converge para $\lambda f$, pelo \nameref{thm:mct}, temos
    \begin{equation*}
    \int_X \lambda f \, d\mu = \lim_{n \to \infty} \int_X \lambda \varphi_n \, d\mu = \lim_{n \to \infty} \lambda \int_X \varphi_n \, d\mu = \lambda \int_X f \, d\mu.
    \end{equation*}

    \textbf{(2)} Seja $(\varphi_n)_{n=1}^{\infty}$ e $(\psi_n)_{n=1}^{\infty}$ sequências crescentes de funções simples, convergindo para $f$ e $g$, respectivamente. Então, $(\varphi_n + \psi_n)_{n=1}^{\infty}$ é uma sequência crescente de funções simples que converge para $f + g$. Pelo \nameref{thm:mct}, temos
    \begin{equation*}
    \int_X (f + g) \, d\mu = \lim_{n \to \infty} \int_X (\varphi_n + \psi_n) \, d\mu = \lim_{n \to \infty} \left( \int_X \varphi_n \, d\mu + \int_X \psi_n \, d\mu \right) = \int_X f \, d\mu + \int_X g \, d\mu.
    \end{equation*}
\end{proof}

}

\newcommand{\integralIsLinear}{
    \begin{proposition}{Linearidade da integral de Lebesgue}{integral_is_linear}
        Sejam $f,g \in M(X,\Sigma)$ e $\lambda \geq 0$. Então
    \begin{enumerate}
        \item $\int_{X} \ \lambda f \ d\mu = \lambda \int_{X} \ f \ d\mu$.
        \item $\int_{X} \ f + g \ d\mu = \int_{X} \ f \ d\mu + \int_{X} \ g \ d\mu$
    \end{enumerate}
    \end{proposition}
    \begin{proof}
    Se $\lambda = 0$, o resultado é claro. Se $\lambda > 0$, então
    \[
    (\lambda f)^+ = \lambda f^+ \quad \text{e} \quad (\lambda f)^- = \lambda f^-.
    \]
    Logo,
    \[
    \begin{aligned}
        \int (\lambda f)^+ \, d\mu &= \int \lambda f^+ \, d\mu = \lambda \int f^+ \, d\mu < \infty, \\
        \int (\lambda f)^- \, d\mu &= \int \lambda f^- \, d\mu = \lambda \int f^- \, d\mu < \infty,
    \end{aligned}
    \]
    e portanto $\lambda f$ é integrável, com
    \[
    \int \lambda f \, d\mu = \int (\lambda f)^+ \, d\mu - \int (\lambda f)^- \, d\mu = \lambda \int f^+ \, d\mu - \lambda \int f^- \, d\mu = \lambda \int f \, d\mu.
    \]

    Agora, se $f, g \in L_1(X, \Sigma, \mu)$, então $|f|, |g| \in L_1(X, \Sigma, \mu)$. Note que $|f + g| \leq |f| + |g|$ e portanto
    \[
    \int (|f| + |g|) \, d\mu < \infty.
    \]
    Logo $f + g \in L_1(X, \Sigma, \mu)$ e daí $f + g \in L_1(X, \Sigma, \mu)$. Como
    \[
    f + g = (f^+ - f^-) + (g^+ - g^-) = (f^+ + g^+) - (f^- + g^-),
    \]
    segue da Observação 1.8.3 que
    \begin{eqnarray*}
        \int (f + g) \, d\mu &=& \int (f^+ + g^+) \, d\mu - \int (f^- + g^-) \, d\mu\\
        &=& \int f^+ \, d\mu + \int g^+ \, d\mu - \int f^- \, d\mu - \int g^- \, d\mu \\
        &=& \int f \, d\mu + \int g \, d\mu.
    \end{eqnarray*}
    
\end{proof}

}

\newcommand{\functionIsIntegrableIffAbsoluteValueIs}{
    \begin{proposition}{Condição para integrabilidade}{function_is_integrable_iff_absolute_value_is}
        Uma função $f \in L$ se, e somente se, $|f| \in L$. Neste caso,
    \begin{equation*}
        \left|\int f \, d\mu \right| \leq \int |f| \, d\mu.
    \end{equation*}
    \end{proposition}
    \begin{proof}
    Comecemos supondo que $f$ é integrável. Sabemos que $\abs{f}=f^{+}+f^{-}$. Neste caso, a parte positiva de $\abs{f}$, denotada por $\abs{f}^{+}$ é a própria soma de funções mensuráveis $f^{+}+f^{-}$. Por outro lado, a parte negativa, $\abs{f}^{-}=0$, pois a função é não negativa. Pelo Lema \ref{lmm:integral_is_zero_iff_function_is_zero_almost_everywhere}, temos que 
    \begin{equation*}
        \int \abs{f}^{-} d\mu = 0 < +\infty.
    \end{equation*}
    Ademais, pela Proposição \ref{prop:integral_of_non_negative_function_is_linear},
    \begin{equation*}
        \int \abs{f}^{+} d\mu = \int f^{+}+f^{-} d\mu = \int f^{+} d\mu + \int f^{-} d\mu.
    \end{equation*}
    Como, pela Definição \ref{def:lebesgue_integral},
    \begin{equation*}
        \int f^{+} d\mu <+\infty \quad \text{e}\quad  \int f^{-} d\mu < +\infty, \text{ então} \int \abs{f}^{+} d\mu < +\infty.
    \end{equation*}
    concluimos que $\abs{f}$ é integrável e
    \begin{equation*}
        \int \abs{f} d\mu = \int \abs{f}^{+} d\mu + \int \abs{f}^{-} d\mu = \int f^{+} d\mu + \int f^{-} d\mu.
    \end{equation*}
    Reciprocamente, se $\abs{f}$ é integrável, então $f^{+}$ e $f^{-}$ são integráveis. Portanto, $f$ é integrável.

    Mostrado isso, podemos concluir, pela desigualdade triangular, que
    \begin{equation*}
        \abs{\int f d\mu} = \abs{\int f^{+} - f^{-} d\mu} \leq \abs{\int f^{+} d\mu} + \abs{\int f^{-} d\mu} = \int f^{+} d\mu + \int f^{-} d\mu = \int \abs{f} d\mu.
    \end{equation*}
\end{proof}
}



%~~~~~~~~~~~~~~~~~~~~~~~~~~~~~~~~~~~~~~~~~~~~~~~~~~~~
% second chapter corollary
%~~~~~~~~~~~~~~~~~~~~~~~~~~~~~~~~~~~~~~~~~~~~~~~~~~~~

\newcommand{\measurableFunctionsInR}{
    \begin{corollary}{Mensurabilidade de funções reais}{measurable_functions_in_r}
        Sejam $(X,\Sigma)$, $(Y,\generateSigmaAlg{\mathcal{C}}) $ \nameref{def:measurable_space} para algum $\mathcal{C}\subset \mathcal{P}(Y)$ e $f:(X,\Sigma)\rightarrow (Y,\generateSigmaAlg{\mathcal{C}})$. Então $f$ é mensurável se, e somente se, $f^{-1}(E)\in\Sigma$ para todo $E\in \mathcal{C}$.
    \end{corollary}
    \begin{proof}
    Seja $f:(X,\Sigma)\rightarrow (\R,\mathcal{B})$ uma função mensurável e $\alpha$ um número real. Então $f^{-1}((\alpha, \infty))\in \Sigma$, isto é, $A_{\alpha}\in \Sigma$. Pela arbitrariedade da escolha de $\alpha$, vale a ida da proposição.

    Reciprocamente, seja $f:(X,\Sigma)\rightarrow (\R,\mathcal{B})$ uma função tal que, para todo $\alpha$ real, $A_{\alpha} \in \Sigma$. Pelo Exercício \ref{prop:basis_of_r} $\mathcal{B}$ pode ser gerado pelos intervalos na forma $(\alpha,\infty)$. Assim, pela Proposição \ref{prop:measurable_functions_and_sigma_algebras_generated_by_set}, só precisamos verificar os intervalos da forma $(\alpha,\infty)$, que pertencem a $\Sigma$  por hipótese. Logo, $f$ é mensurável.
\end{proof}
}

\newcommand{\measurableFunctionsInRExtend}{
    \begin{corollary}{Mensurabilidade de funções reais}{measurable_functions_in_r_extend}
        Seja $(X,\Sigma)$ um \nameref{def:measurable_space} e $f:(X,\Sigma)\rightarrow (\Rextend,\mathcal{B}(\Rextend))$. Sejam,
    \begin{equation*}
        A\coloneq\{x\in X; \ f(x)=+\infty\} \quad \text{e} \quad B\coloneq\{x\in X; \ f(x)=\infty\}
    \end{equation*}
    e a função $f_0:(X,\Sigma)\rightarrow (\R,\mathcal{B})$ definida da seguinte forma
    \begin{equation*}
        f_0(x) = \begin{cases}
            f(x), \text{ se } x \in (A\cup B)^{c} \\
            0, \text{ se } x \in A\cup B.
        \end{cases}
    \end{equation*}
    Então $f$ é mensurável se, e somente se, $A,B\in\Sigma$ e a $f_0$ é $\R$-mensurável. 
    \end{corollary}
    \begin{proof}
    Se $f$ for mensurável, então todos os $A,B\in\Sigma$ trivialmente. Além disso, $f_0$ deve ser $\R$-mensurável pois, caso não fosse, existiria um aberto de $\R\subset \Rextend$ tal que a pré-imagem deste aberto não é $\Rextend$-mensurável, o que é uma contradição.
    
    Reciprocamente, pela Definição \ref{def:basis_of_r_extend}, sabemos que $\mathcal{B}(\Rextend)$ pode ser gerado pelos intervalos na forma $(a,\infty)$ com os conjuntos $\{-\infty\}$, $\{+\infty\}$. Assim, pela Proposição \ref{prop:measurable_functions_and_sigma_algebras_generated_by_set}, a hipótese implica a mensurabilidade de $f$. 
\end{proof}
}

\newcommand{\dominatedFunctionIsIntegrable}{
    \begin{corollary}{Funções dominadas por função integrável é integrável}{dominated_function_is_integrable}
        Se $f\in M$, $g \in L$ e $|f| \leq |g|$, então $f\in L$ e 
    \begin{equation*}
        \int |f| \, d\mu \leq \int |g| \, d\mu.
    \end{equation*}
    \end{corollary}
    \begin{proof}
    Como $\abs{f}\leq \abs{g}$, pelo Lema \ref{lmm:integral_inequalities}, temos que $\int \abs{f} d\mu \leq \int \abs{g} d\mu$ e, portanto, $\abs{f}$ é integrável. Pela Proposição \ref{prop:function_is_integrable_iff_absolute_value_is}, temos, então, que $f$ é integrável.
\end{proof}
}



%~~~~~~~~~~~~~~~~~~~~~~~~~~~~~~~~~~~~~~~~~~~~~~~~~~~~
% second chapter theorems
%~~~~~~~~~~~~~~~~~~~~~~~~~~~~~~~~~~~~~~~~~~~~~~~~~~~~

\newcommand{\MCT}{
    \begin{theorem}{Teorema da Convergência Monótona}{mct}
        Seja $(f_n)\subset M^{+}(X,\Sigma)$ uma sequência monótona crescente tal que $f_n \xrightarrow{p} f$. Então $f\in M^{+}(X,\Sigma)$ e 
    \begin{equation}
        \int_{X} \left(\lim_{n\rightarrow \infty} f_n\right) \ d\mu = \lim_{n\rightarrow \infty} \left( 
 \int_{X} f_n \ d\mu\right)
    \end{equation}
    \end{theorem}
    \begin{proof}
    A mensurabilidade de $f$ segue diretamente da Proposição \ref{prop:measurable_functions_sequences}, mas é um passo importante para que faça sentido integrar essa função segundo a Definição \ref{def:lebesgue_integral_non_negative_function}. Provaremos a segunda parte do teorema mostrando que
    \begin{equation*}
        \int_{X} \left(\lim_{n\rightarrow \infty} f_n\right) \ d\mu \geq \lim_{n\rightarrow \infty} \left( 
 \int_{X} f_n \ d\mu\right) \quad \quad \text{e} \quad \quad \int_{X} \left(\lim_{n\rightarrow \infty} f_n\right) \ d\mu \leq \lim_{n\rightarrow \infty} \left( 
 \int_{X} f_n \ d\mu\right).
    \end{equation*}
    Para mostrar que a primeira desigualdade, note que, como $(f_n)$ é uma sequência monótona crescente, então $f_n\leq f_{n+k}$ para todos $n,k$ natural. Assim, se $k\rightarrow \infty$ então $f_{n+k} \rightarrow f$ para todo $n$. Logo,
    \begin{equation*}
        f_n \leq \lim_{n\rightarrow \infty}f_n \quad \forall \ n\in\N.
    \end{equation*}
    Pelo Lema \ref{lmm:integral_inequalities}, o fato acima nos dá que
    \begin{equation*}
        \int_{X} f_n \ d\mu \leq \int_{X} \left(\lim_{n\rightarrow \infty} f_n\right) \ d\mu \quad \forall \ n\in\N.
    \end{equation*}
    Note que o lado direito da desigualdade é constante. De fato, ele será igual a $\int_X f d\mu$. Assim, podemos tomar o limite dos dois lados para concluir que
    \begin{equation*}
        \lim_{n\rightarrow \infty}\left(\int_{X} f_n \ d\mu\right) \leq \int_{X} \left(\lim_{n\rightarrow \infty} f_n\right) \ d\mu \Longleftrightarrow \int_{X} \left(\lim_{n\rightarrow \infty} f_n\right) \ d\mu \geq \lim_{n\rightarrow \infty} \left( 
 \int_{X} f_n \ d\mu\right).
    \end{equation*}

    Aqui vale notar que o Lema \ref{lmm:integral_inequalities} garante que $\left(\int_X f_n d\mu\right)$ também é uma sequência crescente, o que justifica a existência do limite em $\Rextend$. Resta mostrarmos a outra desigualdade.
    
    A ideia é mostrar que, para toda função simples $\varphi \leq f$, teremos $\int_X \varphi \leq \lim (\int_X f_n d\mu)$. Depois poderemos tomar o supremo dos dois lados da desigualdade para chegar no resultado desejado.

    Começaremos tomando uma função $\varphi\in M^{+}(X,\Sigma)$ simples tal que $\varphi\leq f$ e $\alpha\in\R$ tal que $\alpha\in (0,1)$. Agora, defina o conjunto
    \begin{equation}
        A_n\coloneqq \{x\in X; \ \alpha\varphi(x)\leq f_n(x)\}.
    \end{equation}
    Faremos três afirmações sobre esse conjunto.
    \begin{enumerate}
        \item $A_n\in\Sigma$ para todo $n$ natural.\label{teo/lema:A_n_mensuravel}
        \item $(A_n)_{n=1}^{\infty}\subset \Sigma$ é uma sequência monótona crescente.\label{teo/lema:monotona_crescente}
        \item $\cup_{n=1}^{\infty}A_n=X$.\label{teo/lema:limite_conjunto}
    \end{enumerate}
    
    Vamos provar o Item \ref{teo/lema:A_n_mensuravel}. Aplicando as operações estabelecidas na Proposição \ref{prop:measurable_functions_operations}, temos que $f_n-\alpha\varphi$ é uma função mensurável para todo $n$. Além disso, podemos escrevermos cada $A_n$ como a pré-imagem da função $f_n-\alpha\varphi$.
    \begin{equation*}
        A_n=(f_n-\alpha\varphi)^{-1}((0,\infty)) \ \forall \ n\in\N.
    \end{equation*}
    Logo, pela Definição \ref{def:measurable_functions}, $A_n\in \Sigma$ para todo $n$.
    
    Vamos provar o Item \ref{teo/lema:monotona_crescente}. Para tanto precisamos mostrar que $A_n\subset A_{n+1}$ para todo $n$. Esta afirmação segue imediatamente da monotonicidade de $(f_n)$ uma vez que se $x\in A_n$ então $\alpha\varphi(x)\leq f_n(x)\leq f_{n+1}$.

    Vamos provar o Item \ref{teo/lema:limite_conjunto}. A primeira continência é imediata uma vez que cada $A_n$ é um subconjunto de $X$. No entanto, ainda é preciso mostrar que $X\subseteq \cup_{n=1}^{\infty} A_n$. Tome $x\in X$. Sabemos que $\varphi (x) \leq f(x)$ por hipótese. Como $\alpha\in (0,1)$, então $\alpha\varphi(x)\leq f(x)$. Queremos mostrar que existe um $N$ natural tal que $\alpha \varphi(x) \leq f_N(x)$. Faremos isso por contradição. Suponha que, para todo $n\in\N$, $f_n(x)<\alpha\varphi(x)$. Tomando o limite dos dois lados teríamos que $\lim f_n(x) = f(x) < \alpha \varphi(x)$, o que contradiz a nossa hipótese. Logo, existe $N\in\N$ tal que $\alpha \varphi(x) \leq f_N(x)$. Assim sendo, $x\in A_N \subset \cup_{n}^{\infty} A_n$. Por fim, concluímos que $\cup_{n=1}^{\infty}A_n=X$.

    Provados estes detalhes técnicos, vamos usar o Lema \ref{lmm:integral_inequalities} mais uma vez. Note que $A_n\subset X$ para todo $n$. Portanto, pelo segundo item do lema, temos que $\int_{A_n} f_n d\mu \leq \int_{X} f_n d\mu$ para todo $n$. Agora, pela definição de $A_n$, temos que $\alpha\varphi\chi_{A_n} \leq f_n\chi_{A_n}$. Logo, pelo primeiro item do lema temos que $\int_{A_n} \alpha\varphi d\mu \leq \int_{A_n} f_n d\mu$. Juntando essas duas desigualdades, obtemos
    \begin{equation*}
        \alpha \int_{A_n} \varphi \ d\mu \leq \int_{X} f_n \ d\mu \Rightarrow \alpha \lim_{n\rightarrow \infty} \int_{A_n} \varphi \ d\mu \leq \lim_{n\rightarrow \infty}\left(\int_{X} f_n \ d\mu\right).
    \end{equation*}
    Estamos quase chegando no resultado desejado, resta mostrar que $\lim\int_{A_n} \varphi \ d\mu = \int_{X} \varphi \ d\mu$. Esta demonstração pode ser encontrada em (Fazer).

    Por fim, podemos tomar o limite de $\alpha$ indo para $1$ dos dois lados e depois tomar o supremo de ambos os lados.
    \begin{equation*}
        \alpha \int_{X} \varphi \ d\mu \leq \lim_{n\rightarrow \infty}\left(\int_{X} f_n \ d\mu\right)\Rightarrow \lim_{\alpha\rightarrow 1} \alpha \int_{X} \varphi \ d\mu = \int_{X} \varphi \ d\mu \leq \lim_{n\rightarrow \infty}\left(\int_{X} f_n \ d\mu\right).
    \end{equation*}
    \begin{equation*}
        \therefore \sup\left(\int_{X} \varphi \ d\mu \right) = \int_{X} f \ d\mu \leq \lim_{n\rightarrow \infty}\left(\int_{X} f_n \ d\mu\right).
    \end{equation*}
    Assim, chegamos na desigualdade que buscávamos:
    \begin{equation*}
        \int_{X} \left(\lim_{n\rightarrow \infty} f_n\right) \ d\mu \leq \lim_{n\rightarrow \infty} \left( 
 \int_{X} f_n \ d\mu\right).
    \end{equation*}
\end{proof}
}

\newcommand{\DCT}{
    \begin{theorem}{Teorema da Convergência Dominada}{dct}
        Sejam $(f_n)_{n=1}^{\infty}\subset L$ e $f\in M(X,\Sigma)$ tais que
    \begin{equation*}
        \lim_{n\rightarrow \infty} f_n (x) = f(x).
    \end{equation*}
    Se existe $g\in L$ tal que $|f_n| \leq g$ para todo $n$, então $f\in L$ e
    \begin{equation*}
        \int_X f d\mu = \lim_{n\rightarrow \infty} \int_X f_n d\mu.
    \end{equation*}
    \end{theorem}
    \begin{proof}
    Como $\lim_{n \to \infty} f_n(x) = f(x)$ e cada $f_n$ é mensurável, segue que $f$ é mensurável. Como $|f_n| \leq g$ para todo $n$, temos que $|f| \leq g$. Mas, por hipótese, $g$ é integrável e, consequentemente, $|g|$ é integrável. Portanto, $|f|$ é integrável e, finalmente, $f$ é integrável. Temos ainda 
    \[
    -|g| \leq f_n \leq |g|
    \]
    e daí segue que 
    \[
    |g| + f_n \geq 0 \ \text{para todo } n.
    \]
    
    Portanto,
    \begin{align*}
    \int_X |g| d\mu + \int_X f_n d\mu &= \int_X (|g| + f_n) d\mu \\
    &= \int_X \lim_{n \to \infty} (|g| + f_n) d\mu
    \end{align*}
    Pelo Lema de Fatou,
    \[
    \leq \liminf_{n \to \infty} \int_X (|g| + f_n) d\mu
    \]
    \begin{align*}
    &= \liminf_{n \to \infty} \left( \int_X |g| d\mu + \int_X f_n d\mu \right) \\
    &= \int_X |g| d\mu + \liminf_{n \to \infty} \int_X f_n d\mu.
    \end{align*}
    
    Logo,
    \[
    \int_X f d\mu \leq \liminf_{n \to \infty} \int_X f_n d\mu.
    \]
    
    Por outro lado,
    \[
    |g| - f_n \geq 0 \ \text{para todo } n.
    \]
    
    Portanto,
    \begin{align*}
    \int_X |g| d\mu - \int_X f_n d\mu &= \int_X (|g| - f_n) d\mu \\
    &= \int_X \lim_{n \to \infty} (|g| - f_n) d\mu
    \end{align*}
    Pelo Lema de Fatou,
    \[
    \leq \liminf_{n \to \infty} \int_X (|g| - f_n) d\mu
    \]
    \begin{align*}
    &= \liminf_{n \to \infty} \left( \int_X |g| d\mu - \int_X f_n d\mu \right) \\
    &= \int_X |g| d\mu - \limsup_{n \to \infty} \int_X f_n d\mu.
    \end{align*}
    
    Consequentemente,
    \[
    \int_X f d\mu \geq \limsup_{n \to \infty} \int_X f_n d\mu.
    \]
    
    Finalmente, temos
    \[
    \limsup_{n \to \infty} \int_X f_n d\mu \leq \int_X f d\mu \leq \liminf_{n \to \infty} \int_X f_n d\mu,
    \]
    e, portanto,
    \[
    \int_X f d\mu = \lim_{n \to \infty} \int_X f_n d\mu.
    \]
\end{proof}

}

\newcommand{\DCTAlmostEverywhere}{
    \begin{theorem}{Teorema da Convergência Dominada (\texorpdfstring{$\mu$}{mu}-qtp)}{dct_ae}
        Sejam $(f_n)_{n=1}^{\infty}\subset L$ e $f\in M(X,\Sigma)$ tais que
    \begin{equation*}
        \lim_{n\rightarrow \infty} f_n (x) = f(x) \ (\mu\text{-qtp}).
    \end{equation*}
    Se existe $g\in L$ tal que $|f_n| \leq \abs{g}$ ($\mu$-qtp) para todo $n$, então $f\in L$ e
    \begin{equation*}
        \int_X f d\mu = \lim_{n\rightarrow \infty} \int_X f_n d\mu.
    \end{equation*}
    \end{theorem}
    \begin{proof}
    Seja $N_0 \in \Sigma$ tal que $\mu(N_0) = 0$ e $f(x) = \lim_{n \to \infty} f_n(x)$ para todo $x \in X \setminus N_0$. Defina $N_n \in \Sigma$ com $\mu(N_n) = 0$ tal que $|f_n(x)| \leq |g(x)|$ para todo $x \in X \setminus N_n$.

    Defina $N = \bigcup_{n=0}^{\infty} N_n$. Então, $\mu(N) = 0$ e temos que $f(x) = \lim_{n \to \infty} f_n(x)$ para todo $x \in X \setminus N$ e $|f_n(x)| \leq |g(x)|$ para todo $x \in X \setminus N$.

    Se $A = X \setminus N$, então pelo \nameref{thm:dct}, $f|_A$ é integrável e
    \[
    \int_A f \, d\mu = \lim_{n \to \infty} \int_A f_n \, d\mu.
    \]

    Como
    \[
    f = f|_A + \chi_N \cdot f \quad \text{(logo $f$ é integrável)} \quad \text{e} \quad f_n = f_n|_A + \chi_N \cdot f_n \quad \text{para todo $n$},
    \]
    pelo Lema \ref{lmm:integral_is_zero_iff_function_is_zero_almost_everywhere}, temos que
    \[
    \int_X f_n \, d\mu = \int_A f_n \, d\mu \quad \text{e} \quad \int_X f \, d\mu = \int_A f \, d\mu.
    \]

    Logo,
    \[
    \int_X f \, d\mu = \lim_{n \to \infty} \int_X f_n \, d\mu.
    \]
\end{proof}

}







%%%%%%%%%%%%%%%%%%%%%%%%%%%%%%%%%%%%%%%%%%%%%%%%%%%%
%%%%%%%%%%%%%%%%%%%%%%%%%%%%%%%%%%%%%%%%%%%%%%%%%%%%
% 
%            THIRD CHAPTER STATEMENTS
%
%%%%%%%%%%%%%%%%%%%%%%%%%%%%%%%%%%%%%%%%%%%%%%%%%%%%
%%%%%%%%%%%%%%%%%%%%%%%%%%%%%%%%%%%%%%%%%%%%%%%%%%%%

%~~~~~~~~~~~~~~~~~~~~~~~~~~~~~~~~~~~~~~~~~~~~~~~~~~~~
% third chapter lemmas
%~~~~~~~~~~~~~~~~~~~~~~~~~~~~~~~~~~~~~~~~~~~~~~~~~~~~

\newcommand{\powerInequality}{
    \begin{lemma}{.}{power_inequality}
        Para quaisquer $a, b > 0$, temos
    \begin{equation*}
        a^p b^q \leq \frac{a}{p} + \frac{b}{q}.
    \end{equation*}
    \end{lemma}
    \begin{proof}
    Considere, para cada $0 < \alpha < 1$, a função $f_\alpha : (0, \infty) \to \mathbb{R}$ dada por $f_\alpha(t) = t^\alpha - \alpha t$. Temos
    \begin{equation*}
        f'_\alpha(t) = \alpha t^{\alpha - 1} - \alpha = \alpha(t^{\alpha - 1} - 1).
    \end{equation*}
    Logo,
    \begin{equation*}
        f'_\alpha(t) > 0 \text{ se } 0 < t < 1
        \quad \text{e} \quad
        f'_\alpha(t) < 0 \text{ se } t > 1,
    \end{equation*}
    e $f_\alpha$ tem um máximo em $t = 1$. Portanto,
    \begin{equation*}
        f_\alpha(t) \leq f_\alpha(1)
    \end{equation*}
    para todo $t > 0$, e
    \begin{equation*}
        t^\alpha \leq \alpha t + (1 - \alpha).
    \end{equation*}

    Fazendo $t = \frac{a}{b}$ e $\alpha = \frac{1}{p}$, temos
    \begin{equation*}
        \left( \frac{a}{b} \right)^{\frac{1}{p}} \leq \frac{1}{p} \frac{a}{b} + \left(1 - \frac{1}{p}\right).
    \end{equation*}
    Multiplicando a desigualdade acima por $b$, obtemos
    \begin{equation*}
        a^{\frac{1}{p}} b^{\frac{1}{q}} \leq \frac{a}{p} + \frac{b}{q},
    \end{equation*}
    o que demonstra o lema.
\end{proof}
}

\newcommand{\almostUniformCauchyLemma}{
    \begin{lemma}{Convergência Quase Uniforme para Sequência de Cauchy}{almost_uniform_cauchy_lemma}
        Seja $(f_n)$ uma sequência de Cauchy quase uniformemente. Então, existe uma função mensurável $f$ tal que $(f_n)$ converge quase uniformemente e quase todo ponto (\(\mu\)-(qtp)) para $f$.
    \end{lemma}
    \begin{proof}
    Se $k \in \mathbb{N}$, seja $E_k \subset X$ tal que $\mu(E_k) < 2^{-k}$ e $(f_n)$ é uniformemente convergente em $X \setminus E_k$. Defina $F_k = \bigcup_{i=k}^{\infty} E_i$, de modo que $F_k \subset X$ e $\mu(F_k) < 2^{-(k-1)}$. Note que $(f_n)$ converge uniformemente em $X \setminus F_k$. Defina $g_k$ por
    \begin{equation*}
        g_k(x) =
        \begin{cases}
            \lim f_n(x), & x \notin F_k, \\
            0, & x \in F_k.
        \end{cases}
    \end{equation*}

    Observamos que a sequência $(F_k)$ é decrescente e que, se $F = \bigcap_{k=1}^{\infty} F_k$, então $F \subset X$ e $\mu(F) = 0$. Se $h \leq k$, então $g_h(x) = g_k(x)$ para todo $x \notin F_h$. Portanto, a sequência $(g_k)$ converge em todo $X$ para uma função limite mensurável, que denotamos por $f$. Se $x \notin F_k$, então $f(x) = g_k(x) = \lim f_n(x)$. Segue que $(f_n)$ converge para $f$ em $X \setminus F$, de modo que $(f_n)$ converge para $f$ para quase todo ponto (\(\mu\)-(qtp)) em $X$.

    Para ver que a convergência é quase uniforme, seja $\epsilon > 0$ e escolha $K$ suficientemente grande para que $2^{-(K-1)} < \epsilon$. Então, $\mu(F_K) < \epsilon$, e $(f_n)$ converge uniformemente para $g_K = f$ em $X \setminus F_K$.
\end{proof}

}


%~~~~~~~~~~~~~~~~~~~~~~~~~~~~~~~~~~~~~~~~~~~~~~~~~~~~
% third chapter propositions
%~~~~~~~~~~~~~~~~~~~~~~~~~~~~~~~~~~~~~~~~~~~~~~~~~~~~
\newcommand{\uniformConvergenceLp}{
    \begin{proposition}{Convergência Uniforme em $L_p$}{uniform_convergence_lp}
        Suponha que $\mu(X) < +\infty$ e que $(f_n)$ é uma sequência em $L_p$ que converge uniformemente em $X$ para $f$. Então, $f \in L_p$ e a sequência $(f_n)$ converge em $L_p$ para $f$.
    \end{proposition}
    \begin{proof}
    Seja $\epsilon > 0$ e seja $N(\epsilon)$ tal que $|f_n(x) - f(x)| < \epsilon$ sempre que $n \geq N(\epsilon)$ e $x \in X$. Se $n \geq N(\epsilon)$, então
    \begin{equation}
        \norm{f_n - f}_p = \left( \int_X |f_n(x) - f(x)|^p \, d\mu \right)^{1/p}
        \leq \left( \int_X \epsilon^p \, d\mu \right)^{1/p}
        = \epsilon \mu(X)^{1/p},
    \end{equation}
    de modo que $(f_n)$ converge em $L_p$ para $f$.
\end{proof}

}

\newcommand{\dominatedConvergenceLp}{
    \begin{proposition}{Convergência Dominada em $L_p$}{dominated_convergence_lp}
        Seja $(f_n)$ uma sequência em $L_p$ que converge para uma função mensurável $f$ para quase todo ponto. Se existir uma função $g \in L_p$ tal que
        \begin{equation}\label{eq:dominated_convergence}
            |f_n(x)| \leq g(x), \quad x \in X, \quad n \in \mathbb{N},
        \end{equation}
        então $f \in L_p$ e $(f_n)$ converge em $L_p$ para $f$.
    \end{proposition}
    \begin{proof}
    Pela desigualdade \eqref{eq:dominated_convergence} e a convergência $\mu$-qtp, temos que $|f(x)| \leq g(x)$ para quase todo ponto. Como $g \in L_p$, segue do Corolário \ref{cor:dominated_function_is_integrable} que $f \in L_p$. Agora,
    \begin{equation*}
        |f_n(x) - f(x)|^p \leq [2 g(x)]^p, \quad \mu\text{-}(qtp).
    \end{equation*}
    
    Como $\lim |f_n(x) - f(x)|^p = 0$, \(\mu\text{-}(qtp)\), e $2^p g^p \in L_1$, segue do \nameref{thm:dct} que
    \begin{equation*}
        \lim_{n\rightarrow \infty} \int|f_n - f|^p d\mu = 0.
    \end{equation*}

    Portanto, $(f_n)$ converge em $L_p$ para $f$.
\end{proof}

}

\newcommand{\convergentInLpConvergentInMeasure}{
    \begin{proposition}{Convergência em $L_p$ $\Rightarrow$ convergência na medida}{convergent_in_lp_convergent_in_measure}
        Seja $(f_n)$ uma sequência em $L_p$ tal que $f_n\rightarrow f$ em $L_p$. Então $f_n\rightarrow f$ na medida.
    \end{proposition}
    \begin{proof}
    Fixe um $\varepsilon >0$. Seja $E_{n} = \{x\in X ; \ |f_n(x) - f(x)| \geq \varepsilon \}$. Vamos mostrar que a medida deste conjunto é limitada é vai para zero no limite. Observe que 
    \begin{equation*}
        \norm{f_n - f}_p=\left(\int |f_n - f|^p \right)^{\frac{1}{p}}\geq \left(\int_{E_n} |f_n - f|^p \right)^{\frac{1}{p}} \geq \left(\int_{E_n} \varepsilon^p \right)^{\frac{1}{p}}= \varepsilon \mu(E_n)^{\frac{1}{p}} \geq 0.
    \end{equation*}
    
    Sabemos que $\lim \norm{f_n - f}_p = 0$ por $f_n\rightarrow f$ em $L_p$ por hipótese. Tomando o limite dos dois lados, concluímos, pelo teorema do confronto, que $\mu(E_{n})\rightarrow 0$. Pela arbitrariedade da escolha de $\varepsilon$ temos, por fim, que $f_n \to f$ em medida.
\end{proof}
}

\newcommand{\cauchyInMeasure}{
    \begin{proposition}{Convergência de Sequência Cauchy na Medida}{cauchy_in_measure}
        Seja $(f_n)$ uma sequência de funções reais mensuráveis que é de Cauchy em medida. Então, existe uma subsequência $(f_{n_j})$ que converge para $f$ \(\mu\)-qtp e existe $f$ mensurável tal que $f_n \rightarrow f$ na medida. Além disso, se $f_n\rightarrow g$ na medida, então $f=g$ \(\mu\)-qtp.
    \end{proposition}
    \begin{proof}Vamos começar construindo uma subsequência especial de onda vamos tirar propriedades boas. A construção começa pela hipótese de que $f_n$ é de Cauchy na medida. Escolha $\varepsilon_j = 2^{-j}$, então

\begin{equation*}
    \mu\left(\{x\in X; \ \abs{f_n(x)-f_m(x)} \geq 2^{-j}\}\right)=0.
\end{equation*}
quando $n$ e $m$ tendem ao infinito. Em particular, se $m$ for o sucessor de $n$, temos que

\begin{equation*}
    \lim_{n\rightarrow \infty}\mu\left(\{x\in X; \ \abs{f_n(x)-f_{n+1}(x)} \geq 2^{-j}\}\right)=0.
\end{equation*}

Em outras palavras, para todo $\varepsilon_j=2^{-j}$, existe $N(j)\in\N$ tal que $n\geq N(j)$,

\begin{equation*}
    \mu\left(\{x\in X; \ \abs{f_n(x)-f_{n+1}(x)} \geq 2^{-j}\}\right) < \varepsilon_j = 2^{-j}.
\end{equation*}

Para formar a subsequência escolhemos $j=1$ e $n_1$ o menor natural maior ou igual a $N(1)$. Assim, 

\begin{equation*}
    \mu\left(\{x\in X; \ \abs{f_{n_1}(x)-f_{n_1+1}(x)} \geq 2^{-1}\}\right) < 2^{-1}.
\end{equation*}

Definiremos os dois primeiros elementos da nossa subsequência, $g_1=f_{n_1}$ e $g_2=f_{n_1+1}$. Para simplificar a representação, chamaremos

\begin{equation*}
    E_1=\{x\in X; \ \abs{g_1(x)-g_{1+1}} \geq 2^{-1}\}, \quad \mu(E_1) < 2^{-1}.
\end{equation*}

Seguimos definindo dessa forma para garantir que a subsequência $(g_j)$ é tal que $\mu(E_j) \leq 2^{-j}$ para todo $j$ fazendo a indução.

Se $F_k = \bigcup_{j=k}^{\infty} E_j$, então $\mu(F_k) \leq \sum_{j=k}^{\infty} 2^{-j} = 2^{1-k}$, e se $x \notin F_k$, para $i \geq j \geq k$, temos
    \begin{equation}\label{prop:cauchy_in_measure/eq1}
        |g_j(x) - g_i(x)| \leq \sum_{l=j}^{i-1} |g_{l+1}(x) - g_l(x)| \leq \sum_{l=j}^{i-1} 2^{-l} \leq 2^{1-j}.
    \end{equation}

    Assim, $\{g_j\}$ é de Cauchy pontualmente em $F_k^c$. Isso é razoável, pois $F_k$ é a união de conjuntos onde a sequência não é Cauchy. Definimos $$F = \bigcap_{k=1}^\infty F_k = \limsup E_j\Rightarrow \mu(F) = \lim \mu(F_k) = 0.$$ No passo acima, usamos \nameref{prop:measure_is_continuous_from_above}. Como $\mu(F) = 0$, definimos
    
    \begin{equation*}
        f=\begin{cases}
            \lim g_j(x), & x \notin F, \\
            0, & x \in F.
        \end{cases}
    \end{equation*}
    
    Então, $f$ é mensurável (fica de exercício) e $g_j \to f$ (\(\mu\)-qtp), neste caso $F$ é o conjunto de medida nula em questão. Além disso, \eqref{prop:cauchy_in_measure/eq1} mostra que $|g_j(x) - f(x)| \leq 2^{1-j}$ para $x \notin F_k$ e $j \geq k$. Como $\mu(F_k) \to 0$ quando $k \to \infty$, segue que $g_j \to f$ em medida. Mas $f_n \to f$ em medida, pois
    \begin{equation*}
        \{x : |f_n(x) - f(x)| \geq \epsilon\} \subset \{x : |f_n(x) - g_j(x)| \geq \frac{\epsilon}{2}\} \cup \{x : |g_j(x) - f(x)| \geq \frac{\epsilon}{2}\},
    \end{equation*}
    e os conjuntos à direita têm medida pequena para $n$ e $j$ suficientemente grandes. Da mesma forma, se $f_n \to g$ em medida,
    \begin{equation*}
        \{x : |f(x) - g(x)| \geq \epsilon\} \subset \{x : |f(x) - f_n(x)| \geq \frac{\epsilon}{2}\} \cup \{x : |f_n(x) - g(x)| \geq \frac{\epsilon}{2}\}
    \end{equation*}
    para todo $n$, e portanto $\mu(\{x : |f(x) - g(x)| \geq \epsilon\}) \to 0$ para todo $\epsilon$. Deixando $\epsilon \to 0$, concluímos que $f = g$ \((\mu\)-qtp).


\end{proof}
}

\newcommand{\convergenceInMeasureLpConvergence}{
    \begin{proposition}{Convergência na medida dominada}{convergence_in_measure_lp_convergence}
        Seja $(f_n)$ uma sequência de funções em $L_p$ que converge em medida para $f$, e seja $g \in L_p$ tal que
        \begin{equation*}
            |f_n(x)| \leq g(x), \quad \mu\text{-}(qtp).
        \end{equation*}
        Então $f \in L_p$ e $(f_n)$ converge em $L_p$ para $f$.
    \end{proposition}
    \begin{proof}
    Se $(f_n)$ não converge em $L_p$ para $f$, então existe uma subsequência $(g_k)$ de $(f_n)$ e um $\epsilon > 0$ tal que
    \begin{equation}\label{prop:convergence_in_measure_lp_convergence/eq1}
        \norm{g_k - f}_p > \epsilon \quad \text{para } k \in \mathbb{N}.
    \end{equation}

    Como $(g_k)$ é uma subsequência de $(f_n)$, segue que $(g_k)$ converge em medida para $f$. Pela Proposição \ref{prop:cauchy_in_measure}, existe uma subsequência $(h_r)$ de $(g_k)$ que converge para quase todo ponto \(\mu\)-qtp e em medida para uma função $h$. Pela parte de unicidade do Corolário \ref{cor:cauchy_measure_corollary}, segue que $h = f$ \(\mu\)-qtp.

    Como $(h_r)$ converge para quase todo ponto \(\mu\)-qtp para $f$ e é dominada por $g$, a Proposição \ref{prop:dominated_convergence_lp} implica que $\norm{h_r - f}_p \to 0$. No entanto, isso contradiz a relação \eqref{prop:convergence_in_measure_lp_convergence/eq1}.

    Portanto, $(f_n)$ converge em $L_p$ para $f$.
\end{proof}

}

\newcommand{\almostUniformConvergenceProposition}{
    \begin{proposition}{Convergência Quase Uniforme e Convergência na Medida}{almost_uniform_convergence_proposition}
        Se uma sequência $(f_n)$ converge quase uniformemente para $f$, então ela converge na medida. Reciprocamente, se uma sequência $(h_n)$ converge na medida para $h$, então alguma subsequência converge quase uniformemente para $h$.
    \end{proposition}
    \begin{proof}
    Suponha que $(f_n)$ converge quase uniformemente para $f$. Sejam $\alpha$ e $\epsilon$ números positivos. Então, existe um conjunto $E_\epsilon \subset X$ com $\mu(E_\epsilon) < \epsilon$ tal que $(f_n)$ converge uniformemente para $f$ em $X \setminus E_\epsilon$. Portanto, se $n$ for suficientemente grande, o conjunto $\{x \in X : |f_n(x) - f(x)| \geq \alpha\}$ deve estar contido em $E_\epsilon$. Isso mostra que $(f_n)$ converge na medida para $f$.

    Reciprocamente, suponha que $(h_n)$ converge na medida para $h$. Segue da Proposição \ref{prop:cauchy_in_measure} que existe uma subsequência $(g_k)$ de $(h_n)$ que converge em medida para uma função $g$, e a prova da Proposição \ref{prop:cauchy_in_measure}  mostra que a convergência é quase uniforme. Como $(g_k)$ converge na medida tanto para $h$ quanto para $g$, segue do Corolário \ref{cor:cauchy_measure_corollary} que $h = g$ \(\mu\)-qtp. Portanto, a subsequência $(g_k)$ de $(h_n)$ converge quase uniformemente para $h$.
\end{proof}

}


%~~~~~~~~~~~~~~~~~~~~~~~~~~~~~~~~~~~~~~~~~~~~~~~~~~~~
% third chapter corollary
%~~~~~~~~~~~~~~~~~~~~~~~~~~~~~~~~~~~~~~~~~~~~~~~~~~~~

\newcommand{\boundedConvergenceLp}{
    \begin{corollary}{Convergência Dominada por Constante}{bounded_convergence_lp}
        Suponha que $\mu(X) < +\infty$, e que $(f_n)$ seja uma sequência em $L_p$ que converge para uma função mensurável $f$ para quase todo ponto \(\mu\)-(qtp). Se existir uma constante $K$ tal que
        \begin{equation}
            |f_n(x)| \leq K, \quad x \in X, \quad n \in \mathbb{N},
            \tag{7.3}
        \end{equation}
        então $f \in L_p$ e $(f_n)$ converge em $L_p$ para $f$.
    \end{corollary}
    \begin{proof}
    É uma consequência direta das Proposições \ref{prop:dominated_convergence_lp} e \ref{prop:uniform_convergence_lp} pois a função constante é $L_p$ em espaço com medida finita.
\end{proof}
}

\newcommand{\cauchyMeasureCorollary}{
    \begin{corollary}{Cauchy em Medida é convergente na medida}{cauchy_measure_corollary}
        Seja $(f_n)$ uma sequência de funções reais mensuráveis que é de Cauchy em medida. Então, existe uma função real mensurável $f$ para a qual a sequência converge em medida. Essa função limite $f$ é unicamente determinada\(\mu\)-qtp.
    \end{corollary}
    \begin{proof}
    Vimos que existe uma subsequência $(f_{n_k})$ que converge em medida para uma função $f$. Para mostrar que toda a sequência $(f_n)$ converge em medida para $f$, observe que, como
    \begin{equation*}
        |f(x) - f_n(x)| \leq |f(x) - f_{n_k}(x)| + |f_{n_k}(x) - f_n(x)|,
    \end{equation*}
    segue que
    \begin{equation*}
        \{x \in X : |f(x) - f_n(x)| \geq \alpha\} \subseteq \{x \in X : |f(x) - f_{n_k}(x)| \geq \frac{\alpha}{2}\} \cup \{x \in X : |f_{n_k}(x) - f_n(x)| \geq \frac{\alpha}{2}\}.
    \end{equation*}
    A convergência em medida de $(f_n)$ para $f$ decorre dessa relação.

    Suponha agora que a sequência $(f_n)$ converge em medida tanto para $f$ quanto para $g$. Como
    \begin{equation*}
        |f(x) - g(x)| \leq |f(x) - f_n(x)| + |f_n(x) - g(x)|,
    \end{equation*}
    segue que
    \begin{equation*}
        \{x \in X : |f(x) - g(x)| \geq \alpha\} \subseteq \{x \in X : |f(x) - f_n(x)| \geq \frac{\alpha}{2}\} \cup \{x \in X : |f_n(x) - g(x)| \geq \frac{\alpha}{2}\},
    \end{equation*}
    de modo que $\mu(\{x \in X : |f(x) - g(x)| \geq \alpha\}) = 0$ para todo $\alpha > 0$. Tomando $\alpha = 1/n$, $n \in \mathbb{N}$, concluímos que $f = g$ \(\mu\)-qtp.
\end{proof}

}



%~~~~~~~~~~~~~~~~~~~~~~~~~~~~~~~~~~~~~~~~~~~~~~~~~~~~
% third chapter theorems
%~~~~~~~~~~~~~~~~~~~~~~~~~~~~~~~~~~~~~~~~~~~~~~~~~~~~
\newcommand{\holder}{
    \begin{theorem}{Desigualdade de Hölder}{holder}
        Sejam $f \in L_p$ e $g \in L_q$, com $1 \leq p, q \leq \infty$ e $\frac{1}{p} + \frac{1}{q} = 1$. Então, $fg\in L_1$ e
        \begin{equation*}
            \int |f(x) g(x)| \, dx \leq \norm{f}_p \norm{g}_q.
        \end{equation*}
    \end{theorem}
    \begin{proof}
    O caso em que $\norm{f}_p = 0$ ou $\norm{g}_q = 0$ é triva. De fato, a desigualdade vira $0 \leq 0$. Suponha então $\norm{f}_p \neq 0$ e $\norm{g}_q \neq 0$.

    Pelo Lema \ref{lmm:power_inequality}, temos que, para quaisquer $a, b > 0$,
    \begin{equation}\label{eq:proof/holder/step1}
        a^p b^q \leq \frac{a}{p} + \frac{b}{q}.
    \end{equation}

    Tome, em \eqref{eq:proof/holder/step1}
    \begin{equation*}
        a = \frac{|f(x)|^p}{\norm{f}_p^p}
        \quad \text{e} \quad
        b = \frac{|g(x)|^q}{\norm{g}_q^q}.
    \end{equation*}
    Assim, temos
    \begin{equation*}
        \int_X \frac{|f(x)g(x)|}{\norm{f}_p \norm{g}_q} d\mu \leq \int_X \frac{|f(x)|^p}{p \norm{f}_p^p} d\mu + \int_X \frac{|g(x)|^q}{q \norm{g}_q^q} d\mu = 1,
    \end{equation*}
    e o resultado segue.
\end{proof}

}

\newcommand{\minkowski}{
    \begin{theorem}{Desigualdade de Minkowski}{minkowski}
        Sejam $f, g \in L_p$, com $1 \leq p \leq \infty$. Então, $f+g\in L_p$ e
        \begin{equation}\label{eq:minkowski}
            \norm{f+g}_p \leq \norm{f}_p + \norm{g}_p.
        \end{equation}
    \end{theorem}
    \begin{proof}
    Se $\norm{f + g}_p = 0$, o resultado é claro. Suponha $\norm{f + g}_p \neq 0$. Perceba que para todo $x \in X$, temos
    \begin{eqnarray*}
        |f(x) + g(x)|^p
        &\leq& \left( |f(x)| + |g(x)| \right)^p\\
        &\leq& \left( \max\{|f(x)|, |g(x)|\} + \max\{|f(x)|, |g(x)|\} \right)^p \\
        &\leq& 2^p \max\{ |f(x)|^p, |g(x)|^p \} \\
       &\leq& 2^p \left( |f(x)|^p + |g(x)|^p \right).
    \end{eqnarray*}
    E daí segue que $f + g \in L_p(X, \Sigma, \mu)$.

    Agora vamos provar \eqref{eq:minkowski}. Se $p = 1$, o resultado é consequência direta da Proposição \ref{prop:function_is_integrable_iff_absolute_value_is}. Suponha $p > 1$. Então
    \begin{equation}\label{proof:minkowski/eq1}
        |f + g|^p = |f + g||f + g|^{p-1}
        \leq \left( |f| + |g| \right) |f + g|^{p-1}.
    \end{equation}

    Se $1/p + 1/q = 1$, temos $(p-1)q = p$, e portanto $|f + g|^{p-1} \in L_q(X, \Sigma, \mu)$. Da \nameref{thm:holder}, temos
    \begin{equation*}
        \int_X |f||f + g|^{p-1} d\mu \leq \left( \int_X |f|^p d\mu \right)^{\frac{1}{p}} \left( \int_X |f + g|^q d\mu \right)^{\frac{1}{q}},
    \end{equation*}
    \begin{equation*}
        \int_X |g||f + g|^{p-1} d\mu \leq \left( \int_X |g|^p d\mu \right)^{\frac{1}{p}} \left( \int_X |f + g|^q d\mu \right)^{\frac{1}{q}}.
    \end{equation*}

    Das duas desigualdades acima e de \ref{proof:minkowski/eq1}, temos que
    \begin{equation*}
        \int_X |f + g|^p d\mu \leq \left( \int_X |f|^p d\mu \right)^{\frac{1}{p}} \left( \int_X |f + g|^q d\mu \right)^{\frac{1}{q}} + \left( \int_X |g|^p d\mu \right)^{\frac{1}{p}} \left( \int_X |f + g|^q d\mu \right)^{\frac{1}{q}}.
    \end{equation*}

    Isso é equivalente a
    \begin{equation*}
        \left( \int_X |f + g|^p d\mu \right)^{\frac{1}{p}} \leq \left( \int_X |f|^p d\mu \right)^{\frac{1}{p}} + \left( \int_X |g|^p d\mu \right)^{\frac{1}{p}},
    \end{equation*}
    e, dividindo ambos os membros por $\left( \int_X |f + g|^p d\mu \right)^{\frac{1}{p}}$, o resultado segue.
\end{proof}

}

\newcommand{\rieszFischer}{
    \begin{theorem}{Teorema de Riesz-Fischer}{riesz_fischer}
        Seja $(f_n)$ uma sequência de Cauchy em $L_p$ com $1 \leq p \leq \infty$. Então, existe $f \in L_p$ tal que $f_n \to f$ em $L_p$.
    \end{theorem}
    \begin{proof}
    Faremos a demonstração do teorema em dois casos. Para o primeiro caso, suponha que $p=\infty$.

    Seja $(f_n)$ uma sequência de Cauchy em $L^\infty$. Dado um inteiro $k \geq 1$, existe um $N_k$ tal que
    \begin{equation}\label{proof:rieszFischer/eq1}
        \norm{f_m - f_n}_\infty \leq \frac{1}{k}, \quad \text{para } m, n \geq N_k.
    \end{equation}
    Portanto, existe um conjunto de medida nula $E_k$ tal que
    \begin{equation*}
        |f_m(x) - f_n(x)| \leq \frac{1}{k}, \quad \forall x \in \Omega \setminus E_k, \quad \text{para } m, n \geq N_k.
    \end{equation*}

    Agora defina $E = \bigcup_{k=1}^\infty E_k$, de forma que $E$ seja um conjunto de medida nula. Para quase todo ponto em $\Omega \setminus E$, a sequência $f_n(x)$ é de Cauchy em $\mathbb{R}$. Logo, $f_n(x) \to f(x)$ $\mu$-qtp em $\Omega \setminus E$. Passando ao limite em \eqref{proof:rieszFischer/eq1} conforme $m \to \infty$, obtemos
    \begin{equation*}
        |f(x) - f_n(x)| \leq \frac{1}{k}, \quad \forall x \in \Omega \setminus E, \quad \text{para } n \geq N_k.
    \end{equation*}

    Concluímos, então, que $f \in L^\infty$ e que $\norm{f - f_n}_\infty \leq \frac{1}{k}$ para $n \geq N_k$. Portanto, $f_n \to f$ em $L^\infty$.

    Agora, consideremos o caso $1 \leq p < \infty$. 

    Seja $(f_n)$ uma sequência de Cauchy em $L_p$. Para concluir a prova, basta mostrar que uma subsequência converge em $L_p$. Extraímos uma subsequência $(f_{n_k})$ tal que
    \begin{equation}\label{proof:rieszFischer/eq2}
        \norm{f_{n_{k+1}} - f_{n_k}}_p \leq \frac{1}{2^k}, \quad \forall k \geq 1.
    \end{equation}

    Procedemos da seguinte forma: escolhemos $n_1$ tal que $\norm{f_m - f_n}_p \leq \frac{1}{2}$ para todo $m, n \geq n_1$; depois, escolhemos $n_2 \geq n_1$ tal que $\norm{f_m - f_n}_p \leq \frac{1}{2^2}$ para todo $m, n \geq n_2$; e assim por diante. Com isso, afirmamos que $(f_{n_k})$ converge em $L_p$. Para simplificar a notação, escrevemos $f_k$ em vez de $f_{n_k}$, de modo que temos
    \begin{equation*}
        \norm{f_{k+1} - f_k}_p \leq \frac{1}{2^k}, \quad \forall k \geq 1.
    \end{equation*}

    Definimos
    \begin{equation*}
        g_n(x) = \sum_{k=1}^n |f_{k+1}(x) - f_k(x)|,
    \end{equation*}
    de modo que
    \begin{equation*}
        \norm{g_n}_p \leq 1.
    \end{equation*}

    Como consequência do \nameref{thm:mct}, $g_n(x)$ tende a um limite finito, que chamamos $g(x)$, para quase todo ponto \(\mu\)-qtp em $\Omega$, com $g \in L_p$. Por outro lado, para $m \geq n \geq 2$, temos
    \begin{equation*}
        |f_m(x) - f_n(x)| \leq |f_m(x) - f_{m-1}(x)| + \cdots + |f_{n+1}(x) - f_n(x)| \leq g(x) - g_{n-1}(x).
    \end{equation*}

    Portanto, para quase todo ponto em $\Omega$, $f_n(x)$ é de Cauchy e converge para um limite finito, que chamamos $f(x)$. Temos, então, para quase todo ponto em $\Omega$, que
    \begin{equation*}
        |f(x) - f_n(x)| \leq g(x), \quad \text{para } n \geq 2,
    \end{equation*}
    e, em particular, $f \in L_p$. Finalmente, concluímos pela convergência dominada que $\norm{f_n - f}_p \to 0$, uma vez que $|f_n(x) - f(x)|^p \to 0$ para quase todo ponto e também $|f_n - f|^p \leq g^p \in L_1$.
\end{proof}

}

\newcommand{\egoroff}{
    \begin{theorem}{Teorema de Egoroff}{egoroff}
        Suponha que $\mu(X) < +\infty$ e que $(f_n)$ é uma sequência de funções reais mensuráveis que converge para quase todo ponto (\(\mu\)-qtp) em $X$ para uma função real mensurável $f$. Então, a sequência $(f_n)$ converge quase uniformemente e na medida para $f$.
    \end{theorem}
    \begin{proof}
    Por hipótese, $f_n$ converge para $f$ para quase todo ponto. Portanto, existe um conjunto de medida nula $N$ tal que $f_n\chi_N$ converge para $f\chi_N$ para todo ponto em $X\setminus N$. Em símbolos,

    \begin{equation*}
        f_n \rightarrow f (\mu \text{-qtp}) \Rightarrow \exists N\in \Sigma, \mu (N)=0; \ f_n(x) \rightarrow f(x) \ \forall x\in X\setminus N.
    \end{equation*}

    De fato, é preferível trabalhar apenas nos lugares onde temos a convergência pontual. Assim, ao invés de olharmos para a nossa função toda, podemos olhar só para a função restrita ao subconjunto onde temos a convergência pontual garantida, elimiando um detalhe técnico da hipótese. 
    
    Seguiremos, então, partindo do princípio que $(f_n)$ converge em todo ponto de $X$ para $f$. Caso isso não ocorra, basta olhar para $f_n\chi_N$ que converge pontualmente para $f\chi_N$. Portanto, não há perda de generalidade.

    Para o próximo passo, vamos usar a convergência pontual para criar um conjunto vazio. Pela Definição de \nameref{def:pointwise_convergence}, em um $x\in X$ onde a sequência converge, temos que

    \begin{equation*}
        \forall \varepsilon >0, \exists N(\varepsilon, x) \in \N ; n\geq N(\varepsilon, x)\Rightarrow \abs{f_n(x) - f(x)} < \varepsilon.
    \end{equation*}

    Assim, se $x\in X$ é um ponto onde a sequência NÃO converge, existe $\varepsilon>0$ tal que, para todo natural $N$, é possível achar um $n>N$ tal que $\abs{f_n(x)-f(x)}\geq\varepsilon$. Em símbolos,

    \begin{equation}\label{proof:egoroff/eq1}
        \exists \, \varepsilon >0; \forall N \in \N , \exists \, n\geq N(\varepsilon, x); \abs{f_n(x) - f(x)} \geq \varepsilon.
    \end{equation}
    
    Guardemos esta ideia. Vamos definir um conjunto de pontos de $x$ onde vale a segunda parte da sentença acima. Sejam $m, n \in \mathbb{N}$, e defina
    \begin{equation*}
        E_n(m) = \bigcup_{k=n}^{\infty} \left\{ x \in X : |f_k(x) - f(x)| \geq \frac{1}{m} \right\},
    \end{equation*}
    de modo que $E_n(m)$ pertence a $X$ e $E_{n+1}(m) \subseteq E_n(m)$ (toda vez que aumentamos o $n$, tiramos um elemento da união, então os conjuntos vão ficando menores e a sequência é decrescente). Note que, se $x\in E_n(m)$, então existe $k\geq n$ tal que $\abs{f_k(x) - f(x)} \geq \frac{1}{m}$. Repare na semelhança com \eqref{proof:egoroff/eq1}.

    Agora, considere a interseção de todos os $E_n(m)$, dada por $\cup_{n=1}^{\infty} E_n(m)$. Se $x$ pertence a este conjunto então, para todo $n$ natural, $x\in E_n(m)$. Como argumentamos no parágrafo acima, isto significa que, para todo $n$, existe $k\geq n$ tal que $\abs{f_k(x) - f(x)} \geq \frac{1}{m}$. Repare mais uma vez na semelhança com \eqref{proof:egoroff/eq1}. Desta vez, estamos com a sentença completa! Neste caso, $\frac{1}{m}$ faz o papel do nosso $\varepsilon$. Concluimos, então, que $f_n(x)$ diverge para todo ponto em $\cup_{n=1}^{\infty} E_n(m)$.

    Eis aqui a ideia inteligente. Como $f_n(x) \to f(x)$ para todo $x \in X$ (por hipótese), segue que não existem pontos onde a sequência diverge, ou seja,
    \begin{equation*}
        \bigcap_{n=1}^{\infty} E_n(m) = \varnothing.
    \end{equation*}

    Como $\mu(X) < +\infty$ e a sequência é descrescente, estamos nas hipóteses da \nameref{prop:measure_is_continuous_from_above}. Assim
    
    \begin{equation*}
        0=\mu(\varnothing)=\mu\left(\bigcap_{n=1}^{\infty} E_n(m)\right) = \lim_{n\rightarrow \infty} \mu (E_n),
    \end{equation*}
    
    e concluimos que $\mu(E_n(m)) \to 0$ conforme $n \to +\infty$.
    
    Vamos usar mais uma vez as definições básicas de limite vindas da análise real. Comecemos por fixar um $\delta >0$. Agora, como sabemos que $\mu(E_n(m))$ converge para $0$, temos, para cada $m$, que $\frac{\delta}{2^m}>0$ e, portanto, podemos afirmar que
    
    \begin{equation*}
        \exists k_m \in \N ; \abs{\mu(E_{k_m}(m))-0}=\mu(E_{k_m}(m)) < \frac{\delta}{2^m}.
    \end{equation*}
    
    Defina $E_\delta = \bigcup_{m=1}^{\infty} E_{k_m}(m)$, de modo que $E_\delta \subset X$ e $\mu(E_\delta) < \delta$ (basta usar \nameref{prop:measure_is_subadditive} e soma de progressão geométrica para verificar este fato). Lembre-se que estamos tentando provar que a sequência converge quase uniformemente. Jé conseguimos construir um conjunto de medida arbitrariamente pequena, resta mostrar que a sequência converge uniformemente no seu complemento.
    
    Observe que, se $x \notin E_\delta$, então $x \notin E_{k_m}(m)$. Dessa forma, 
    \begin{equation*}
        |f_k(x) - f(x)| < \frac{1}{m}
    \end{equation*}
    para todo $k \geq k_m$. Portanto, $(f_k)$ é uniformemente convergente no complemento de $E_\delta$.
\end{proof}

}




%%%%%%%%%%%%%%%%%%%%%%%%%%%%%%%%%%%%%%%%%%%%%%%%%%%%
%%%%%%%%%%%%%%%%%%%%%%%%%%%%%%%%%%%%%%%%%%%%%%%%%%%%
%
%                APPENDIX STATEMENTS
%
%%%%%%%%%%%%%%%%%%%%%%%%%%%%%%%%%%%%%%%%%%%%%%%%%%%%
%%%%%%%%%%%%%%%%%%%%%%%%%%%%%%%%%%%%%%%%%%%%%%%%%%%%

%~~~~~~~~~~~~~~~~~~~~~~~~~~~~~~~~~~~~~~~~~~~~~~~~~~~~
% appendix lemmas
%~~~~~~~~~~~~~~~~~~~~~~~~~~~~~~~~~~~~~~~~~~~~~~~~~~~~
\newcommand{\basisOfR}{
    \begin{proposition}{Bases de $\R$}{basis_of_r}
        Os conjuntos abaixo são bases para $\R$.
        \begin{enumerate}
            \item $C_1 = \{(a,b) : a,b \in \mathbb{R}\}$,
            \item $C_2 = \{[a,b] : a,b \in \mathbb{R}\}$,
            \item $C_3 = \{[a,b) : a,b \in \mathbb{R}\}$,
            \item $C_4 = \{(a,b] : a,b \in \mathbb{R}\}$,
            \item $C_5 = \{(a,\infty) : a \in \mathbb{R}\}$,
            \item $C_6 = \{(\infty,a) : a \in \mathbb{R}\}$,
            \item $C_7 = \{[a,\infty) : a \in \mathbb{R}\}$,
            \item $C_8 = \{(\infty,a] : a \in \mathbb{R}\}$.
        \end{enumerate}
    \end{proposition}
}



%~~~~~~~~~~~~~~~~~~~~~~~~~~~~~~~~~~~~~~~~~~~~~~~~~~~~
% appendix propositions
%~~~~~~~~~~~~~~~~~~~~~~~~~~~~~~~~~~~~~~~~~~~~~~~~~~~~
\newcommand{\finiteProductOfOpenSetsIsOpen}{
    \begin{proposition}{O produto cartesiano de abertos é aberto}{finite_product_of_open_sets_is_open}
        Sejam $(X_1,\tau_1),(X_2,\tau_2)$ \nameref{def:topological_space} e $X_1\times X_2$ equipado com a \nameref{def:product_topology} $\tau$. Se $U\in\tau_1$ e $V\in\tau_2$, então $U\times V\in\tau$. Por indução, o resultado vale para famílias finitas.
    \end{proposition}
}

\newcommand{\metrizableSeparableSpaceIsSecondCountable}{
    \begin{proposition}{Espaços metrizáveis separáveis possuem base enumerável}{metrizable_separable_space_is_second_countable}
        Seja $(X,\tau)$ um \nameref{def:topological_space} separável e metrizável. Então, $X$ admite uma base $\mathcal{B}$ enumerável.
    \end{proposition}
}

\newcommand{\rnIsSeparable}{
    \begin{proposition}{\texorpdfstring{$\R^n$}{Rn} é separável}{rn_is_separable}
        Para todo $n\in\N$. $\Q_n$ é um subconjunto denso e enumerável de $\R^n$ e $\R^n$ é separável.
    \end{proposition}
}

%~~~~~~~~~~~~~~~~~~~~~~~~~~~~~~~~~~~~~~~~~~~~~~~~~~~~
% appendix corollary
%~~~~~~~~~~~~~~~~~~~~~~~~~~~~~~~~~~~~~~~~~~~~~~~~~~~~

%~~~~~~~~~~~~~~~~~~~~~~~~~~~~~~~~~~~~~~~~~~~~~~~~~~~~
% appendix theorems
%~~~~~~~~~~~~~~~~~~~~~~~~~~~~~~~~~~~~~~~~~~~~~~~~~~~~