% modules\theorems.tex

%%%%%%%%%%%%%%%%%%%%%%%%%%%%%%%%%%%%%%%%%%%%%%%%%%%%
%%%%%%%%%%%%%%%%%%%%%%%%%%%%%%%%%%%%%%%%%%%%%%%%%%%%
%
%            FIRST CHAPTER STATEMENTS
%
%%%%%%%%%%%%%%%%%%%%%%%%%%%%%%%%%%%%%%%%%%%%%%%%%%%%
%%%%%%%%%%%%%%%%%%%%%%%%%%%%%%%%%%%%%%%%%%%%%%%%%%%%



%~~~~~~~~~~~~~~~~~~~~~~~~~~~~~~~~~~~~~~~~~~~~~~~~~~~~
% first chapter lemmas
%~~~~~~~~~~~~~~~~~~~~~~~~~~~~~~~~~~~~~~~~~~~~~~~~~~~~

\newcommand{\sigmaAlgebraIntersection}{
    \begin{lemma}{Interseção de \texorpdfstring{$\sigma$}{sigma}-álgebras é \texorpdfstring{$\sigma$}{sigma}-álgebra}{sigma_algebra_intersection}
        Seja $X$ um conjunto e $\{\Sigma_i\}_{i\in\mathcal{I}}$ uma família de \nameref{def:sigma_algebra} de $X$. Então, $\cap_{i\in\mathcal{I}} \Sigma_i $ é uma \texorpdfstring{$\sigma$}{sigma}-álgebra de $X$.
    \end{lemma}
    \begin{proof}
    Denotemos $\Sigma = \cap_{i\in\mathcal{I}}\Sigma_i$. Como $A_n\in \Sigma$ para todo $n\in \N$, então $A_n^{c}\in \Sigma$ para todo $n$ e $\left(A_n^{c}\right)_{n=1}^\infty \subset \Sigma$. Assim $\bigcup_{n=1}^\infty \left( A_n^{c} \right)\in \Sigma$. Note bem, pelo \nameref{thm:de_morgan}, 
    \begin{equation*}
        \bigcup_{n=1}^\infty \left( A_n^{c} \right) = \left( \bigcap_{n=1}^{\infty} A_n \right)^{c} \in \Sigma.
    \end{equation*}
    Portanto $\left(\bigcup_{n=1}^\infty \left( A_n^{c} \right)\right)^{c} = \bigcap_{n=1}^{\infty} A_n \in \Sigma$.
\end{proof}
}

\newcommand{\generatedSigmaAlgebraIsUnique}{
    \begin{lemma}{Unicidade da \texorpdfstring{$\sigma$}{sigma}-álgebra gerada}{generated_sigma_algebra_is_unique}
        Seja $X$ um conjunto e $\mathcal{C}\subset \powerset{X}$. Então, a \nameref{def:generated_sigma_algebra} por $\mathcal{C}$ é única.
    \end{lemma}
    \begin{proof}
    Sejam $\mathcal{A}$ e $\mathcal{B}$ duas $\sigma$-álgebras geradas por $\mathcal{C}$. Então, pelo primeiro item da definição de $\mathcal{A}$, temos que $\mathcal{C}\subset \mathcal{A}$. Por outro lado, usando o segundo item da definição de $\mathcal{B}$, temos que $\mathcal{B}\subset \mathcal{A}$. Analogamente, mostramos que $\mathcal{A}\subset \mathcal{B}$ e $\mathcal{A}=\mathcal{B}$.
\end{proof}
}

\newcommand{\sigmaAlgebraGeneratedBySubset}{
    \begin{lemma}{\texorpdfstring{$\sigma$}{sigma}-álgebra Gerada Por Subconjunto}{sigma_algebra_generated_by_subset}
        Seja $\mathcal{C}\in \powerset{X}$. Se $E\in \sigma [\mathcal{C}]$, então $\sigma [E] \subset \sigma [\mathcal{C}]$.
    \end{lemma}
    \begin{proof}
    Temos, por hipótese que $\sigma [\mathcal{C}]$ é uma $\sigma$-álgebra de $X$ que contém $E$. Logo, pelo segundo item da Definição \ref{def:generated_sigma_algebra}, $\sigma [E] \subset \sigma [\mathcal{C}]$.
\end{proof}
}



%~~~~~~~~~~~~~~~~~~~~~~~~~~~~~~~~~~~~~~~~~~~~~~~~~~~~
% first chapter propositions
%~~~~~~~~~~~~~~~~~~~~~~~~~~~~~~~~~~~~~~~~~~~~~~~~~~~~
\newcommand{\generatedSigmaAlgebraCharacterization}{
    \begin{proposition}{Caracterização da \texorpdfstring{$\sigma$}{sigma}-álgebra gerada}{generated_sigma_algebra_characterization}
        Sejam $X$ um conjunto, $\mathcal{C}\subset \powerset{X}$ e $\mathcal{J}=\{\Sigma_j\}$ a família de todas as $\sigma$-álgebras de $X$ que contém $\mathcal{C}$. Então,
        \begin{equation*}
            \sigma [\mathcal{C}] = \bigcap_{\Sigma_j\in\mathcal{J}} \Sigma_{j}.
        \end{equation*}
    \end{proposition}
    \begin{proof}
    Primeiramente, note que $\mathcal{J}$ não é uma família vazia, pois $\powerset{X}$ é uma $\sigma$-álgebra (Exemplo \ref{exm:power_set_sigma_algebra}) que contém $\mathcal{C}$ trivialmente, logo $\mathcal{C}$ está na interseção. Em segundo lugar, observe que, pelo Lema \ref{lmm:sigma_algebra_intersection}, a interseção dos elementos de $\mathcal{J}$ é uma $\sigma$-álgebra. Por último, verifique que $\bigcap_{\Sigma_j\in\mathcal{J}} \Sigma_{j} \subseteq \Sigma_j$ para todo $\Sigma_j \in \mathcal{J}$. Portanto, $\bigcap_{\Sigma_j\in\mathcal{J}} \Sigma_{j}$ é uma $\sigma$-álgebra gerada por $\mathcal{C}$ e, pelo Lema \ref{lmm:generated_sigma_algebra_is_unique}, concluímos que $\sigma [\mathcal{C}] = \bigcap_{\Sigma_j\in\mathcal{J}} \Sigma_{j}$.
\end{proof}
}

\newcommand{\productSigmaAlgebraOfCountableFamily}{
    \begin{proposition}{\sigmaAlg produto de família enumerável}{product_sigma_algebra_of_countable_family}
        Seja $I$ um conjunto enumerável e $\{X_i\}_{i\in I}$ uma família de espaços mensuráveis. Então
    \begin{equation*}
        \bigotimes_{i\in I} \Sigma_i=\generateSigmaAlg{\left\{\prod_{i\in I} E_i; E_i\in \Sigma_i\right\}}.
    \end{equation*}
    \end{proposition}
    \begin{proof}
    Vamos denotar por $\mathcal{F}$ a família de conjuntos do enunciado. Considere $\mathcal{C}$ a família como na Definição \ref{def:product_sigma_algebra}. Pretendemos mostrar que  $\generateSigmaAlg{\mathcal{C}}=\generateSigmaAlg{\mathcal{F}}$. Uma forma natural de chegar nesse resultado é mostrar que $\generateSigmaAlg{\mathcal{C}}\subseteq\generateSigmaAlg{\mathcal{F}}$ e $\generateSigmaAlg{\mathcal{F}}\subseteq\generateSigmaAlg{\mathcal{C}}$.
    
    Por um lado, podemos mostrar (ver Exercício X) que
    
    \begin{equation*}
        \prod_{i\in I} E_i = \bigcap_{i\in I} \pi_{i}^{-1} E_i.
    \end{equation*}
    
    Usando o Lema \ref{lmm:sigma_algebra_intersection}, mostramos que $\bigcap_{i\in I} \pi_{i}^{-1} (E_i) \in \generateSigmaAlg{\mathcal{C}}$. Com isso, concluímos que $\mathcal{F} \subseteq \generateSigmaAlg{\mathcal{C}}$ e, pelo Lema \ref{lmm:sigma_algebra_generated_by_subset}, $\generateSigmaAlg{\mathcal{F}}=\generateSigmaAlg{\mathcal{C}}$.
    
    Resta mostrar a segunda inclusão. Usaremos a mesma ideia. Para tanto, precisamos mostrar (ver Exercício Y) que, para todo $E_i\in\Sigma_i$ onde $i\in I$,
    
    \begin{eqnarray*}
        \pi_{i}^{-1} (E_i) 
        &=& X_1 \times \dots \times E_i \times\dots \\
        &=& \prod_{j\in I} E_j, \quad j\neq i \Rightarrow E_j=X_j.
    \end{eqnarray*}
    
    Sabemos que cada $X_j\in\Sigma_j$ pela definição de $\sigma$-álgebra e $E_i\in\Sigma_i$ por hipótese. Logo, $\pi_{i}^{-1} (E_i) \in \mathcal{F} \subset \generateSigmaAlg{\mathcal{F}}$. Usamos o Lema \ref{lmm:sigma_algebra_generated_by_subset} mais uma vez para concluir que $\generateSigmaAlg{\mathcal{C}}\subseteq\generateSigmaAlg{\mathcal{F}}$.
\end{proof}
}

\newcommand{\productSigmaAlgebraOfFamiliesGeneratedBySet}{
    \begin{proposition}{\sigmaAlg produto de famílias geradas por conjuntos}{product_sigma_algebra_of_families_generated_by_sets}
        Seja $\{X_i\}_{i\in I}$ uma família de espaços mensuráveis tal que cada $\Sigma_i=\sigma [\mathcal{C}_i]$. Então
    \begin{equation*}
        \bigotimes_{i\in I} \Sigma_i=\generateSigmaAlg{\left\{\pi_i^{-1}(E_i); E_i\in \mathcal{C}_i, i\in I\right\}}.
    \end{equation*}
    \end{proposition}
    \begin{proof}
    Vamos denotar por $\mathcal{F}$ a família de conjuntos do enunciado. Considere $\mathcal{C}$ a família como na Definição \ref{def:product_sigma_algebra}. Pretendemos mostrar que  $\generateSigmaAlg{\mathcal{C}}=\generateSigmaAlg{\mathcal{F}}$. Uma forma natural de chegar nesse resultado é mostrar que $\generateSigmaAlg{\mathcal{C}}\subseteq\generateSigmaAlg{\mathcal{F}}$ e $\generateSigmaAlg{\mathcal{F}}\subseteq\generateSigmaAlg{\mathcal{C}}$.
    
    Por um lado, podemos mostrar (ver Exercício X) que
    
    \begin{equation*}
        \prod_{i\in I} E_i = \bigcap_{i\in I} \pi_{i}^{-1} E_i.
    \end{equation*}
    
    Usando o Lema \ref{lmm:sigma_algebra_intersection}, mostramos que $\bigcap_{i\in I} \pi_{i}^{-1} (E_i) \in \generateSigmaAlg{\mathcal{C}}$. Com isso, concluímos que $\mathcal{F} \subseteq \generateSigmaAlg{\mathcal{C}}$ e, pelo Lema \ref{lmm:sigma_algebra_generated_by_subset}, $\generateSigmaAlg{\mathcal{F}}=\generateSigmaAlg{\mathcal{C}}$.
    
    Resta mostrar a segunda inclusão. Usaremos a mesma ideia. Para tanto, precisamos mostrar (ver Exercício Y) que, para todo $E_i\in\Sigma_i$ onde $i\in I$,
    
    \begin{eqnarray*}
        \pi_{i}^{-1} (E_i) 
        &=& X_1 \times \dots \times E_i \times\dots \\
        &=& \prod_{j\in I} E_j, \quad j\neq i \Rightarrow E_j=X_j.
    \end{eqnarray*}
    
    Sabemos que cada $X_j\in\Sigma_j$ pela definição de $\sigma$-álgebra e $E_i\in\Sigma_i$ por hipótese. Logo, $\pi_{i}^{-1} (E_i) \in \mathcal{F} \subset \generateSigmaAlg{\mathcal{F}}$. Usamos o Lema \ref{lmm:sigma_algebra_generated_by_subset} mais uma vez para concluir que $\generateSigmaAlg{\mathcal{C}}\subseteq\generateSigmaAlg{\mathcal{F}}$.
\end{proof}
}

\newcommand{\productSigmaAlgebraOfMetricSpaces}{
    \begin{proposition}{\sigmaAlg produto de família de espaços métricos}{product_sigma_algebra_of_metric_spaces}
        Seja $\{X_i\}_{i=1}^{n}$ uma família finita de espaços métricos com a topologia, $\tau_i$, induzida pela distância. Então
    \begin{equation*}
        \bigotimes_{i=1}^{n} \borel{X_i}\subseteq \borel{\prod_{i=1}^{n}X_i}.
    \end{equation*}
    \end{proposition}
    \begin{proof}
    Note bem, estamos trabalhando com um número finito de espaços métricos, cuja $\sigma$-álgebra é gerada pelos abertos de cada conjunto. Assim, valem, as hipóteses do Corolário \ref{cor:product_sigma_algebra_of_countable_generated_families}. Com isso, ganhamos que a $\sigma$-álgebra produto de $X$ é gerada pelo produto de todos os abertos. 

    \begin{equation*}
        \mathcal{C}\coloneq \left\{\prod_{i=1}^{n} E_i; E_i \in \tau_i \right\} \Longrightarrow \generateSigmaAlg{\mathcal{C}} = \bigotimes_{i=1}^{n} \mathcal{B}(X_i)
    \end{equation*}
    
    Por outro lado, o produto cartesiano de finitos abertos na \nameref{def:product_topology} (ver Proposição \ref{prop:finite_product_of_open_sets_is_open}) e, portanto, pertence a $\borel{X}$. Portanto, $\mathcal{C}\in\borel{X}$ e, pelo Lema \ref{lmm:sigma_algebra_generated_by_subset}, temos que
    \begin{equation*}
        \generateSigmaAlg{\mathcal{C}} = \bigotimes_{i=1}^{n} \mathcal{B}(X_i) \subseteq \borel{X}.
    \end{equation*}
\end{proof}
}

\newcommand{\measureIsMonotonic}{
    \begin{proposition}{Monotonicidade da Medida}{measure_is_monotonic}
        Seja $(X,\Sigma,\mu)$ um espaço de medida. Então, $\mu$ é uma \nameref{def:monotonic_function}.
    \end{proposition}
    \begin{proof}
    Sejam $E,F\in\Sigma$ tais que $E\subset F$. Podemos escrever $F$ como a união disjunta entre $E$ e $F\setminus E$. Como a medida $\mu$ é aditiva contável, então $\mu(F)=\mu(E)+\mu(F\setminus E)$. Note que $F\setminus E=F\cap E^{c}$, logo $F\setminus E\in \Sigma$. Como, $\mu(A)\geq 0$ para todo $A\in \Sigma$, temos que $\mu(E)\leq \mu(E)+\mu(F\setminus E)=\mu(F)$.
\end{proof}s
}

\newcommand{\measureIsSubadditive}{
    \begin{proposition}{Subaditividade da medida}{measure_is_subadditive}
        Seja $(X,\Sigma,\mu)$ um espaço de medida. Então $\mu$ é uma \nameref{def:countably_subadditive_function}.
    \end{proposition}
    \begin{proof}
    Vamos construir uma sequência disjunta $(F_n)$ a partir da sequência $(E_n)$ dada. Seja $F_1=E_1$ e defina
    \begin{equation*}
        F_k = E_k \setminus \left(\bigcup_{n=1}^{k-1} E_n\right).
    \end{equation*}
    Por construção, $F_i \cap F_j =\varnothing$ para quaisquer $i,j\in \N$ tais que $i\neq j$. Também temos que $F_n \subset E_n$ para todo $n\in \N$. Por último, notamos que
    \begin{equation}\label{eq/proof/measure_subadditivity}
        \bigcup_{n=1}^{\infty} E_n = \bigcup_{n=1}^{\infty} F_n \Longrightarrow \mu \left(\bigcup_{n=1}^{\infty} E_n\right) = \mu \left(\bigcup_{n=1}^{\infty} F_n\right).
    \end{equation}
    Como $(F_n)$ é disjunta e $\mu$ é \nameref{def:countably_additive_function}, concluímos, a partir da Equação \eqref{eq/proof/measure_subadditivity} que
    \begin{equation}\label{eq/proof/measure_monotonicity}
        \mu \left(\bigcup_{n=1}^{\infty} E_n\right) = \mu \left(\bigcup_{n=1}^{\infty} F_n\right) = \sum_{n=1}^{\infty} \mu (F_n).
    \end{equation}
    Agora, aplicando o fato de que $F_n\subset E_n$ \hyperref[prop:measure_is_monotonic]{medida é uma função monotônica}, temos que $\mu(F_n) \leq \mu(E_n)$ para todo $n\in N$. Assim, as séries também seguem esta desigualdade. Portanto, pela Equação \eqref{eq/proof/measure_monotonicity},
    \begin{equation*}
        \mu \left(\bigcup_{n=1}^{\infty} E_n\right) = \sum_{n=1}^{\infty} \mu (F_n)\leq \sum_{n=1}^{\infty} \mu (E_n).
    \end{equation*}
\end{proof}
}

\newcommand{\measureIsContinuousFromBelow}{
    \begin{proposition}{Continuidade por baixo}{measure_is_continuous_from_below}
        Seja $(X,\Sigma,\mu)$ um espaço de medida. Se $(E_n)_{n=1}^{\infty}\subset \Sigma$ é uma sequência crescente, então
        \begin{equation*}
            \mu\left(\bigcup_{n=1}^{\infty} E_n\right) = \lim \mu (E_n).
        \end{equation*}
    \end{proposition}
    \begin{proof}
    Vamos começar mostrando o primeiro item. Seja $(E_n)_{n=1}^{\infty}\subset \Sigma$ uma sequência crescente, isto é, $E_n\subset E_{n+1}$ para todo $n$ natural. Então, como na demonstração da Proposição \ref{prop:measure_is_monotonic}, $E_{n}=E_{n-1}\cup (E_{n}\setminus E_{n-1})$ para todo $n$ natural.
    
    Se definirmos $A_1=E_1$ e $A_n=E_{n}\setminus E_{n-1}$ para todo $n\geq 2$, então a sequência $(A_j)$ é disjunta. Para provar esta afirmação basta tomar $A_i$ e $A_j$ com $i\neq j$. Suponha, sem perda de generalidade, que $i < j$. Neste caso, $A_i=E_{i}\setminus E_{i-1}$ e $A_j=E_{j}\setminus E_{j-1}$ mas, por hipótese, $E_i\subseteq E_{j-1}$. Portanto, nenhum elemento de $A_i$ pode pertencer a $A_j$.

    Mostrado isso, podemos reescrever $E_n$ como a união de todos os $A_j$ até $n$.
    \begin{equation*}
        E_n=\bigcup_{j=1}^{n}A_j.
    \end{equation*}
    De fato, se $x\in E_n$ então $x\in E_{n-1} \cup A_n$. Por indução, $x\in\cup_{j=1}^{n}A_j$. Por outro lado, se $x\in \cup_{j=1}^{n}A_j$, então existe $j$ entre $1$ e $n$ tal que $x\in A_j=E_j\setminus E_{j-1}=E_j\cap E_{j-1}^{c}$. Logo, $x\in E_j\subseteq E_n$.

    Por fim, podemos concluir que
    \begin{equation*}
        \mu\left(\bigcup_{n=1}^{\infty} E_n\right)=\mu\left(\bigcup_{j=1}^{\infty} A_j\right)=\sum_{j=1}^{\infty} \mu(A_j)=\lim_{k\rightarrow \infty} \sum_{j=1}^{k} \mu(A_j).
    \end{equation*}
    Pela Proposição \ref{prop:measure_is_monotonic}, temos que
    \begin{equation*}
        \sum_{j=1}^{k} \mu(A_j) = \sum_{j=1}^{k} \left(\mu(E_j)-\mu(E_{j-1})\right) = \mu(E_k)\Longrightarrow \lim_{k\rightarrow \infty} \sum_{j=1}^{k} \mu(A_j)=\lim_{n\rightarrow \infty}\mu(E_n).
    \end{equation*}
    Note que, como a sequência $(E_n)$ é estritamente crescente, podemos nos certificar que cada $A_j$ é finito. Assim vale a \nameref{prop:measure_is_continuous_from_below}.
    
\end{proof}
}

\newcommand{\measureIsContinuousFromAbove}{
    \begin{proposition}{Continuidade por cima}{measure_is_continuous_from_above}
        Seja $(X,\Sigma,\mu)$ um espaço de medida. Se $(F_n)_{n=1}^{\infty}\subset \Sigma$ é uma sequência decrescente e a medida de $\mu (F_1)$ é finita, então
        \begin{equation*}
            \mu\left(\bigcap_{n=1}^{\infty} F_n\right) = \lim \mu (F_n).
        \end{equation*}
    \end{proposition}
    \begin{proof}
    Para provar, vamos usar o resultado da Proposição \ref{prop:measure_is_continuous_from_below}. A ideia é criar uma sequência crescente. Note bem, se a sequência é decrescente e o primeiro conjunto é finito, então todos os outros também são. Definimos, assim, os conjuntos $E_n=F_1\setminus F_n$ para formar a sequência crescente $(E_n)$.

    Por um lado, por \nameref{prop:measure_is_continuous_from_below} temos que
    \begin{equation*}
        \mu\left(\bigcup_{n=1}^{\infty} E_n\right) = \lim \mu (E_n) = \mu(F_1) - \lim \mu(F_n).
    \end{equation*}

    Por outro lado,
    \begin{equation*}
        \mu\left(\bigcup_{n=1}^{\infty} E_n\right) = \mu\left(F_1\setminus \bigcap_{n=1}^{\infty} F_n\right) = \mu(F_1) - \mu\left( \bigcap_{n=1}^{\infty} F_n\right).
    \end{equation*}
    Igualando as duas equações e cancelando o termo $\mu(F_1)$ temos que $\mu(\cap F_n)=\lim \mu(F_n)$.
    
\end{proof}
}



%~~~~~~~~~~~~~~~~~~~~~~~~~~~~~~~~~~~~~~~~~~~~~~~~~~~~
% first chapter corollary
%~~~~~~~~~~~~~~~~~~~~~~~~~~~~~~~~~~~~~~~~~~~~~~~~~~~~
\newcommand{\productSigmaAlgebraOfCountableGeneratedFamilies}{
    \begin{corollary}{\sigmaAlg produto de família enumerável geradas}{product_sigma_algebra_of_countable_generated_families}
        Seja $I$ um conjunto enumerável e $\{X_i\}_{i\in I}$ uma família de espaços mensuráveis tal que cada $\Sigma_i=\sigma [\mathcal{C}_i]$. Então $\bigotimes_{i\in I} \Sigma_i$ é a $\sigma$-álgebra gerada por 
    \begin{equation*}
        \bigotimes_{i\in I} \Sigma_i=\generateSigmaAlg{\left\{\prod_{i\in I} E_i; E_i\in \mathcal{C}_i\right\}}.
    \end{equation*}
    \end{corollary}
    \begin{proof}
    Este corolário é uma consequência direta das aplicações das Proposições \ref{prop:product_sigma_algebra_of_countable_family} e \ref{prop:product_sigma_algebra_of_families_generated_by_sets}.
\end{proof}
}

\newcommand{\productSigmaAlgebraOfSeparableMetricSpaces}{
    \begin{corollary}{\sigmaAlg produto de espaços métricos separáveis}{product_sigma_algebra_of_separable_metric_spaces}
        Sejam $\{X_i\}_{i=1}^{n}$ e $X$ como na Proposição \ref{prop:product_sigma_algebra_of_metric_spaces}. Se cada $X_i$ for separável, então $\bigotimes_{i=1}^{n} \mathcal{B}(X_i)= \mathcal{B}(X)$.
        \begin{equation*}
            \bigotimes_{i=1}^{n} \borel{X_i} = \borel{\prod_{i=1}^{n}X_i}.
        \end{equation*}
    \end{corollary}
    \begin{proof}
    Pela Proposição \ref{prop:product_sigma_algebra_of_metric_spaces}, temos a primeira inclusão. Queremos mostrar que $\borel{X}\subseteq \bigotimes_{i=1}^{n} \mathcal{B}(X_i)$.  
    
    Como cada $X_i$ é separável por hipótese, então existe $E_i\subset X_i$ denso e enumerável. É possível mostrar que existe uma base enumerável $B_i$ para cada $X_i$ (ver Proposição \ref{prop:metrizable_separable_space_is_second_countable}). Por outro lado, $\borel{X}$ é gerado por $\{\prod_{i=1}^{n} A_i; A_i\in B_i \} \subseteq \mathcal{C}$ (definido como na Proposição \ref{prop:product_sigma_algebra_of_metric_spaces}). Logo, pelo Lema \ref{lmm:sigma_algebra_generated_by_subset}, temos que $\borel{X}\subseteq \bigotimes_{i=1}^{n} \mathcal{B}(X_i)$.
\end{proof}
}

\newcommand{\productSigmaAlgebraOfRn}{
    \begin{corollary}{\sigmaAlg produto de $\R^n$}{product_sigma_algebra_of_Rn}
        \begin{equation*}
            \mathcal{B}(\R^n)=\bigotimes_{i=1}^{n} \mathcal{B}(\R).
        \end{equation*}
    \end{corollary}
    \begin{proof}
    Usando o Fato \ref{prop:rn_is_separable} de que $\R$ é separável, podemos aplicar diretamente o Corolário \ref{cor:product_sigma_algebra_of_countable_generated_families} para provar o resultado.
\end{proof}
}

\newcommand{\measureOfDiffenrece}{
    \begin{corollary}{Medida da Diferença}{measure_of_diffenrece}
        Nas condições da Proposição \ref{prop:measure_is_monotonic}, se $E,F\in \Sigma$ e $\mu(E)<\infty$, então
        \begin{equation*}
            \mu (F \setminus E) = \mu (F) - \mu (E).
        \end{equation*}
    \end{corollary}
    \begin{proof}
    Se $\mu(E)<\infty$, então podemos subtrair este valor dos dois lados sem correr o risco de chegar em uma diferença de infinitos. Portanto, $\mu (F \setminus E) = \mu (F) - \mu (E)$.
\end{proof}
}



%~~~~~~~~~~~~~~~~~~~~~~~~~~~~~~~~~~~~~~~~~~~~~~~~~~~~
% first chapter theorems
%~~~~~~~~~~~~~~~~~~~~~~~~~~~~~~~~~~~~~~~~~~~~~~~~~~~~
\newcommand{\deMorgan}{
    \begin{theorem}{Teorema de De Morgan}{de_morgan}
        Seja $X$ um conjunto e $\{O_i\}_{i\in \mathcal{I}}\subset \powerset{X}$. Então,
        \begin{enumerate}
            \item \begin{equation*}
                \complement \left(\bigcup_{i\in\mathcal{I}}O_i\right) = \bigcap_{i\in\mathcal{I}} \complement \left(O_i \right),
            \end{equation*}
            \item \begin{equation*}
                \complement \left(\bigcap_{i\in\mathcal{I}}O_i\right) = \bigcup_{i\in\mathcal{I}} \complement \left(O_i \right).
            \end{equation*}
        \end{enumerate}
    \end{theorem}
}

\newcommand{\measureCompletion}{
    \begin{theorem}{Completamento de Medida}{measure_completion}
        Seja $(X,\Sigma,\mu)$ um espaço de medida. Existe uma única $\overline{\mu}$, extensão de $\mu$ a uma $\sigma$-álgebra $\overline{\Sigma}$, tal que $(X,\overline{\Sigma},\overline{\mu})$ é um \nameref{def:complete_measure} .
    \end{theorem}
    \begin{proof}
    Denotemos por $\mathcal{N}$ a família de todos os conjuntos de medida nula.
    \begin{equation*}
        \mathcal{N}\coloneqq \{N\in\Sigma; \ \mu(N)=0\}.
    \end{equation*}
    
    Agora vamos expandir a nossa \texorpdfstring{$\sigma$}{sigma}-álgebra original para incluir todos os subconjuntos de conjuntos de medida nula.
    
    \begin{equation*}
        \overline{\Sigma}\coloneqq \left\{E\cup F; \ E\in \Sigma \text{ e } F\subset N\in\mathcal{N}\right\}.
    \end{equation*} 
    A ideia aqui é criar uma segunda função $\overline{\mu}:\overline{\Sigma}\rightarrow \left[0,+\infty\right]$ tal que $\overline{\mu}$ restrita a $\Sigma$ coincida com $\mu$, uma extensão da medida original. Como queremos que $\overline{\mu}$ seja uma medida, é preciso mostrar que $\overline{\Sigma}$ é uma \texorpdfstring{$\sigma$}{sigma}-álgebra (Passo XXX).

    Como argumentamos anteriormente, para $\overline{\mu}$ ser extensão $\mu$, ela deve coincidir com o valor assumido por $\mu$ para conjuntos em $\Sigma$. Definiremos, então
    \begin{equation*}
    \overline{\mu}(E\cup F) = \mu(E) \text{ para todo } E\cup F \in \overline{\Sigma}.
    \end{equation*}

    Para que isso funcione, é preciso mostrar que a função está bem definida (Passo YYY). Também precisamos mostrar que $\overline{\mu}$ satisfaz a definição de medida (Passo ZZZ), além de mostrar que ela é a única extensão existente (Passo AAA).

    Assim, mostramos que sempre podemos considerar um espaço de medida maior do que o original onde a nova medida será completa.
\end{proof}
}

\newcommand{\caratheodory}{
    \begin{theorem}{Teorema de Extensão de Carathéodory}{caratheodory}
        Sejam $\mathcal{H}$ um \texorpdfstring{$\sigma$}{sigma}-anel hereditário e $\mu^{*}$ uma medida exterior em $\mathcal{H}$. Então:
    \begin{enumerate}
        \item A coleção $\mathfrak{M}\subset \mathcal{H}$ dos conjuntos $\mu^{*}$-mensuráveis é um \texorpdfstring{$\sigma$}{sigma}-anel.\label{teo:caratheodory/item/sigma_anel}
        \item Dados $A\in\mathcal{H}$ e uma sequência $(E_n)_{n=1}^{\infty}\subset \mathfrak{M}$ de elementos dois a dois disjuntos, então
        \begin{equation*}\label{teo:caratheodory/item/aditiva_contavel}
            \mu^{*}\left(A\cap \bigcup_{n=1}^{\infty} E_n \right) = \sum_{n=1}^{\infty} \mu^{*}\left(A\cap E_n\right).
        \end{equation*}
        \item A restrição de $\mu^{*}$ a $\mathfrak{M}$ é uma medida \texorpdfstring{$\sigma$}{sigma}-aditiva.\label{teo:caratheodory/item/sigma_aditiva}
        \item Se $E\in\mathcal{H}$ é tal que $\mu^{*}(E)=0$ então $E\in\mathfrak{M}$.\label{teo:caratheodory/item/medida_mula}
    \end{enumerate}
    \end{theorem}
    \begin{proof}
    Para demonstrar o teorema, precisamos mostrar vários resultados. Aqui serão apresentadas as ideias gerais. Você poderá seguir até o Apêndice para conferir a demonstração passo a passo.
    
    Para mostrar o item \ref{teo:caratheodory/item/sigma_anel}, precisamos provar que $\mathfrak{M}$ é fechado para a união enumerável e para a diferença de conjuntos (Exercício \ref{exe:m_is_sigma_ring}).
    
    A fim de verificar o item \ref{teo:caratheodory/item/aditiva_contavel}, podemos usar indução para mostrar que, para algum $t\in\N$,
    
    \begin{equation*}
        \mu^{*}\left(A\cap \bigcup_{n=1}^{t} E_n \right) = \sum_{n=1}^{t} \mu^{*}\left(A\cap E_n\right).
    \end{equation*}
    Depois, podemos usar a monotonicidade em 
    \begin{equation*}
        A\cap\left(\bigcup_{n=1}^{\infty} E_n\right)^{\complement}\subset A\cap\left(\bigcup_{n=1}^{t} E_n\right)^{\complement}
    \end{equation*}
    para mostrar que 
    \begin{equation*}
        \mu^{*}(A)\geq\left(\sum_{n=1}^{t}\mu^{*}(A\cap E_n)\right) + \mu^{*}\left(A\cap\left(\bigcup_{n=1}^{\infty} E_n\right)^{\complement}\right).
    \end{equation*}
    Por fim, usamos a subaditividade e tomamos o limite $t\rightarrow \infty$ para concluir que
    \begin{equation*}
        \mu^{*}(A) =\left(\sum_{n=1}^{\infty}\mu^{*}(A\cap E_n)\right) + \mu^{*}\left(A\cap\left(\bigcup_{n=1}^{\infty} E_n\right)^{\complement}\right).
    \end{equation*}
    Chegamos no resultado desejado substituindo $A$ por $A\cap\left(\bigcup_{n=1}^{\infty} E_n\right)$.
    
    Os últimos dois itens são mais simples.

    Para mostrar o Item \ref{teo:caratheodory/item/sigma_aditiva}, tome $A=\bigcup_{k=1}^{\infty} E_k$ e aplique no Item \ref{teo:caratheodory/item/aditiva_contavel}. Temos, então

    \begin{equation*}
        \mu^{*}\left(A\cap \bigcup_{n=1}^{\infty} E_n \right) = \mu^{*}\left(\bigcup_{n=1}^{\infty} E_n \right) = \sum_{n=1}^{\infty} \mu^{*}\left(E_n\right).
    \end{equation*}

    Para mostrar o Item \ref{teo:caratheodory/item/medida_mula}, tome um conjunto $E\in\mathcal{H}$ de medida nula e um outro $A\in\mathcal{H}$ qualquer. Pela \hyperref[prop:measure_is_monotonic]{monotonicidade} da medida,

    \begin{equation*}
        \mu^*(A\cap E) + \mu^*(A\cap \complement E) \leq \mu^*(E) + \mu^*(A) = \mu^*(A).
    \end{equation*}

    Logo, $E$ é $\mu^*$-mensurável.
    \end{proof}
}





%%%%%%%%%%%%%%%%%%%%%%%%%%%%%%%%%%%%%%%%%%%%%%%%%%%%
%%%%%%%%%%%%%%%%%%%%%%%%%%%%%%%%%%%%%%%%%%%%%%%%%%%%
%
%            SECOND CHAPTER STATEMENTS
%
%%%%%%%%%%%%%%%%%%%%%%%%%%%%%%%%%%%%%%%%%%%%%%%%%%%%
%%%%%%%%%%%%%%%%%%%%%%%%%%%%%%%%%%%%%%%%%%%%%%%%%%%%

%~~~~~~~~~~~~~~~~~~~~~~~~~~~~~~~~~~~~~~~~~~~~~~~~~~~~
% second chapter lemmas
%~~~~~~~~~~~~~~~~~~~~~~~~~~~~~~~~~~~~~~~~~~~~~~~~~~~~

\newcommand{\integralOfSimpleFunctionSemiStandard}{
    \begin{lemma}{Integral de uma função simples na representação SP}{integral_of_simple_function_semi_standard}
        Seja $\varphi\in M^{+}(X,\Sigma)$ uma função simples. Se
        \begin{equation*}
            \varphi=\sum_{i=1}^{n} b_i\chi E_i
        \end{equation*}
        é uma representação semi-padrão de $\varphi$, então
        \begin{equation*}
            \int_X \varphi \ d\mu = \sum_{i=1}^{n} b_i \mu(E_i).
        \end{equation*}
    \end{lemma}
    \begin{proof}
    Suponha que $\varphi$ seja uma função simples com a representação 
    \[
    \varphi = \sum_{j=1}^{m} \alpha_j 1_{A_j}.
    \]
    A representação semi-padrão de $\varphi$ é
    \[
    \varphi = \sum_{k=1}^{n} \beta_k 1_{B_k}.
    \]
    Para cada $j = 1, \dots, m$, defina
    \[
    K_j = \{ k : 1 \leq k \leq n \ \text{e} \ \beta_k = \alpha_j \}.
    \]
    Daí,
    \[
    A_j = \bigcup_{k \in K_j} B_k.
    \]
    Assim, temos:
    \begin{align*}
        \int_X \varphi d\mu &= \sum_{j=1}^{m} \alpha_j \mu(A_j) \\
        &= \sum_{j=1}^{m} \alpha_j \sum_{k \in K_j} \mu(B_k) \\
        &= \sum_{j=1}^{m} \sum_{k \in K_j} \alpha_j \mu(B_k) \\
        &= \sum_{j=1}^{m} \sum_{k \in K_j} \beta_k \mu(B_k) \\
        &= \sum_{k=1}^{n} \beta_k \mu(B_k),
    \end{align*}
    pois
    \[
    \bigcup_{j=1}^{m} K_j = \{ 1, \dots, n \}.
    \]
\end{proof}

}

\newcommand{\integralInequalities}{
    \begin{lemma}{Desigualdade de integrais}{integral_inequalities}
        Sejam $f,g\in M^{+}(X,\Sigma)$ e $E,F\subset \Sigma$.
    \begin{enumerate}
        \item Se $f\leq g$, então $\int_{X} f \ d\mu \leq \int_{X} g \ d\mu$.
        \item Se $E\subseteq F$, então $\int_{E} f \ d\mu \leq \int_{F} f \ d\mu$.
    \end{enumerate}
    \end{lemma}
    \begin{proof}
    Para mostrar o primeiro item, perceba que toda função $\varphi \in M^{+}(X,\Sigma)$ simples tal que $\varphi \leq f$ também é, por hipótese, tal que $\varphi \leq g$. Ou seja,
    \begin{equation*}
        \left\{ \int_X \varphi \, d\mu; \ \varphi \in  M^{+}(X,\Sigma) \text{ e } 0 \leq \varphi \leq f \right\} \subseteq \left\{ \int_X \varphi \, d\mu; \ \varphi \in  M^{+}(X,\Sigma) \text{ e } 0 \leq \varphi \leq g \right\}.
    \end{equation*}
    Tomando o supremo dos dois lados, obtemos, pela Definição \ref{def:lebesgue_integral_non_negative_function},
    \begin{equation*}
        \int_{X} f \ d\mu \leq \int_{X} g \ d\mu.
    \end{equation*}
    Já para o segundo item basta notar que $f\chi_E\leq f\chi_F$ por hipótese. A demonstração segue da aplicação do primeiro item e da Definição \ref{def:lebesgue_integral_restricted}.
\end{proof}
}

\newcommand{\fatou}{
    \begin{lemma}{Lema de Fatou}{fatou}
        Seja $(f_n)\subset M^{+}(X,\Sigma)$. Então,
    \begin{equation}
        \int_{X} \left(\liminf_{n\rightarrow \infty} f_n\right) \ d\mu \leq \liminf_{n\rightarrow \infty} \left( 
 \int_{X} f_n \ d\mu\right).
    \end{equation}
    \end{lemma}
    \begin{proof}
    Defina $g_n=\inf_{k\geq n} f_k$. Pela definição de $g_n$, para todo $k\geq n$, $g_n\leq f_k$. Assim, pelo Lema \ref{lmm:integral_inequalities},
    \begin{equation*}
        \int_{X} g_n \ d\mu \leq \int_{X} f_k \ d\mu \quad \forall \ k\geq n.
    \end{equation*}
    Portanto, tomando o ínfimo dos dois lados,
    \begin{equation*}
        \int_{X} g_n \ d\mu \leq \inf_{k\geq n}\int_{X} f_k \ d\mu.
    \end{equation*}
    Tomemos o limite dos dois lados.
    \begin{equation}\label{eq/passo:fatou}
        \lim_{n\rightarrow \infty}\left(\int_{X} g_n \ d\mu\right) \leq \lim_{n\rightarrow \infty}\left(\inf_{k\geq n}\int_{X} f_k \ d\mu\right)=\liminf_{n\rightarrow\infty}\left(\int_{X} f_k \ d\mu\right) .
    \end{equation}
    Estamos bem próximo dos resultado final, basta observar que $(g_n)\subset M^{+}(X,\Sigma)$, pelo Lema \ref{prop:measurable_functions_sequences}, e que esta é uma sequência monótona crescente que converge para $\liminf f_n$. Portanto, podemos aplicar o Teorema da Convergência Monótona (\ref{thm:mct}) e concluir que
    \begin{equation*}
        \lim_{n\rightarrow\infty}\left(\int_{X} g_n \ d\mu\right) = \int_{X} \liminf_{n\rightarrow\infty} f_n \ d\mu.
    \end{equation*}
    Assim, juntando isso com \eqref{eq/passo:fatou}, temos que 
    \begin{equation*}
        \int_{X} \left(\liminf_{n\rightarrow \infty} f_n\right) \ d\mu \leq \liminf_{n\rightarrow \infty} \left( 
 \int_{X} f_n \ d\mu\right).
    \end{equation*}
\end{proof}
}


%~~~~~~~~~~~~~~~~~~~~~~~~~~~~~~~~~~~~~~~~~~~~~~~~~~~~
% second chapter propositions
%~~~~~~~~~~~~~~~~~~~~~~~~~~~~~~~~~~~~~~~~~~~~~~~~~~~~
\newcommand{\measurableFunctionsAndSigmaAlgebrasGeneratedBySet}{
    \begin{proposition}{Funções mensuráveis e \sigmaAlgs geradas por \texorpdfstring{$\mathcal{C}$}{C}}{measurable_functions_and_sigma_algebras_generated_by_set}
        Sejam $(X,\Sigma)$, $(Y,\generateSigmaAlg{\mathcal{C}}) $ \nameref{def:measurable_space} para algum $\mathcal{C}\subset \mathcal{P}(Y)$ e $f:(X,\Sigma)\rightarrow (Y,\generateSigmaAlg{\mathcal{C}})$. Então $f$ é mensurável se, e somente se, $f^{-1}(E)\in\Sigma$ para todo $E\in \mathcal{C}$.
    \end{proposition}
    \begin{proof}
    A ida é imediata, segue da Definição \ref{def:measurable_functions}. Para mostrar a volta, precisamos provar que $f^{-1}(E)\in\Sigma$ para cada $E\in \sigma(\mathcal{C})$. Definamos o conjunto abaixo.
    \begin{equation*}
        \mathcal{S}=\{E\in \sigma(\mathcal{C}); \ f^{-1}(E)\in \Sigma\}.
    \end{equation*}

    Note bem, temos, por hipótese, que $\mathcal{C}\subset \mathcal{S}$ por hipótese. Se este conjunto for uma $\sigma$-álgebra, então $\sigma(\mathcal{C}) \subset \mathcal{S}$, Logo, $\mathcal{S}=\mathcal{C}$ e $f$ é mensurável. Vamos provar, portanto, que $\mathcal{S}$ é uma $\sigma$-álgebra.

    Primeiramente, note que $\varnothing$ e $X$ pertencem a $\mathcal{S}$ pois $f^{-1}(\varnothing)=\varnothing$ e $f^{-1}(Y)=X$. Em segundo lugar, perceba que se $A\in \mathcal{S}$ então $f^{-1}(A)\in \Sigma$. Como $\Sigma$ é $\sigma$-álgebra, então $(f^{-1}(A))^{c}=f^{-1}(A^{c})\in \Sigma$. Assim, $A^{c}\in \mathcal{S}$. Por último, verifique que:
    \begin{equation*}
        \left(A_n\right)_{n=1}^\infty \subset \mathcal{S} \Rightarrow \left(f^{-1}(A_n)\right)_{n=1}^\infty \in \Sigma\Rightarrow  \bigcup_{n=1}^{\infty} f^{-1}(A_n)= f^{-1}\left(\bigcup_{n=1}^{\infty} A_n\right) \in \Sigma \Rightarrow \bigcup_{n=1}^{\infty} A_n\in S.
    \end{equation*}
    Mostrado que $S$ é uma $\sigma$-álgebra, concluímos o argumento de que $f$ é mensurável.
\end{proof}
}

\newcommand{\measurableFunctionsOperations}{
    \begin{proposition}{Operações com funções reais mensuráveis}{measurable_functions_operations}
        Sejam $f,g:X\rightarrow \R$ funções mensuráveis. Então,
    \begin{itemize}
        \item $f+g$,
        \item $\lambda f$ para todo $\lambda\in \R$,
        \item $f^2$,
        \item $|f|$,
        \item $\max\{f,g\}$,
        \item $\min\{f,g\}$,
        \item $fg$
    \end{itemize}
    são mensuráveis.
    \end{proposition}
    \begin{proof}
    Seja $\alpha\in\R$. Como $f$ e $g$ são funções mensuráveis, então os conjuntos
    \begin{eqnarray*}
        \{x\in X; \ f(x)> r\}\in \Sigma\\
        \{x\in X; \ g(x)> \alpha-r\}\in \Sigma
    \end{eqnarray*}
    para todo $r$ racional. Chamemos
    \begin{equation*}
        S_r=\{x\in X; \ f(x)> r\}\cap \{x\in X; \ g(x)> \alpha-r\}\in\Sigma.
    \end{equation*}
    Como, cada $S_r$ é mensurável, a sua união também é pois os racionais são enumeráveis. Assim, basta mostrar que
    \begin{equation*}
        \{x\in X; \ f(x)+g(x)> \alpha\}= \bigcup_{r\in \mathbb{Q}} S_r\in\Sigma.
    \end{equation*}
    Assim, pela propriedade, $f+g$ é mensurável.

    Para mostrar que $\lambda f$ é mensurável, é razoável desconsiderar o caso em que $\lambda =0$ pois a função identicamente nula é mensurável. Quando este não for o caso, basta mostrar que

    \begin{equation*}
        \{x\in X; \ \lambda f(x)> \alpha\}= \{x\in X; \ f(x)> \frac{\alpha}{\lambda}\}\in \Sigma
    \end{equation*}
        pois $f$ é mensurável por hipótese.

        Para mostrar que $f^2$ é mensurável, usamos uma estratégia similar. De fato, se $\alpha$ for negativo então a pré-imagem será $X\in\Sigma$. Caso contrário, então 
        \begin{equation*}
        \{x\in X; \ \lambda f^2(x)> \alpha\}= \{x\in X; \ f(x)> \sqrt{\alpha}\}\cup\{x\in X; \ f(x)< -\sqrt{\alpha}\} \in \Sigma
    \end{equation*}
    pois $f$ é mensurável. O caso é análogo para $|f|$.

    Para mostrar que $\max\{f,g\}$ e $\min\{f,g\}$ são mensuráveis, mostre que
    \begin{eqnarray*}
        \{x\in X; \ \max\{f,g\}> \alpha\}= \{x\in X; \ f(x)< \alpha\}\cap\{x\in X; \ g(x)<\alpha\} \in \Sigma\\
        \{x\in X; \ \min\{f,g\}> \alpha\}= \{x\in X; \ f(x)> \alpha\}\cap\{x\in X; \ g(x)>\alpha\} \in \Sigma\\
    \end{eqnarray*}

    Para mostrar que $fg$ é mensurável, escreva $fg=\frac{1}{4}((f+g)^2-(f-g)^2)$ e aplique as propriedades anteriores.
\end{proof}
}

\newcommand{\measurableFunctionsOperationsRExtend}{
    \begin{proposition}{Operações com funções \texorpdfstring{$\Rextend$}{R}-mensuráveis}{measurable_functions_operations_r_extend}
        Sejam $f,g:X\rightarrow \Rextend$ funções mensuráveis. Então,
    \begin{itemize}
        \item $\lambda f$ para todo $\lambda\in \R$,
        \item $f^2$,
        \item $|f|$,
        \item $\max\{f,g\}$,
        \item $\min\{f,g\}$,
        \item $fg$
    \end{itemize}
    são mensuráveis.
    \end{proposition}
    \begin{proof}
    Aqui os argumentos são os mesmos, mas a soma pode falhar quando as duas funções assumem valores infinitos, caindo no caso patológico $\infty - \infty$.
\end{proof}
}

\newcommand{\measurableFunctionsSequences}{
    \begin{proposition}{Sequências de funções mensuráveis}{measurable_functions_sequences}
        Seja $(f_n)\subset M(X,\Sigma)_{n=1}^{\infty}$ uma sequência de funções mensuráveis. Então,
    \begin{enumerate}
        \item $f(x)=\inf_{n\in\N} f_n(x)$,
        \item $F(x)=\sup_{n\in\N} f_n(x)$,
        \item $f^{*}(x)=\lim\inf_{n\in\N} f_n(x)$,
        \item $F^{*}(x)=\lim\sup_{n\in\N} f_n(x)$,
        \item $g(x)=\lim_{n\in\N} f_n(x)$ (dado que a sequência é pontualmente convergente),
    \end{enumerate}
    são mensuráveis
    \end{proposition}
    \begin{proof}
    \textbf{(1)} Para provar que $ f(x) = \inf_{n \in \mathbb{N}} f_n(x) $ é mensurável, observe que, para qualquer número real $\alpha$,
    \begin{equation*}
    \{ x \in X : f(x) \geq \alpha \} = \bigcap_{n=1}^{\infty} \{ x \in X : f_n(x) \geq \alpha \}.
    \end{equation*}
    Como cada $ f_n $ é mensurável, o conjunto $ \{ x \in X : f_n(x) \geq \alpha \} \in \Sigma $. Logo, $ f(x) $ é mensurável, já que a interseção enumerável de conjuntos mensuráveis pertence a $\Sigma$.

    \textbf{(2)} De forma análoga, para mostrar que $ F(x) = \sup_{n \in \mathbb{N}} f_n(x) $ é mensurável, observe que, para qualquer número real $\alpha$,
    \begin{equation*}
    \{ x \in X : F(x) \leq \alpha \} = \bigcap_{n=1}^{\infty} \{ x \in X : f_n(x) \leq \alpha \}.
    \end{equation*}
    Como cada $ f_n $ é mensurável, $ \{ x \in X : f_n(x) \leq \alpha \} \in \Sigma $, o que implica que $ F(x) $ é mensurável.

    \textbf{(3)} Para $ f^{*}(x) = \liminf_{n \to \infty} f_n(x) $, recorde que
    \begin{equation*}
    f^{*}(x) = \sup_{k \geq 1} \inf_{n \geq k} f_n(x).
    \end{equation*}
    Dado que $ f(x) = \inf_{n \in \mathbb{N}} f_n(x) $ e que o $ \sup $ de funções mensuráveis é mensurável, segue que $ f^{*}(x) $ é mensurável.

    \textbf{(4)} Para $ F^{*}(x) = \limsup_{n \to \infty} f_n(x) $, temos que
    \begin{equation*}
    F^{*}(x) = \inf_{k \geq 1} \sup_{n \geq k} f_n(x).
    \end{equation*}
    Usando o mesmo raciocínio do $ f^{*}(x) $, as operações de $ \sup $ e $ \inf $ pontuais preservam a mensurabilidade, então $ F^{*}(x) $ é mensurável.

    \textbf{(5)} Finalmente, se $ g(x) = \lim_{n \to \infty} f_n(x) $, dado que a sequência $ (f_n(x)) $ converge para todo $x \in X$, então $ g(x)=f^{*}(x)=F^{*}(x) $. Portanto, $ g(x) $ é mensurável.
\end{proof}

}

\newcommand{\integralOfSimpleFunctionIsLinear}{
    \begin{proposition}{Linearidade da integral de funções simples}{integral_of_simple_function_is_linear}
        Sejam $\varphi,\psi \in M^{+}(X,\Sigma)$ funções simples e $c\geq 0$. Então
    \begin{enumerate}
        \item $\int_{X} \ c\varphi \ d\mu = c\int_{X} \ \varphi \ d\mu$.
        \item $\int_{X} \ \varphi + \psi \ d\mu = \int_{X} \ \varphi \ d\mu + \int_{X} \ \psi \ d\mu$
    \end{enumerate}
    \end{proposition}
    \begin{proof}
    Para o primeiro item precisamos considerar dois casos. No primeiro caso, $c=0$ e a função simples que devemos integrar é a função simples constante. Isto valida a propriedade trivialmente pois teríamos que multiplicar a medida do conjunto inteiro por $0$. Mesmo se a medida for infinita, assumiremos ao longo destas notas que $0\cdot \infty=0$. Se $c > 0$ e $\varphi$ assume os valores $\{c_j\}_{j=1}^{n}$, então
    \begin{equation*}
        \int_{X}\ c\varphi \ d\mu = \sum_{j=i}^{n} c \, c_j \mu(f^{-1}(c_j)) = c\sum_{j=i}^{n}c_j \mu(f^{-1}(c_j))=c \int_{X}\varphi \ d\mu.
    \end{equation*}

    Vamos demonstrar o segundo item. Comece escrevendo $\varphi$ e $\psi$ na forma canônica.

    \begin{equation*}
        \varphi = \sum_{i=1}^{n} a_i \chi_{A_i} \quad \quad \text{e} \quad \quad 
        \psi = \sum_{j=1}^{m} b_j \chi_{B_j}.
    \end{equation*}

    Pelo Lema (Colocar no Apêndice), podemos $\chi_{A_i}=\sum_{j=1}^{m}\chi_{A_i\cap B_j}$. Portanto,

    \begin{equation*}
        \varphi = \sum_{i=1}^{n} a_i \chi_{A_i} = \sum_{i=1}^{n} a_i \sum_{j=1}^{m}\chi_{A_i\cap B_j}
    \end{equation*}

    Como $a_i$ é constante (com relação ao somatório de dentro), podemos passar para dentro. O processo é análogo para $\psi$, de onde vem que

    \begin{equation*}
        \varphi = \sum_{i=1}^{n}\sum_{j=1}^{m} a_i \chi_{A_i\cap B_j} \quad \quad \text{e} \quad \quad 
        \psi = \sum_{j=1}^{m}\sum_{i=1}^{n} b_j \chi_{A_i\cap B_j}.
    \end{equation*}

    Podemos trocar a ordem dos somatórios de $\psi$ uma vez que é uma soma finita. Com isso, temos

    \begin{equation*}
        \varphi + \psi = \sum_{i=1}^{n}\sum_{j=1}^{m} (a_i + b_j) \chi_{A_i\cap B_j}.
    \end{equation*}

    Note que esta é uma representação semi-padrão da função simples $\varphi+\psi$. Logo, pelo Lema \ref{lmm:integral_of_simple_function_semi_standard},
    \begin{eqnarray*}
        \int_X \varphi + \psi \ d\mu &=& \sum_{i=1}^{n}\sum_{j=1}^{m} (a_i + b_j) \mu(A_i\cap B_j)\\
        &=&\sum_{i=1}^{n}\sum_{j=1}^{m} a_i \mu(A_i\cap B_j) + \sum_{j=1}^{m}\sum_{i=1}^{n} b_j \mu(A_i\cap B_j)\\
        &=& \int_X \varphi d\mu + \int_X \psi d\mu
    \end{eqnarray*}
    
\end{proof}
}

\newcommand{\functionIsIntegrableIffAbsoluteValueIs}{
    \begin{proposition}{Condição para integrabilidade}{function_is_integrable_iff_absolute_value_is}
        Uma função $f \in L$ se, e somente se, $|f| \in L$. Neste caso,
    \begin{equation*}
        \left|\int f \, d\mu \right| \leq \int |f| \, d\mu.
    \end{equation*}
    \end{proposition}
    \begin{proof}
    Comecemos supondo que $f$ é integrável. Sabemos que $\abs{f}=f^{+}+f^{-}$. Neste caso, a parte positiva de $\abs{f}$, denotada por $\abs{f}^{+}$ é a própria soma de funções mensuráveis $f^{+}+f^{-}$. Por outro lado, a parte negativa, $\abs{f}^{-}=0$, pois a função é não negativa. Pelo Lema \ref{lmm:integral_is_zero_iff_function_is_zero_almost_everywhere}, temos que 
    \begin{equation*}
        \int \abs{f}^{-} d\mu = 0 < +\infty.
    \end{equation*}
    Ademais, pela Proposição \ref{prop:integral_of_non_negative_function_is_linear},
    \begin{equation*}
        \int \abs{f}^{+} d\mu = \int f^{+}+f^{-} d\mu = \int f^{+} d\mu + \int f^{-} d\mu.
    \end{equation*}
    Como, pela Definição \ref{def:lebesgue_integral},
    \begin{equation*}
        \int f^{+} d\mu <+\infty \quad \text{e}\quad  \int f^{-} d\mu < +\infty, \text{ então} \int \abs{f}^{+} d\mu < +\infty.
    \end{equation*}
    concluimos que $\abs{f}$ é integrável e
    \begin{equation*}
        \int \abs{f} d\mu = \int \abs{f}^{+} d\mu + \int \abs{f}^{-} d\mu = \int f^{+} d\mu + \int f^{-} d\mu.
    \end{equation*}
    Reciprocamente, se $\abs{f}$ é integrável, então $f^{+}$ e $f^{-}$ são integráveis. Portanto, $f$ é integrável.

    Mostrado isso, podemos concluir, pela desigualdade triangular, que
    \begin{equation*}
        \abs{\int f d\mu} = \abs{\int f^{+} - f^{-} d\mu} \leq \abs{\int f^{+} d\mu} + \abs{\int f^{-} d\mu} = \int f^{+} d\mu + \int f^{-} d\mu = \int \abs{f} d\mu.
    \end{equation*}
\end{proof}
}



%~~~~~~~~~~~~~~~~~~~~~~~~~~~~~~~~~~~~~~~~~~~~~~~~~~~~
% second chapter corollary
%~~~~~~~~~~~~~~~~~~~~~~~~~~~~~~~~~~~~~~~~~~~~~~~~~~~~

\newcommand{\measurableFunctionsInR}{
    \begin{corollary}{Mensurabilidade de funções reais}{measurable_functions_in_r}
        Sejam $(X,\Sigma)$, $(Y,\generateSigmaAlg{\mathcal{C}}) $ \nameref{def:measurable_space} para algum $\mathcal{C}\subset \mathcal{P}(Y)$ e $f:(X,\Sigma)\rightarrow (Y,\generateSigmaAlg{\mathcal{C}})$. Então $f$ é mensurável se, e somente se, $f^{-1}(E)\in\Sigma$ para todo $E\in \mathcal{C}$.
    \end{corollary}
    \begin{proof}
    Seja $f:(X,\Sigma)\rightarrow (\R,\mathcal{B})$ uma função mensurável e $\alpha$ um número real. Então $f^{-1}((\alpha, \infty))\in \Sigma$, isto é, $A_{\alpha}\in \Sigma$. Pela arbitrariedade da escolha de $\alpha$, vale a ida da proposição.

    Reciprocamente, seja $f:(X,\Sigma)\rightarrow (\R,\mathcal{B})$ uma função tal que, para todo $\alpha$ real, $A_{\alpha} \in \Sigma$. Pelo Exercício \ref{prop:basis_of_r} $\mathcal{B}$ pode ser gerado pelos intervalos na forma $(\alpha,\infty)$. Assim, pela Proposição \ref{prop:measurable_functions_and_sigma_algebras_generated_by_set}, só precisamos verificar os intervalos da forma $(\alpha,\infty)$, que pertencem a $\Sigma$  por hipótese. Logo, $f$ é mensurável.
\end{proof}
}

\newcommand{\measurableFunctionsInRExtend}{
    \begin{corollary}{Mensurabilidade de funções reais}{measurable_functions_in_r_extend}
        Seja $(X,\Sigma)$ um \nameref{def:measurable_space} e $f:(X,\Sigma)\rightarrow (\Rextend,\mathcal{B}(\Rextend))$. Sejam,
    \begin{equation*}
        A\coloneq\{x\in X; \ f(x)=+\infty\} \quad \text{e} \quad B\coloneq\{x\in X; \ f(x)=\infty\}
    \end{equation*}
    e a função $f_0:(X,\Sigma)\rightarrow (\R,\mathcal{B})$ definida da seguinte forma
    \begin{equation*}
        f_0(x) = \begin{cases}
            f(x), \text{ se } x \in (A\cup B)^{c} \\
            0, \text{ se } x \in A\cup B.
        \end{cases}
    \end{equation*}
    Então $f$ é mensurável se, e somente se, $A,B\in\Sigma$ e a $f_0$ é $\R$-mensurável. 
    \end{corollary}
    \begin{proof}
    Se $f$ for mensurável, então todos os $A,B\in\Sigma$ trivialmente. Além disso, $f_0$ deve ser $\R$-mensurável pois, caso não fosse, existiria um aberto de $\R\subset \Rextend$ tal que a pré-imagem deste aberto não é $\Rextend$-mensurável, o que é uma contradição.
    
    Reciprocamente, pela Definição \ref{def:basis_of_r_extend}, sabemos que $\mathcal{B}(\Rextend)$ pode ser gerado pelos intervalos na forma $(a,\infty)$ com os conjuntos $\{-\infty\}$, $\{+\infty\}$. Assim, pela Proposição \ref{prop:measurable_functions_and_sigma_algebras_generated_by_set}, a hipótese implica a mensurabilidade de $f$. 
\end{proof}
}

\newcommand{\dominatedFunctionIsIntegrable}{
    \begin{corollary}{Funções dominadas por função integrável é integrável}{dominated_function_is_integrable}
        Se $f\in M$, $g \in L$ e $|f| \leq |g|$, então $f\in L$ e 
    \begin{equation*}
        \int |f| \, d\mu \leq \int |g| \, d\mu.
    \end{equation*}
    \end{corollary}
    \begin{proof}
    Como $\abs{f}\leq \abs{g}$, pelo Lema \ref{lmm:integral_inequalities}, temos que $\int \abs{f} d\mu \leq \int \abs{g} d\mu$ e, portanto, $\abs{f}$ é integrável. Pela Proposição \ref{prop:function_is_integrable_iff_absolute_value_is}, temos, então, que $f$ é integrável.
\end{proof}
}



%~~~~~~~~~~~~~~~~~~~~~~~~~~~~~~~~~~~~~~~~~~~~~~~~~~~~
% second chapter theorems
%~~~~~~~~~~~~~~~~~~~~~~~~~~~~~~~~~~~~~~~~~~~~~~~~~~~~

\newcommand{\MCT}{
    \begin{theorem}{Teorema da Convergência Monótona}{mct}
        Seja $(f_n)\subset M^{+}(X,\Sigma)$ uma sequência monótona crescente tal que $f_n \xrightarrow{p} f$. Então $f\in M^{+}(X,\Sigma)$ e 
    \begin{equation}
        \int_{X} \left(\lim_{n\rightarrow \infty} f_n\right) \ d\mu = \lim_{n\rightarrow \infty} \left( 
 \int_{X} f_n \ d\mu\right)
    \end{equation}
    \end{theorem}
    \begin{proof}
    A mensurabilidade de $f$ segue diretamente da Proposição \ref{prop:measurable_functions_sequences}, mas é um passo importante para que faça sentido integrar essa função segundo a Definição \ref{def:lebesgue_integral_non_negative_function}. Provaremos a segunda parte do teorema mostrando que
    \begin{equation*}
        \int_{X} \left(\lim_{n\rightarrow \infty} f_n\right) \ d\mu \geq \lim_{n\rightarrow \infty} \left( 
 \int_{X} f_n \ d\mu\right) \quad \quad \text{e} \quad \quad \int_{X} \left(\lim_{n\rightarrow \infty} f_n\right) \ d\mu \leq \lim_{n\rightarrow \infty} \left( 
 \int_{X} f_n \ d\mu\right).
    \end{equation*}
    Para mostrar que a primeira desigualdade, note que, como $(f_n)$ é uma sequência monótona crescente, então $f_n\leq f_{n+k}$ para todos $n,k$ natural. Assim, se $k\rightarrow \infty$ então $f_{n+k} \rightarrow f$ para todo $n$. Logo,
    \begin{equation*}
        f_n \leq \lim_{n\rightarrow \infty}f_n \quad \forall \ n\in\N.
    \end{equation*}
    Pelo Lema \ref{lmm:integral_inequalities}, o fato acima nos dá que
    \begin{equation*}
        \int_{X} f_n \ d\mu \leq \int_{X} \left(\lim_{n\rightarrow \infty} f_n\right) \ d\mu \quad \forall \ n\in\N.
    \end{equation*}
    Note que o lado direito da desigualdade é constante. De fato, ele será igual a $\int_X f d\mu$. Assim, podemos tomar o limite dos dois lados para concluir que
    \begin{equation*}
        \lim_{n\rightarrow \infty}\left(\int_{X} f_n \ d\mu\right) \leq \int_{X} \left(\lim_{n\rightarrow \infty} f_n\right) \ d\mu \Longleftrightarrow \int_{X} \left(\lim_{n\rightarrow \infty} f_n\right) \ d\mu \geq \lim_{n\rightarrow \infty} \left( 
 \int_{X} f_n \ d\mu\right).
    \end{equation*}

    Aqui vale notar que o Lema \ref{lmm:integral_inequalities} garante que $\left(\int_X f_n d\mu\right)$ também é uma sequência crescente, o que justifica a existência do limite em $\Rextend$. Resta mostrarmos a outra desigualdade.
    
    A ideia é mostrar que, para toda função simples $\varphi \leq f$, teremos $\int_X \varphi \leq \lim (\int_X f_n d\mu)$. Depois poderemos tomar o supremo dos dois lados da desigualdade para chegar no resultado desejado.

    Começaremos tomando uma função $\varphi\in M^{+}(X,\Sigma)$ simples tal que $\varphi\leq f$ e $\alpha\in\R$ tal que $\alpha\in (0,1)$. Agora, defina o conjunto
    \begin{equation}
        A_n\coloneqq \{x\in X; \ \alpha\varphi(x)\leq f_n(x)\}.
    \end{equation}
    Faremos três afirmações sobre esse conjunto.
    \begin{enumerate}
        \item $A_n\in\Sigma$ para todo $n$ natural.\label{teo/lema:A_n_mensuravel}
        \item $(A_n)_{n=1}^{\infty}\subset \Sigma$ é uma sequência monótona crescente.\label{teo/lema:monotona_crescente}
        \item $\cup_{n=1}^{\infty}A_n=X$.\label{teo/lema:limite_conjunto}
    \end{enumerate}
    
    Vamos provar o Item \ref{teo/lema:A_n_mensuravel}. Aplicando as operações estabelecidas na Proposição \ref{prop:measurable_functions_operations}, temos que $f_n-\alpha\varphi$ é uma função mensurável para todo $n$. Além disso, podemos escrevermos cada $A_n$ como a pré-imagem da função $f_n-\alpha\varphi$.
    \begin{equation*}
        A_n=(f_n-\alpha\varphi)^{-1}((0,\infty)) \ \forall \ n\in\N.
    \end{equation*}
    Logo, pela Definição \ref{def:measurable_functions}, $A_n\in \Sigma$ para todo $n$.
    
    Vamos provar o Item \ref{teo/lema:monotona_crescente}. Para tanto precisamos mostrar que $A_n\subset A_{n+1}$ para todo $n$. Esta afirmação segue imediatamente da monotonicidade de $(f_n)$ uma vez que se $x\in A_n$ então $\alpha\varphi(x)\leq f_n(x)\leq f_{n+1}$.

    Vamos provar o Item \ref{teo/lema:limite_conjunto}. A primeira continência é imediata uma vez que cada $A_n$ é um subconjunto de $X$. No entanto, ainda é preciso mostrar que $X\subseteq \cup_{n=1}^{\infty} A_n$. Tome $x\in X$. Sabemos que $\varphi (x) \leq f(x)$ por hipótese. Como $\alpha\in (0,1)$, então $\alpha\varphi(x)\leq f(x)$. Queremos mostrar que existe um $N$ natural tal que $\alpha \varphi(x) \leq f_N(x)$. Faremos isso por contradição. Suponha que, para todo $n\in\N$, $f_n(x)<\alpha\varphi(x)$. Tomando o limite dos dois lados teríamos que $\lim f_n(x) = f(x) < \alpha \varphi(x)$, o que contradiz a nossa hipótese. Logo, existe $N\in\N$ tal que $\alpha \varphi(x) \leq f_N(x)$. Assim sendo, $x\in A_N \subset \cup_{n}^{\infty} A_n$. Por fim, concluímos que $\cup_{n=1}^{\infty}A_n=X$.

    Provados estes detalhes técnicos, vamos usar o Lema \ref{lmm:integral_inequalities} mais uma vez. Note que $A_n\subset X$ para todo $n$. Portanto, pelo segundo item do lema, temos que $\int_{A_n} f_n d\mu \leq \int_{X} f_n d\mu$ para todo $n$. Agora, pela definição de $A_n$, temos que $\alpha\varphi\chi_{A_n} \leq f_n\chi_{A_n}$. Logo, pelo primeiro item do lema temos que $\int_{A_n} \alpha\varphi d\mu \leq \int_{A_n} f_n d\mu$. Juntando essas duas desigualdades, obtemos
    \begin{equation*}
        \alpha \int_{A_n} \varphi \ d\mu \leq \int_{X} f_n \ d\mu \Rightarrow \alpha \lim_{n\rightarrow \infty} \int_{A_n} \varphi \ d\mu \leq \lim_{n\rightarrow \infty}\left(\int_{X} f_n \ d\mu\right).
    \end{equation*}
    Estamos quase chegando no resultado desejado, resta mostrar que $\lim\int_{A_n} \varphi \ d\mu = \int_{X} \varphi \ d\mu$. Esta demonstração pode ser encontrada em (Fazer).

    Por fim, podemos tomar o limite de $\alpha$ indo para $1$ dos dois lados e depois tomar o supremo de ambos os lados.
    \begin{equation*}
        \alpha \int_{X} \varphi \ d\mu \leq \lim_{n\rightarrow \infty}\left(\int_{X} f_n \ d\mu\right)\Rightarrow \lim_{\alpha\rightarrow 1} \alpha \int_{X} \varphi \ d\mu = \int_{X} \varphi \ d\mu \leq \lim_{n\rightarrow \infty}\left(\int_{X} f_n \ d\mu\right).
    \end{equation*}
    \begin{equation*}
        \therefore \sup\left(\int_{X} \varphi \ d\mu \right) = \int_{X} f \ d\mu \leq \lim_{n\rightarrow \infty}\left(\int_{X} f_n \ d\mu\right).
    \end{equation*}
    Assim, chegamos na desigualdade que buscávamos:
    \begin{equation*}
        \int_{X} \left(\lim_{n\rightarrow \infty} f_n\right) \ d\mu \leq \lim_{n\rightarrow \infty} \left( 
 \int_{X} f_n \ d\mu\right).
    \end{equation*}
\end{proof}
}

\newcommand{\DCT}{
    \begin{theorem}{Teorema da Convergência Dominada}{dct}
        Sejam $(f_n)\subset L$ e $f\in M(X,\Sigma)$ tais que
    \begin{equation*}
        \lim_{n\rightarrow \infty} f_n (x) = f(x) \ (\mu\text{-qtp}).
    \end{equation*}
    Se existe $g\in L$ tal que $|f_n| \leq g$ para todo $n$, então $f\in L$ e
    \begin{equation}
        \int_X f d\mu = \lim_{n\rightarrow \infty} \int_X f_n d\mu.
    \end{equation}
    \end{theorem}
    \begin{proof}
    Como $\lim_{n \to \infty} f_n(x) = f(x)$ e cada $f_n$ é mensurável, segue que $f$ é mensurável. Como $|f_n| \leq g$ para todo $n$, temos que $|f| \leq g$. Mas, por hipótese, $g$ é integrável e, consequentemente, $|g|$ é integrável. Portanto, $|f|$ é integrável e, finalmente, $f$ é integrável. Temos ainda 
    \[
    -|g| \leq f_n \leq |g|
    \]
    e daí segue que 
    \[
    |g| + f_n \geq 0 \ \text{para todo } n.
    \]
    
    Portanto,
    \begin{align*}
    \int_X |g| d\mu + \int_X f_n d\mu &= \int_X (|g| + f_n) d\mu \\
    &= \int_X \lim_{n \to \infty} (|g| + f_n) d\mu
    \end{align*}
    Pelo Lema de Fatou,
    \[
    \leq \liminf_{n \to \infty} \int_X (|g| + f_n) d\mu
    \]
    \begin{align*}
    &= \liminf_{n \to \infty} \left( \int_X |g| d\mu + \int_X f_n d\mu \right) \\
    &= \int_X |g| d\mu + \liminf_{n \to \infty} \int_X f_n d\mu.
    \end{align*}
    
    Logo,
    \[
    \int_X f d\mu \leq \liminf_{n \to \infty} \int_X f_n d\mu.
    \]
    
    Por outro lado,
    \[
    |g| - f_n \geq 0 \ \text{para todo } n.
    \]
    
    Portanto,
    \begin{align*}
    \int_X |g| d\mu - \int_X f_n d\mu &= \int_X (|g| - f_n) d\mu \\
    &= \int_X \lim_{n \to \infty} (|g| - f_n) d\mu
    \end{align*}
    Pelo Lema de Fatou,
    \[
    \leq \liminf_{n \to \infty} \int_X (|g| - f_n) d\mu
    \]
    \begin{align*}
    &= \liminf_{n \to \infty} \left( \int_X |g| d\mu - \int_X f_n d\mu \right) \\
    &= \int_X |g| d\mu - \limsup_{n \to \infty} \int_X f_n d\mu.
    \end{align*}
    
    Consequentemente,
    \[
    \int_X f d\mu \geq \limsup_{n \to \infty} \int_X f_n d\mu.
    \]
    
    Finalmente, temos
    \[
    \limsup_{n \to \infty} \int_X f_n d\mu \leq \int_X f d\mu \leq \liminf_{n \to \infty} \int_X f_n d\mu,
    \]
    e, portanto,
    \[
    \int_X f d\mu = \lim_{n \to \infty} \int_X f_n d\mu.
    \]
\end{proof}

}