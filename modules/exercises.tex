\newcommand{\extendedSigmaAlgebra}{
    \begin{exercise}{\texorpdfstring{$\overline{\Sigma}$}{Sigma} é \sigmaAlg}{extended_sigma_algebra}
        Nas condições do Teorema \ref{thm:measure_completion}, mostre que
        \begin{equation*}
            \overline{\Sigma}\coloneqq \left\{E\cup F; \ E\in \Sigma \text{ e } F\subset N\in\mathcal{N}\right\}
        \end{equation*}
        é uma \nameref{def:sigma_algebra}, sabendo que
        \begin{equation*}
            \mathcal{N}\coloneqq \{N\in\Sigma; \ \mu(N)=0\}.
        \end{equation*}

    \end{exercise}
    \begin{proof}
    Vamos mostrar que $\overline{\Sigma}$ é fechado por união enumerável. Para tanto, provaremos que tanto $\Sigma$ quanto $\mathcal{N}$ são fechados por união enumerável. $\Sigma$ o é por definição, uma vez que é \sigmaAlg. Agora, vamos mostrar que $\mathcal{N}$ também é. 

    Seja $(N_i)_{i=1}^{\infty}\subseteq \mathcal{N}$. Então, como \hyperref[prop:measure_is_subadditive]{a medida é subaditiva},

    \begin{equation*}
        \mu\left(\bigcup_{i=1}^{\infty} N_i\right) \leq \sum_{i=1}^{\infty} \mu(N_i) = 0.
    \end{equation*}

    Portanto, $\mathcal{N}$ é fechado por união enumerável. Assim, se tomarmos qualquer $(E_i\cup F_i)_{i=1}^{\infty}\subseteq \overline{\Sigma}$, onde $F_i\subset N_i$ com $N_i\in \mathcal{N}$ para todo $i\in\N$, temos

    \begin{equation*}
        \bigcup_{i=1}^{\infty} \left(E_i\cup F_i\right) = \left(\bigcup_{i=1}^{\infty} E_i\right) \cup \left(\bigcup_{i=1}^{\infty} F_i\right).
    \end{equation*}

    Note que

    \begin{eqnarray*}
        \bigcup_{i=1}^{\infty} E_i &\in& \Sigma\\
        \bigcup_{i=1}^{\infty} F_i&\subseteq& \bigcup_{i=1}^{\infty} N_i \in \mathcal{N}.
    \end{eqnarray*}

    Logo, $\overline{\Sigma}$ é fechado por união enumerável.

    Ainda precisamos mostrar que $\overline{\Sigma}$ é fechado pelo complemento. Fixe um $(E\cup F) = A \in \overline{\Sigma}$. Nessas condições, $E\in\Sigma$ e $F\subseteq N\in \mathcal{N}$. Vamos assumir que $E$ e $N$ são disjuntos. Se este não for o caso, basta reescrever $A$ substituindo $F$ por $F'=F\setminus E$ e $N$ por $N'=N\setminus E$. Note que $F'\subseteq N'\in \Sigma$ e 
    
    \begin{equation*}
        A = E \cup F = E \cup ((F\cap E) \cup (F\setminus E)) = E \cup (F\setminus E)=E\cup F'.
    \end{equation*}

    Agora que sabemos que podemos tomar $(E\cup F)$ com $E$ e $N$ disjuntos, façamos

    \begin{equation*}
        E\cup F = (E\cup N) \cap (\complement(N) \cup F).
    \end{equation*}

    Vamos provar esta igualdade de conjuntos.Vale resstaltar que esta relação só vale quando $E$ e $N$ são disjuntos. De fato, se $x\in E\cup F$, então $x\in E$ ou $x\in F$. Por consequência, $x\in (E\cup N)$ e $x\in (F\cup \complement(N))$. Assim, $x\in (E\cup N) \cap (\complement(N) \cup F)$. Portanto,

    \begin{equation}\label{pf:extended_sigma_algebra:eq:1}
        E\cup F \subseteq (E\cup N) \cap (\complement(N) \cup F).
    \end{equation}

    Para mostrar a igualdade, tomamos agora um $x \in (E\cup N) \cap (\complement(N) \cup F)$, isto é, $x \in (E\cup N)$ e $x \in (\complement(N) \cup F)$. Logo, pela primeira relação, $x\in E$ ou $x\in N$ e, pela segunda, $x\in F$ ou $x\in \complement(N)$. Note que $x$ só pode pertencer a $N$ ou $\complement(N)$ pois eles são disjuntos. Da mesma forma, $x$ só pode pertencer a $E$ ou a $N$ pois eles são disjuntos por hipótese. Se $x\in N$, então $x\not\in \complement(N)$. Como $x\in(\complement(N) \cup F)$, temos que $x\in F$. Por outro lado, se $x\in\complement(N)$, então $x\not\in N$. Como $x\in(E\cup N)$, temos que $x\in E$. Portanto,

    \begin{equation}\label{pf:extended_sigma_algebra:eq:2}
        E\cup F \supseteq (E\cup N) \cap (\complement(N) \cup F).
    \end{equation}

    Juntando \ref{pf:extended_sigma_algebra:eq:1} e \ref{pf:extended_sigma_algebra:eq:2}, temos que

    \begin{equation*}
        E\cup F = (E\cup N) \cap (\complement(N) \cup F).
    \end{equation*}

    Mostrada esta igualdade, retomemos o nosso objetivo de mostrar que $\overline{\Sigma}$ é fechado pelo complemento. Com o que fizemos anteriomente, e aplicando o \nameref{thm:de_morgan}, podemos ver que, para qualquer $E\cup F\in\overline{\Sigma}$, é tal que,

    \begin{eqnarray*}
        \complement(E\cup F) 
        &=& \complement\left((E\cup N) \cap (\complement(N) \cup F)\right)\\
        &=& \complement(E\cup N) \cup \complement(\complement(N) \cup F)\\
        &=& \complement(E\cup N) \cup N \cap \complement(F)\\
        &=& \complement(E\cup N) \cup (N \setminus F).\\
    \end{eqnarray*}

    Note que, como $\Sigma$ é fechado para as operações conjuntistas, $\complement(E\cup N)\in\Sigma$ e $(N \setminus F)\subset N$. Logo, $\complement(E\cup F) \in\overline{\Sigma}$.
    
    Concluimos, por fim, que $\overline{\Sigma}$ é uma \nameref{def:sigma_algebra}.
\end{proof}
}

\newcommand{\muIsWellDefined}{
    \begin{exercise}{\texorpdfstring{$\overline{\mu}$}{mu} está bem definida}{mu_is_well_defined}
        Nas condições do Teorema \ref{thm:measure_completion}, mostre que
        \begin{equation*}
            \overline{\mu}(E\cup F)\coloneqq \mu(E) \text{ para todo } E\cup F \in \overline{\Sigma}
        \end{equation*}
        está bem definida.

    \end{exercise}
    \begin{proof}
    Sejam $N_0,M_0\in\mathcal{N}$. Suponha que $E_1 \cup F_1 = E_2 \cup F_2$, com $E_1, E_2 \in \Sigma$ e $F_1\subset N_0$, $F_2\subset M_0$. Mostraremos que $\overline{\mu}(E_1 \cup F_1) = \overline{\mu}(E_2 \cup F_2)$, ou seja, $\mu(E_1) = \mu(E_2)$.

    Por construção, temos
    \begin{equation*}
        E_1 \subseteq E_1 \cup F_1 = E_2 \cup F_2 \subseteq E_2 \cup M_0.
    \end{equation*}

    Assim, como a medida é \hyperref[prop:measure_is_monotonic]{monotônica} e \hyperref[prop:measure_is_subadditive]{subaditiva},
    \begin{equation*}
        \mu(E_1) \leq \mu(E_2 \cup M_0) \leq \mu(E_2) + \mu(M_0) = \mu(E_2).
    \end{equation*}

    Analogamente, temos 
    \begin{equation*}
        E_2 \subseteq E_2 \cup F_2 = E_1 \cup F_1 \subseteq E_1 \cup N_0,
    \end{equation*}
    
    o que implica que
    \begin{equation*}
        \mu(E_2) \leq \mu(E_1 \cup N_0) \leq \mu(E_1) + \mu(N_0) = \mu(E_1).
    \end{equation*}

    Portanto, $\mu(E_1) = \mu(E_2)$, e assim $\overline{\mu}$ está bem definida.
\end{proof}

}

\newcommand{\muIsMeasure}{
    \begin{exercise}{\texorpdfstring{$\overline{\mu}$}{mu} é medida}{mu_is_measure}
        Nas condições do Teorema \ref{thm:measure_completion}, mostre que $\overline{\mu}$ é uma medida.
    \end{exercise}
    \begin{proof}
    Precisamos mostrar que $\overline{\mu}$ satisfaz a definição de medida. Para tanto, basta verificar que $\overline{\mu}(\varnothing)=0$ e que $\overline{\mu}$ é \nameref{def:countably_additive_function}. Note que a função já é positiva por construção, pois $\overline{\mu}(E\cup F)=\mu(E)\geq 0$ para todo $E\cup F\in\overline{\Sigma}$

    Note que $\varnothing\in\Sigma$ e $\varnothing\subset\varnothing\in\mathcal{N}$. Logo, $\varnothing=\varnothing\cup\varnothing\in\overline{\Sigma}$. Portanto, $\overline{\mu}(\varnothing)=\mu(\varnothing)=0$. Agora, só nos resta provar a segunda propriedade.

    Vamos mostrar que $\overline{\mu}$ é \nameref{def:countably_additive_function}. Tome uma sequência disjunta $(E_n\cup F_n)_{n=1}^{\infty}\subseteq \overline{\Sigma}$. Também vamos impor a condição de $E_n$ e $N_n$ serem disjuntos, como no Exercício \ref{exe:extended_sigma_algebra}. Assim,

    \begin{equation*}
        \bigcup_{n=1}^{\infty} \left(E_n\cup F_n\right) = \left(\bigcup_{n=1}^{\infty}E_n\right) \cup \left(\bigcup_{n=1}^{\infty}F_n\right).
    \end{equation*}

    Aplicando este resultado a $\overline{\mu}$, temos,

    \begin{eqnarray*}
        \overline{\mu}\left(\bigcup_{n=1}^{\infty} \left(E_n\cup F_n\right)\right)
        &=& \overline{\mu}\left(\left(\bigcup_{n=1}^{\infty}E_n\right) \cup \left(\bigcup_{n=1}^{\infty}F_n\right)\right)\\
        &=& \mu\left(\bigcup_{n=1}^{\infty}E_n\right) (\text{pela definição de } \overline{\mu})\\
        &=& \sum_{n=1}^{\infty} \mu(E_n)\\
        &=& \sum_{n=1}^{\infty} \overline{\mu}(E_n\cup F_n).
    \end{eqnarray*}

    Portanto, $\overline{\mu}$ é \nameref{def:countably_additive_function}. Em conclusão, $\overline{\mu}$ é uma medida.
\end{proof}
}

\newcommand{\muIsUnique}{
    \begin{exercise}{\texorpdfstring{$\overline{\mu}$}{mu} é única}{mu_is_unique}
        Nas condições do Teorema \ref{thm:measure_completion}, mostre que $\overline{\mu}$ é a única medida que estende $\mu$ a uma medida completa.
    \end{exercise}
    \begin{proof}
    Sejam $\overline{\nu}$ e $\overline{\mu}$ extensões de $\mu$ como no Teorema \ref{thm:measure_completion}. 

    Para todo $E\cup F\in \overline{\Sigma}$, temos pela \hyperref[prop:measure_is_monotonic]{monotonicidade} e \hyperref[prop:measure_is_subadditive]{subaditividadde} da medida,
    \begin{equation*}
        \overline{\nu}(E\cup F) \leq \overline{\nu}(E) + \overline{\nu}(F) = \overline{\nu}(E) = \mu(E) = \overline{\mu}(E)\leq \overline{\mu}(E\cup F).
    \end{equation*}

    Analogamente, temos 
    \begin{equation*}
        \overline{\mu}(E\cup F) \leq \overline{\mu}(E) + \overline{\mu}(F) = \overline{\mu}(E) = \mu(E) = \overline{\nu}(E) \leq \overline{\nu}(E\cup F).
    \end{equation*}

    Portanto, $\overline{\nu}(E\cup F) = \overline{\mu}(E\cup F)$, e assim $\overline{\mu}$ é única.
\end{proof}
}

\newcommand{\productOfSetsThatGenerateSigmaAlgebra}{
    \begin{exercise}{\texorpdfstring{$\prod_{i\in I} E_i = \bigcap_{i\in I} \pi_{i}^{-1} (E_i)$}{Produto de E é interseção de projeções}}{product_of_sets_that_generate_sigma_algebra}
        Nas condições da Proposição \ref{prop:product_sigma_algebra_of_countable_family}, temos que, para cada $i\in \mathcal{I}$, $E_i$ é um elemento da família que gera $\Sigma_i$. Mostre que
        \begin{equation*}
            \prod_{i\in I} E_i = \bigcap_{i\in I} \pi_{i}^{-1} (E_i).
        \end{equation*}
    \end{exercise}
    \begin{proof}
    Pela Definição de \nameref{def:cartesian_product},
    \begin{equation*}
        \prod_{i \in \mathcal{I}} E_i = \left\{f:\mathcal{I}\rightarrow \bigcup_{i \in \mathcal{I}} E_i; \forall i \in \mathcal{I}, f(i)\in E_i \right\}.
    \end{equation*}

    Olhemos para o lado esquerdo da equação. Fixado um $i\in\mathcal{I}$, a pré-imagem de uma \nameref{def:projection} é dada por
    \begin{equation*}
        \pi_{i}^{-1}(E_i) = \left\{f\in \prod_{i\in\mathcal{I}}E_i; f(i)\in E_i\right\}.
    \end{equation*}
    Assim,
    \begin{equation*}
        \bigcap_{i \in \mathcal{I}} \pi_{i}^{-1}(E_i) = \left\{f\in \prod_{i\in\mathcal{I}}E_i; \forall i \in \mathcal{I},f(i)\in E_i\right\}.
    \end{equation*}

    Por definição, todo $f\in \bigcap_{i \in \mathcal{I}} \pi_{i}^{-1}(E_i)$ também pertence a $\prod_{i \in \mathcal{I}} E_i$, logo,

    \begin{equation*}
        \prod_{i \in \mathcal{I}} E_i \supseteq \bigcap_{i \in \mathcal{I}} \pi_{i}^{-1}(E_i). 
    \end{equation*}

    Por outro lado, qualquer $f\in \bigcap_{i \in \mathcal{I}} \pi_{i}^{-1}(E_i)$ é uma função $f:\mathcal{I}\rightarrow \bigcup_{i\in\mathcal{I}} E_i$ tal que, para todo $i\in\mathcal{I}$, $f(i)\in E_i$. Logo,

    \begin{equation*}
        \prod_{i \in \mathcal{I}} E_i \subseteq \bigcap_{i \in \mathcal{I}} \pi_{i}^{-1}(E_i). 
    \end{equation*}

    Portanto, 

    \begin{equation*}
        \prod_{i \in \mathcal{I}} E_i = \bigcap_{i \in \mathcal{I}} \pi_{i}^{-1}(E_i). 
    \end{equation*}
\end{proof}
}

\newcommand{\preimageOnCountableFamilyOfSets}{ 
    \begin{exercise}{\texorpdfstring{$ \pi_{i}^{-1} (E_i) =\prod_{j\in I} E_j, \text{onde } j\neq i \Rightarrow E_j=X_j$}{pré-imagem de E é produto}}{preimage_on_countable_family_of_sets}
        Nas condições da Proposição \ref{prop:product_sigma_algebra_of_countable_family}, temos que $\mathcal{I}$ é uma família enumerável. Mostre que
        \begin{equation*}
            \pi_{i}^{-1} (E_i) =\prod_{j\in I} E_j, \text{onde } j\neq i \Rightarrow E_j=X_j.
        \end{equation*}
    \end{exercise}
    \begin{proof}
    
    Pela definição de \nameref{def:cartesian_product} aplicada ao conjunto descrito no enunciado, temos
    \begin{equation*}
        \prod_{i \in \mathcal{I}} E_i = \left\{f:\mathcal{I}\rightarrow \bigcup_{i \in \mathcal{I}} E_i; f(i)\in E_i \text{ e } \forall j\neq i \in \mathcal{I}, f(j)\in X_j\right\}.
    \end{equation*}

    Olhemos para o lado direito da equação. Fixado um $i\in\mathcal{I}$, a pré-imagem de uma \nameref{def:projection} é dada por
    \begin{equation*}
        \pi_{i}^{-1}(E_i) = \left\{f\in \prod_{j\in\mathcal{I}}E_j; f(i)\in E_i\right\}.
    \end{equation*}
    Observe que $\pi_{i}^{-1}(E_i)$ é, por definição, um subconjunto do produto cartesiano dos $E_i$. Seja $f\in\pi_{i}^{-1}(E_i)$. Então, $f(i)\in E_i$ e, para qualquer $j\neq i$, $f(j)\in X_j$ pela definição de \nameref{def:cartesian_product}. Portanto, $f\in \prod_{i \in \mathcal{I}} E_i$ e já temos um dos lados da igualdade que procuramos, $\pi_{i}^{-1}(E_i)\subseteq \prod_{i \in \mathcal{I}} E_i$.

    Para mostrar a outra relação de continência, tome uma $f\in \prod_{i \in \mathcal{I}} E_i$. Vamos tentar mostrar que $f\in \pi_{i}^{-1}(E_i)$. Note bem, $f$ já satisfaz a propriedade de que $f(i)=E_i$. Portanto, $f\in \pi_{i}^{-1}(E_i)$ e $\pi_{i}^{-1}(E_i)\supseteq \prod_{i \in \mathcal{I}} E_i$.
    
    Concluimos, então, que 

    \begin{equation*}
        \pi_{i}^{-1}(E_i) = \prod_{i \in \mathcal{I}} E_i.
    \end{equation*}
\end{proof}
}

\newcommand{\sigmaAlgebraGeneratedByCountableTopologicalBasis}{ 
    \begin{exercise}{A \sigmaAlg gerada por uma base topológica}{sigma_algebra_generated_by_countable_topological_basis}
        Seja $(X,\tau)$ um \nameref{def:topological_space} tal que $\tau$ admite base enumerável, $\mathcal{C}$. Então $\borel{\tau}=\borel{\mathcal{C}}$. Em particular, o resultado vale para espaços separáveis.
    \end{exercise}
    \input{modules/proofs/exercises/sigmaAlgebraGeneratedByCountableTopologicalBasis.tex}
}

\newcommand{\mIsSigmaRing}{
    \begin{exercise}{\texorpdfstring{$\mathfrak{M}$}{m} é \texorpdfstring{$\sigma$}{sigma}-anel}{m_is_sigma_ring}
        Nas condições do Teorema \ref{thm:caratheodory}, mostre que $\mathfrak{M}$ é $\sigma$-anel.
    \end{exercise}
    \begin{proof}
    Precisamos provar que $\mathfrak{M}$ é fechado para a união enumerável e para a diferença de conjuntos.

    \begin{claim}{.}{1}
        $\mathfrak{M}$ é fechado para a união de conjuntos.
    \end{claim}
    \textbf{Demonstração da afirmação \ref{clm:1}:} Se $E_1, E_2 \in \mathfrak{M}$, então $E_1 \cup E_2 \in \mathfrak{M}$. Seja dado $A \in \mathcal{H}$. Usando o fato de que $E_1$ e $E_2$ são $\mu^*$-mensuráveis, obtemos:
    \begin{equation*}
        \mu^*(A) = \mu^*(A \cap E_1) + \mu^*(A \cap E_1^c) = \mu^*(A \cap E_1) + \mu^*(A \cap E_1^c \cap E_2) + \mu^*(A \cap E_1^c \cap E_2^c),
    \end{equation*}
    pois $A \cap (E_1 \cup E_2) = (A \cap E_1) \cup (A \cap E_1^c \cap E_2)$ e portanto:
    \begin{equation*}
        \mu^*(A \cap (E_1 \cup E_2)) \leq \mu^*(A \cap E_1) + \mu^*(A \cap E_1^c \cap E_2).
    \end{equation*}
    Das expressões acima, temos:
    \begin{equation*}
        \mu^*(A) = \mu^*(A \cap E_1) + \mu^*(A \cap E_1^c \cap E_2) + \mu^*(A \cap (E_1 \cup E_2)^c) \geq \mu^*(A \cap (E_1 \cup E_2)) + \mu^*(A \cap (E_1 \cup E_2)^c),
    \end{equation*}
    o que prova que $E_1 \cup E_2 \in \mathfrak{M}$.

    Agora, vamos mostrar que $\mathfrak{M}$ é fechado para a diferença de conjuntos.

    \begin{claim}{.}{4}
        $\mathfrak{M}$ é fechado para a diferença de conjuntos.
    \end{claim}
    \textbf{Demonstração da afirmação \ref{clm:4}:} Primeiramente consideraremos o caso em que um conjunto está contido no outro. Mostraremos que, se $E_1, E_2 \in \mathfrak{M}$ e $E_1 \subset E_2$, então $E_2 \setminus E_1 \in \mathfrak{M}$. Seja dado $A \in \mathcal{H}$. Evidentemente:
    \begin{equation*}
        \mu^*(A \cap (E_2 \setminus E_1)) = \mu^*(A \cap E_2 \cap E_1^c);
    \end{equation*}
    Como $E_1 \subset E_2$, temos $E_2 \setminus E_1 = E_1^c \cap E_2$, e portanto:
    \begin{equation*}
        \mu^*(A \cap (E_2 \setminus E_1)^c) = \mu^*((A \cap E_1) \cup (A \cap E_2^c)) \leq \mu^*(A \cap E_1) + \mu^*(A \cap E_2^c).
    \end{equation*}
    Somando as duas expressões acima, obtemos:
    \begin{equation*}
        \mu^*(A \cap (E_2 \setminus E_1)) + \mu^*(A \cap (E_2 \setminus E_1)^c) \leq \mu^*(A \cap E_1) + \mu^*(A \cap E_1^c \cap E_2) + \mu^*(A \cap E_2^c) = \mu^*(A),
    \end{equation*}
    o que prova que $E_2 \setminus E_1 \in \mathfrak{M}$.
    
    Agora, considere $E_1, E_2 \in \mathfrak{M}$ quaisquer. Pela Afirmação \ref{clm:1} anterior, sabemos que $\mathfrak{M}$ é fechado pela união, ou seja, $E_1 \cup E_2 \in \mathfrak{M}$. Portanto, pelo que acabamos de provar, temos que $E_2 \setminus E_1 \in \mathfrak{M}$.

    \begin{claim}{.}{2}
        Se $E_1, E_2 \in \mathfrak{M}$, $A \in \mathcal{H}$ e $E_1 \cap E_2 = \emptyset$, então:
        \begin{equation*}
            \mu^*(A \cap (E_1 \cup E_2)) = \mu^*(A \cap E_1) + \mu^*(A \cap E_2).
        \end{equation*}
    \end{claim}
    \textbf{Demonstração da afirmação \ref{clm:2}:} Como $A \cap (E_1 \cup E_2) \in \mathcal{H}$ e $E_1 \in \mathfrak{M}$, temos:
    \begin{equation*}
        \mu^*(A \cap (E_1 \cup E_2)) = \mu^*(A \cap (E_1 \cup E_2) \cap E_1) + \mu^*(A \cap (E_1 \cup E_2) \cap E_1^c).
    \end{equation*}
    Como $E_1 \cap E_2 = \emptyset$, a última igualdade implica que:
    \begin{equation*}
        \mu^*(A \cap (E_1 \cup E_2)) = \mu^*(A \cap E_1) + \mu^*(A \cap E_2),
    \end{equation*}
    onde usamos o fato de que $E_1 \cap E_2 = \emptyset$.
    

    \begin{claim}{.}{6}
        Se $(E_k)_{k \geq 1}$ é uma sequência de elementos dois a dois disjuntos de $\mathfrak{M}$, então $\bigcup_{k=1}^{\infty} E_k \in \mathfrak{M}$.
    \end{claim}
    \textbf{Demonstração da afirmação \ref{clm:6}:} Usando indução e as Afirmações \ref{clm:1} e \ref{clm:2}, obtemos que $\bigcup_{k=1}^{t} E_k \in \mathfrak{M}$:
    \begin{equation*}
        \mu^*(A \cap \bigcup_{k=1}^{t} E_k) = \sum_{k=1}^{t} \mu^*(A \cap E_k),
    \end{equation*}
    para todo $A \in \mathcal{H}$ e todo $t \geq 1$; daí:
    \begin{equation*}
        \mu^*(A) = \mu^*(A \cap \bigcup_{k=1}^{t} E_k) + \mu^*(A \cap (\bigcup_{k=1}^{t} E_k)^c) = \left(\sum_{k=1}^{t} \mu^*(A \cap E_k)\right) + \mu^*(A \cap (\bigcup_{k=1}^{t} E_k)^c).
    \end{equation*}
    Como $A \cap (\bigcup_{k=1}^{t} E_k)^c \supseteq A \cap (\bigcup_{k=1}^{\infty} E_k)^c$, temos:
    \begin{equation*}
        \mu^*(A) \geq \left(\sum_{k=1}^{\infty} \mu^*(A \cap E_k)\right) + \mu^*(A \cap (\bigcup_{k=1}^{\infty} E_k)^c),
    \end{equation*}
    fazendo $t \to \infty$:
    \begin{equation*}
        \mu^*(A) \geq \mu^*(A \cap \bigcup_{k=1}^{\infty} E_k) + \mu^*(A \cap (\bigcup_{k=1}^{\infty} E_k)^c) \geq \mu^*(A),
    \end{equation*}
    provando que $\bigcup_{k=1}^{\infty} E_k \in \mathfrak{M}$.
    

    \begin{claim}{.}{7}
        $\mathfrak{M}$ é fechado para a união enumerável.
    \end{claim}
    \textbf{Demonstração da afirmação \ref{clm:7}:} Se $(E_k)_{k\geq 1}$ é uma sequência em $\mathfrak{M}$, então $\bigcup_{k=1}^{\infty} E_k \in \mathfrak{M}$. Para cada $k \geq 1$, seja $F_k = E_k \setminus \bigcup_{i=0}^{k-1} E_i$, onde $E_0 = \emptyset$. Segue das Afirmações \ref{clm:1} e \ref{clm:4} que $F_k \in \mathfrak{M}$, para todo $k \geq 1$. Além disso, os conjuntos $(F_k)_{k\geq 1}$ são dois a dois disjuntos, e $\bigcup_{k=1}^{\infty} E_k = \bigcup_{k=1}^{\infty} F_k$. Segue então da Afirmação \ref{clm:6} que $\bigcup_{k=1}^{\infty} E_k \in \mathfrak{M}$.

    Assim, mostramos que $\mathfrak{M}$ é fechado para a união enumerável (Afirmação \ref{clm:7}) e para a diferença de conjuntos (Afirmação \ref{clm:4}), o que prova que $\mathfrak{M}$ é um $\sigma$-anel.
\end{proof}

}