\newcommand{\hereditaryCollection}{
    \begin{definition}{Família Hereditária}{hereditaryCollection}
        Dizemos que uma família de conjuntos $\mathcal{H}$ é um \textbf{hereditária} quando ela possui todos os subconjuntos dos seus elementos, isto é,
        \begin{equation*}\label{def:hereditary_collection_a}
            A\in \mathcal{H}, B\subset A\Longrightarrow B\in \mathcal{H}.
        \end{equation*}
    \end{definition}
}

\newcommand{\ringOfSets}{
    \begin{definition}{Anel de Conjuntos}{ring_of_sets}
        Dizemos que uma família de conjuntos $\mathcal{R}$ é um \textbf{anel} quando ela é fechada pela união e pela diferença, isto é,
        \begin{enumerate}
            \item[(A1)] Se $A,B \in \mathcal{R} $, então  $A\bigcup B \in \mathcal{R}$.\label{def:ring_of_sets_a_union}
            \item[(A2)] Se $A,B \in \mathcal{R} $, então  $A\setminus B \in \mathcal{R}$.\label{def:ring_of_sets_a_difference}
        \end{enumerate}
    \end{definition}
}

\newcommand{\sigmaRing}{
    \begin{definition}{\texorpdfstring{$\sigma$}{sigma}-anel}{sigma_ring}
        Dizemos que uma família de conjuntos $\Sigma$ é um \textbf{\texorpdfstring{$\sigma$}{sigma}-anel} quando ela é um \nameref{def:ring_of_sets} fechado pela união enumerável, isto é,
        \begin{equation*}\label{def:sigma_ring_a_union}
            \left(A_n\right)_{n=1}^\infty \subset \Sigma \Longrightarrow\bigcup_{n=1}^\infty A_n \in \Sigma.
        \end{equation*}
    \end{definition}
}

\newcommand{\algebraOfSets}{
    \begin{definition}{Álgebra de Conjuntos}{algebra_of_sets}
        Seja $X$ um conjunto não vazio e $\mathcal{A}\subset \powerset{X}$. Dizemos que $\mathcal{A}$ é uma \textbf{Álgebra de Conjuntos} quando $\mathcal{A}$ é um \nameref{def:ring_of_sets} com unidade, isto é, $X\in\mathcal{A}$.
    \end{definition}
}

\newcommand{\sigmaAlgebra}{
    \begin{definition}{\texorpdfstring{$\sigma$}{sigma}-álgebra}{sigma_algebra}
        Seja $X$ um conjunto não vazio e $\Sigma\subset \powerset{X}$. Dizemos que $\Sigma$ é uma \textbf{\texorpdfstring{$\sigma$}{sigma}-álgebra} quando $\Sigma$ é um \nameref{def:sigma_ring} com unidade.
    \end{definition}
}

\newcommand{\measurableSpace}{
    \begin{definition}{Espaço Mensurável}{measurable_space}
        Seja $X$ um conjunto não vazio e $\Sigma$ uma \nameref{def:sigma_algebra} no conjunto $X$. Então o par ordenado $(X,\Sigma)$ é chamado de \textbf{espaço mensurável} e cada elemento de $\Sigma$ é dito um conjunto \textbf{\texorpdfstring{$\Sigma$-mensurável}{sigma-mensurável}}.
    \end{definition}
}

\newcommand{\generatedSigmaAlgebra}{
    \begin{definition}{\texorpdfstring{$\sigma$}{sigma}-álgebra gerada}{generated_sigma_algebra}
        Seja $\mathcal{C} \subset \powerset{X}$ uma família de subconjuntos de $X$. Dizemos que a \nameref{def:sigma_algebra} $\sigma [\mathcal{C}]$ é a \textbf{$\sigma$-álgebra gerada por $A$} quando

        \begin{enumerate}
            \item $\mathcal{C} \subset \sigma[\mathcal{C}]$;
            \item Se $\Sigma$ é uma $\sigma$-álgebra de $X$ tal que $\mathcal{C} \subset \Sigma$, então $\sigma [\mathcal{C}]\subset \Sigma$.
        \end{enumerate}
    \end{definition}
}

\newcommand{\topology}{
    \begin{definition}{Topologia}{topology}
        Seja $X$ um conjunto não vazio e $\tau\subset\powerset{X}$. Dizemos que $\tau$ é uma \textbf{topologia} quando ela contém o conjunto vazio e $X$, é fechada pela união arbitrária e pela interseção finita, isto é,
        \begin{enumerate}
            \item[(A1)] $\varnothing, X\in \tau$.\label{def:topology_sets}
            \item[(A2)] Se $\{O_i\}_{i\in \mathcal{I}}$, então  $\bigcup_{i\in \mathcal{I}} O_i \in \tau$.\label{def:topology_union}
            \item[(A3)] Se $(O_i)_{i=1}^{n} $, então  $\bigcap_{i=1}^{n} O_i \in \tau$.\label{def:topology_intersection}
        \end{enumerate}
    \end{definition}
}

\newcommand{\topologicalSpace}{
    \begin{definition}{Espaço Topológico}{topological_space}
        Seja $X$ um conjunto não vazio e $\tau$ uma \nameref{def:topology} no conjunto $X$. Então o par ordenado $(X,\tau)$ é chamado de \textbf{espaço topológico} e cada elemento de $\tau$ é dito um conjunto \textbf{aberto} em $X$.
    \end{definition}
}

\newcommand{\productTopology}{
    \begin{definition}{Topologia Produto}{product_topology}
        Seja $\{(X_i,\tau_i)\}_{i\in I}$ uma família de \nameref{def:topological_space} indexada por $I$. Considere a família
        \begin{equation*}
            \mathcal{C}\coloneqq \left\{\pi_i^{-1}(U); U\in \tau_i, i\in I\right\}.
        \end{equation*}
        Chamamos de \textbf{topologia produto} em $\Pi_{i_\in I} X_i$ a topologia cuja base é $\mathcal{C}$. Denotaremos esta topologia por $\Pi_{i_\in I}\tau_i$.
    \end{definition}
}

\newcommand{\borelSigmaAlgebra}{
    \begin{definition}{\texorpdfstring{$\sigma$}{sigma}-álgebra de Borel}{borel_sigma_algebra}
        Seja $(X,\tau)$ um \nameref{def:topological_space}. Então $\mathcal{B}(X)=\sigma [\tau]$ é a \textbf{\texorpdfstring{$\sigma$}{sigma}-álgebra de Borel}. Os elementos de $\mathcal{B}(X)$ são chamados de \textbf{boreleanos}.
    \end{definition}
}

\newcommand{\cartesianProduct}{
    \begin{definition}{Produto Cartesiano}{cartesian_product}
        Seja $\{X_i \}_{i \in \mathcal{I}}$ uma família de conjuntos indexada por $\mathcal{I}$. Definimos o \textbf{produto cartesiano} da seguinte forma:
        \begin{equation*}
            \prod_{i \in \mathcal{I}} X_i \coloneqq \left\{f:\mathcal{I}\rightarrow \bigcup_{i \in \mathcal{I}} X_i; \forall i \in \mathcal{I}, f(i)\in X_i \right\}.
        \end{equation*}
    \end{definition}
}

\newcommand{\projection}{
    \begin{definition}{Projeção}{projection}
        Seja $\{X_i\}_{i \in \mathcal{I}}$ uma família de conjuntos indexada por $\mathcal{I}$. Considere a função
        \begin{equation*}
            \pi_{j} : \prod_{i\in \mathcal{I}} X_i \rightarrow X_j
        \end{equation*}
        definida por $\pi_j (f)=f(j)$. Chamamos esta função de \textbf{$j$-ésima projeção}.
    \end{definition}
}

\newcommand{\productSigmaAlgebra}{
    \begin{definition}{\texorpdfstring{$\sigma$}{sigma}-álgebra Produto}{product_sigma_algebra}
        Seja $\{(X_i,\Sigma_i)\}_{i\in I}$ uma família de \nameref{def:measurable_space} indexada por $I$. Considere a família
        \begin{equation*}
            \mathcal{C}\coloneqq \left\{\pi_i^{-1}(E_i); E_i\in \Sigma_i, i\in I\right\}.
        \end{equation*}
        Chamamos $\generateSigmaAlg{\mathcal{C}}$ de \textbf{\texorpdfstring{$\sigma$}{sigma}-álgebra produto} em $\Pi_{i_\in I} X_i$. Denotaremos esta $\sigma$-álgebra por $\bigotimes_{i\in I} \Sigma_i$.
    \end{definition}
}

\newcommand{\countablyAdditiveFunction}{
    \begin{definition}{Função Aditiva Contável}{countably_additive_function}
        Uma função real cujo domínio é uma família de conjuntos, $f:\mathcal{R}\rightarrow \Rextend$, é dita uma \textbf{função aditiva contável} quando satisfaz a seguinte propriedade:
        \begin{equation*}
            \text{Se} \ (A_i)_{i=1}^{\infty}\subset \mathcal{R} \ \text{é dois a dois disjunta, então} \ f\left(\bigcup_{i=1}^{\infty}A_1\right)=\sum_{i=1}^{\infty}f(A_i).
        \end{equation*}
    \end{definition}
}

\newcommand{\countablySubadditiveFunction}{
    \begin{definition}{Função Subaditiva Contável}{countably_subadditive_function}
        Uma função real cujo domínio é uma família de conjuntos, $f:\mathcal{R}\rightarrow \Rextend$, é dita uma \textbf{função subaditiva contável} quando satisfaz a seguinte propriedade:
        \begin{equation*}
            \text{Se} \ (A_i)_{i=1}^{\infty}\subset \mathcal{R} \ \text{, então} \ f\left(\bigcup_{i=1}^{\infty}A_1\right)\leq\sum_{i=1}^{\infty}f(A_i).
        \end{equation*}
    \end{definition}
}

\newcommand{\monotonicFunction}{
    \begin{definition}{Função Monotônica}{monotonic_function}
        Uma função real cujo domínio é uma família de conjuntos, $f:\mathcal{R}\rightarrow \Rextend$, é dita uma \textbf{função monotônica} quando satisfaz a seguinte propriedade:
        \begin{equation*}
            \text{Se} \ A,B\subset \mathcal{R} \ \text{e} \ A\subseteq B \text{, então} \ f(A)\leq f(B).
        \end{equation*}
    \end{definition}
}

\newcommand{\preMeasure}{
    \begin{definition}{Pré-medida}{pre_measure}
        Seja $X$ um conjunto não vazio e $\mathcal{R}\subset \powerset{X}$ um \nameref{def:ring_of_sets}. Dizemos que uma função $\mu_0:\mathcal{R}\rightarrow [0,\infty]$ é uma \textbf{pré-medida} quando $\mu_0(\varnothing)=0$ e ela é uma \nameref{def:countably_additive_function}.
    \end{definition}
}

\newcommand{\measure}{
    \begin{definition}{Medida}{measure}
        Dizemos que uma \nameref{def:pre_measure} é uma \textbf{medida} quando o domínio é uma \nameref{def:sigma_algebra}.
    \end{definition}
}

\newcommand{\measureSpace}{
    \begin{definition}{Espaço de Medida}{measure_space}
        Seja $X$ um conjunto não vazio, $\Sigma$ uma \nameref{def:sigma_algebra} no conjunto $X$ e $\mu$ uma \nameref{def:measure} definida em $\Sigma$. Então a tripla ordenada $(X,\Sigma, \mu)$ é chamada de \textbf{espaço de medida}.
    \end{definition}
}

\newcommand{\nullSet}{
    \begin{definition}{Conjunto de Medida Nula}{null_set}
        Seja $(X,\Sigma, \mu)$ um \nameref{def:measure_space}. Se $N\in \Sigma$ é tal que $\mu (N)=0$, então dizemos que $N$ é um \textbf{conjunto de medida nula}.
    \end{definition}
}

\newcommand{\almostEverywhere}{
    \begin{definition}{Para Quase Todo Ponto}{almost_everywhere}
        Seja $(X,\Sigma, \mu)$ um \nameref{def:measure_space} e $P(x)$ uma propriedade dos elementos de $X$. Se existe um conjunto $N\in\Sigma$ de medida nula tal que $P$ vale para todo $x\in N^{\complement}$, então dizemos que $P$ vale \textbf{para quase todo ponto}, o que abreviaremos para ($\mu$-qtp).
    \end{definition}
}

\newcommand{\completeMeasureSpace}{
    \begin{definition}{Espaço de Medida Completa}{complete_measure}
        Seja $(X,\Sigma, \mu)$ um \nameref{def:measure_space}. Dizemos que ele é um \textbf{espaço de medida completa} quando a família dos \nameref{def:null_set}, $\mathcal{N}_\mu$, é \nameref{def:hereditaryCollection}.
    \end{definition}
}

\newcommand{\outerMeasure}{
    \begin{definition}{Medida Exterior}{outer_measure}
        Seja $\mathcal{H}$ um \nameref{def:ring_of_sets} que é \nameref{def:hereditaryCollection}. Dizemos que uma função  $\mu^{*}:\mathcal{H}\rightarrow [0,\infty]$ é uma \textbf{medida exterior} quando $\mu^{*}(\varnothing)=0$, ela é \nameref{def:monotonic_function} e \nameref{def:countably_subadditive_function} .
    \end{definition}
}

\newcommand{\measurableInOuterMeasure}{
    \begin{definition}{Conjunto \texorpdfstring{$\mu^{*}$}{mu}-mensurável}{measurable_in_outer_measure}
        Seja $\mathcal{H}$ um \nameref{def:sigma_ring} \nameref{def:hereditaryCollection} e $\mu^{*}:\mathcal{H}\rightarrow \left[0,+\infty\right]$ uma \nameref{def:outer_measure} em $\mathcal{H}$. Dizemos que $E\in \mathcal{H}$ é \textbf{$\mu^{*}$-mensurável} quando, para todo $A\in\mathcal{H}$, vale a igualdade:
        \begin{equation*}
            \mu^{*}(A)=\mu^{*}(A\cap E)+\mu^{*}(A\cap \complement(E))
        \end{equation*}
    \end{definition}
}

\newcommand{\finiteMeasure}{
    \begin{definition}{Medida Finita}{finite_measure}
        Seja $\mu$ uma \nameref{def:measure} em um \nameref{def:measure_space} $(X,\Sigma)$. Dizemos que $\mu$ é uma \textbf{medida finita} quando $\mu(X)<\infty$.
    \end{definition}
}

\newcommand{\sigmaFiniteMeasure}{
    \begin{definition}{Medida \texorpdfstring{$\sigma$}{sigma}-finita}{sigma_finite_measure}
        Seja $\mu$ uma \nameref{def:measure} em um \nameref{def:measure_space} $(X,\Sigma)$. Dizemos que $\mu$ é uma \textbf{medida \texorpdfstring{$\sigma$}{sigma}-finita} quando existe $(E_i)_{i=1}^{\infty}$ com $\mu(E_i)<\infty$ para todo $i\geq 1$ tal que $X=\cup_{i=1}^{\infty} E_i$. 
    \end{definition}
}

\newcommand{\measurableFunctions}{
    \begin{definition}{Funções Mensuráveis}{measurable_functions}
        Sejam $(X,\Sigma)$ e $(Y,\Sigma')$ \nameref{def:measurable_space}. Seja $f:(X,\Sigma) \rightarrow (Y,\Sigma')$ uma função tal que, para todo $E\in \Sigma'$, $f^{-1}(E)\in \Sigma$. Então, a função $f:(X,\Sigma) \rightarrow (Y,\Sigma')$ é chamada de \textbf{função mensurável}. 
    \end{definition}
}

\newcommand{\simpleFunctions}{
    \begin{definition}{Funções Simples}{simple_functions}
        Uma função $\varphi:X\rightarrow \R$ é dita \textbf{simples} quando $\varphi(\R)$ é finita.
    \end{definition}
}

\newcommand{\simpleFunctionsSemiStandardRepresentation}{
    \begin{definition}{Representação semi-padrão de uma funções simples}{simple_functions_semi_standard_representation}
        Seja $(E_i)_{i=1}^{n}\subset \Sigma$ uma partição de $X$. Dizemos que uma função simples $\varphi$  está escrita na \textbf{representação semi-padrão} quando existe $\left(a_i\right)_{i=1}^{n}\subset \R$ tal que 
    \begin{equation*}
        \varphi = \sum_{i=1}^{n} a_i \chi_{E_i}.
    \end{equation*}
    \end{definition}
}

\newcommand{\simpleFunctionsStandardRepresentation}{
    \begin{definition}{Representação padrão de uma funções simples}{simple_functions_standard_representation}
        Considere a função simples $\varphi$ escrita na representação semi-padrão.
    \begin{equation*}
        \varphi = \sum_{i=1}^{n} a_i \chi_{E_i}.
    \end{equation*}
    Dizemos que $\varphi$ está na \textbf{representação padrão} quando a sequência $(a_i)$ é dois a dois disjunta.
    \end{definition}
}

\newcommand{\integralOfSimpleFunction}{
    \begin{definition}{Representação padrão de uma funções simples}{integral_of_simple_function}
        Seja $\varphi \in M^{+}(X,\Sigma)$ uma função simples na representação padrão. Então
    \begin{equation*}
        \int_X \varphi \  d\mu = \sum_{i=1}^{n} a_i \mu(E_i).
    \end{equation*}
    \end{definition}
}

\newcommand{\lebesgueIntegralofNonNegativeFunction}{
    \begin{definition}{Integral de Lebesgue de Função Positiva}{lebesgue_integral_non_negative_function}
        Seja $f \in M^{+}(X,\Sigma)$. Então
    \begin{equation*}
        \int_X f \, d\mu = \sup \left\{ \int_X \varphi \, d\mu; \ \varphi \in  M^{+}(X,\Sigma) \text{ e } 0 \leq \varphi \leq f \right\}.
    \end{equation*}
    \end{definition}
}

\newcommand{\lebesgueIntegral}{
    \begin{definition}{Integral de Lebesgue}{lebesgue_integral}
        A integral de uma função mensurável pode ser definida para funções na classe 
    \begin{equation*}
        \mathcal{L}_1(X,\Sigma, \mu) = \left\{f \in M(X,\Sigma); \ \int f^{+} d\mu \text{ e } \int f^{-} d\mu \text{ são limitadas}\right\}.
    \end{equation*}
    A integral de $f$ é então dada por:
    \begin{equation*}
        \int f \, d\mu = \int f^{+} \, d\mu - \int f^{-} \, d\mu.
    \end{equation*}
    \end{definition}
}

\newcommand{\lebesgueIntegralRestricted}{
    \begin{definition}{Integral restrita a um conjunto mensurável}{lebesgue_integral_restricted}
        Seja $f \in M^{+}(X,\Sigma)$ e $E\in \Sigma$. Então
    \begin{equation*}
        \int_E f \, d\mu = \int_X f\chi_E \, d\mu.
    \end{equation*}
    \end{definition}
}

\newcommand{\basisOfRExtend}{
    \begin{definition}{Base de $\Rextend$}{basis_of_r_extend}
        Seja $\tau$ a família dos abertos da topologia usual de $\R$. Então,
        a topologia de $\Rextend$ é a topologia gerada pela família
        \begin{equation*}
            \mathcal{C}=\tau \cup \left\{ [-\infty,a ); a\in\R \right\} \cup \left\{ (a,\infty ]; a\in\R \right\}.
        \end{equation*}
    \end{definition}
}

\newcommand{\semiNorm}{
    \begin{definition}{Semi-norma}{semi_norm}
        Seja $V$ um espaço vetorial real. Uma função $N:V\mapsto\R$ é dita uma \textbf{seminorma} se:
        \begin{enumerate}
            \item $N(v)\geq 0 \quad \forall v\in V$,
            \item $N(\alpha v)=\abs{a} N(v) \quad \forall v\in V, \forall \alpha\in\R$,
            \item $N(u+v)\leq N(u)+N(v) \quad  \forall u,v\in V$.
        \end{enumerate}
    \end{definition}
}

\newcommand{\vectorNorm}{
    \begin{definition}{Norma}{norm}
        Seja $V$ um espaço vetorial real. Uma função $N:V\mapsto\R$ é dita uma \textbf{norma} se é \nameref{def:semi_norm} tal que
        \begin{itemize}
            \item $N(v)=0\Longleftrightarrow v=0$.
        \end{itemize}
    \end{definition}
}

\newcommand{\functionSeminorm}{
    \begin{definition}{Semi-norma \texorpdfstring{$\mathcal{L}_p$}{Lp}}{lp_seminorm}
        Seja $M(X,\Sigma)$ o espaço vetorial das funções reais mensuráveis e $p\in [1,\infty)$. Definimos
        \begin{equation*}
            \norm{f}_{p} \coloneq \left( \int_{X}\abs{f}^p d\mu \right)^\frac{1}{p}
        \end{equation*}
    \end{definition}
}

\newcommand{\spaceLpSeminorm}{
    \begin{definition}{Espaço \texorpdfstring{$\mathcal{L}_p$}{Lp}}{lp_seminorm_space}
        Seja $M(X,\Sigma)$ o espaço vetorial das funções reais mensuráveis e $p\in [1,\infty)$. Definimos
        \begin{equation*}
            \mathcal{L}_p \coloneq \left\{f\in\mathcal{F}(X,\R); \ \norm{f}_p < \infty\right\}.
        \end{equation*}
    \end{definition}
}

\newcommand{\equivalenceRelationLp}{
    \begin{definition}{Relação de Equivalência para o Espaço \texorpdfstring{$L_p$}{Lp}}{equivalence_relation_lp}
        Sejam $(X,\Sigma,\mu)$ um espaço de medida e $f,g : X \rightarrow \R$. Dizemos que $f$ e $g$ são \textbf{equivalentes} (denotado $f \sim g$) se $f=g \ (\mu\text{-qtp})$.
    \end{definition}
}

\newcommand{\equivalenceClassOfFunction}{
    \begin{definition}{Classe de Equivalência de uma Função}{equivalence_class_of_function}
        Seja $(X,\Sigma,\mu)$ um espaço de medida e $f: X \rightarrow \R$ uma função mensurável. Definimos a \textbf{classe de equivalência} de $f$ como
        \begin{equation*}
            [f] \coloneq \left\{ g : X \rightarrow \R; \ g = f \ (\mu\text{-qtp}) \right\}.
        \end{equation*}
    \end{definition}
}

\newcommand{\LpSpace}{
    \begin{definition}{Espaço \texorpdfstring{$L_p$}{Lp}}{lp_space}
        Seja $(X,\Sigma,\mu)$ um espaço de medida. Para $1 \leq p < \infty$, definimos o conjunto 
        \begin{equation*}
            L_p(X,\Sigma,\mu) \coloneq \left\{ [f]; \norm{[f]}_{p}=\norm{f}_p < \infty \right\}.
        \end{equation*}
    \end{definition}
}

\newcommand{\LinftySpace}{
    \begin{definition}{Espaço \texorpdfstring{$L_{\infty}$}{Linfty}}{l_infty_space}
        Seja $\mathcal{L}_{\infty}(X,\Sigma,\mu)$ o espaço vetorial das funções reais limitadas $\mu$-(qtp). Definimos,
        \begin{equation*}
            L_{\infty}(X,\Sigma,\mu) \coloneq \left\{ [f]; f\in \mathcal{L}_{\infty}\right\}.
        \end{equation*}
    \end{definition}
}

\newcommand{\functionInftySeminorm}{
    \begin{definition}{Semi-norma \texorpdfstring{$\mathcal{L}_{\infty}$}{Linfty}}{l_infty_seminorm}
        Seja $\mathcal{L}_{\infty}(X,\Sigma,\mu)$ o espaço vetorial das funções reais limitadas $\mu$-(qtp). Definimos,
        \begin{equation*}
            S_f(N) \coloneq \sup_{x\in X}\left\{ \abs{f(x)}; \ x\not\in N \right\},
        \end{equation*}
        \begin{equation*}
            \norm{f}_{p} \coloneq \inf \{S_f{N}; \ N\in\Sigma, \mu(N)=0\}.
        \end{equation*}
    \end{definition}
}

\newcommand{\banachSpace}{
    \begin{definition}{Espaço de Banach}{banach_space}
        Um \textbf{espaço de Banach} é um espaço vetorial normado $(X, \norm{\cdot})$ que é um espaço métrico completo em relação à distância induzida pela norma. Isso significa que, nestes espaços, toda sequência de Cauchy é convergente.
    \end{definition}
}

\newcommand{\cauchySequenceLp}{
    \begin{definition}{Sequência de Cauchy no Espaço $L_p$}{cauchy_sequence_lp}
        Uma sequência $(f_n)$ em $L_p$ é uma \textbf{sequência de Cauchy} em $L_p$ se,

        \begin{equation*}
            \forall \varepsilon >0, \ \exists N(\varepsilon) \in \N; \ n,m \geq N(\varepsilon) \Longrightarrow \norm{f_m - f_n}_p < \varepsilon.
        \end{equation*}
        
        Uma sequência $(f_n)$ em $L_p$ é \textbf{convergente} para $f$ em $L_p$ se, para todo número positivo $\varepsilon$, existe um $N(\varepsilon)$ tal que, se $n \geq N(\varepsilon)$, então 
        \begin{equation*}
            \norm{f - f_n}_p < \varepsilon.
        \end{equation*}
    \end{definition}
}

\newcommand{\aeConvergence}{
    \begin{definition}{Convergência para quase todo ponto}{ae_convergence}
        Uma sequência $(f_n)$ converge para quase todo ponto ($\mu$-qtp) para $f$ se
        \begin{equation*}
            \forall \varepsilon >0, \ \exists N(\varepsilon,x) \in \N; \ n \geq N(\varepsilon,x) \Longrightarrow \abs{f_n(x) - f}_p < \varepsilon \quad(\mu\text{-qtp}).
        \end{equation*}
    \end{definition}
}

\newcommand{\pointwiseConvergence}{
    \begin{definition}{Convergência pontual}{pointwise_convergence}
        Uma sequência $(f_n)$ converge pontualmente para $f$ se
        \begin{equation*}
            \forall \varepsilon >0, \ \exists N(\varepsilon,x) \in \N; \ n \geq N(\varepsilon,x) \Longrightarrow \abs{f_n(x) - f}_p < \varepsilon
        \end{equation*}
    \end{definition}
}

\newcommand{\uniformConvergence}{
    \begin{definition}{Convergência uniforme}{uniform_convergence}
        Uma sequência $(f_n)$ converge uniformemente para $f$ se
        \begin{equation*}
            \forall \varepsilon >0, \ \exists N(\varepsilon) \in \N; \ n \geq N(\varepsilon) \Longrightarrow \abs{f_n(x) - f}_p < \varepsilon
        \end{equation*}
    \end{definition}
}

\newcommand{\LpConvergence}{
    \begin{definition}{Convergência no espaço \texorpdfstring{$L_p$}{Lp}}{Lp_convergence}
        Uma sequência $(f_n)\subset L_p$ converge para $f\in L_p$ se
        \begin{equation*}
            \forall \varepsilon >0, \ \exists N(\varepsilon) \in \N; \ n \geq N(\varepsilon) \Longrightarrow \norm{f_n - f}_p < \varepsilon
        \end{equation*}
    \end{definition}
}

\newcommand{\measureConvergence}{
    \begin{definition}{Convergência na Medida}{cauchy_convergence}
        Uma sequência $(f_n)$ converge na medida para $f$ se, para todo $\varepsilon > 0$
        \begin{equation*}
            \lim_{n\rightarrow \infty}\mu\left(\{x\in X; \ \abs{f_n(x)-f(x)} \geq \varepsilon\}\right)=0.
        \end{equation*}
    \end{definition}
}

\newcommand{\measureCauchyConvergence}{
    \begin{definition}{Convergênciade Cauchy na Medida}{cauchy_convergence}
        Uma sequência $(f_n)$ converge na medida para $f$ se, para todo $\varepsilon > 0$
        \begin{equation*}
            \mu\left(\{x\in X; \ \abs{f_n(x)-f_m(x)} \geq \varepsilon\}\right)=0
        \end{equation*}
        quando $n$ e $m$ tendem ao infinito.
    \end{definition}
}

\newcommand{\almostUniformConvergence}{
    \begin{definition}{Convergência Quase Uniforme}{almost_uniform_convergence}
        Uma sequência $(f_n)$ de funções mensuráveis é dita \textbf{quase uniformemente convergente} para uma função mensurável $f$ se, para cada $\delta > 0$, existe um conjunto $E_{\delta} \subset X$ com $\mu(E_{\delta}) < \delta$ tal que $(f_n)$ converge uniformemente para $f$ em $X \setminus E_{\delta}$. A sequência $(f_n)$ é dita uma \textbf{sequência de Cauchy quase uniformemente} se, para cada $\delta > 0$, existe um conjunto $E_{\delta} \subset X$ com $\mu(E_{\delta}) < \delta$ tal que $(f_n)$ é uniformemente convergente em $X \setminus E_{\delta}$.
    \end{definition}
}
